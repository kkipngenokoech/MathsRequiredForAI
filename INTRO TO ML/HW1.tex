\documentclass[a3paper,12pt]{extarticle} % Use extarticle for A3 paper size
\usepackage{graphicx} % Include this package for \includegraphics
\usepackage{amsmath}
\usepackage{amssymb} % Include this package for \mathbb
\usepackage[margin=1in]{geometry} % Adjust the margin as needed

\begin{document}

\author{kipngeno koech - bkoech}
\title{Homework 1 - Introduction to Machine Learning for Engineers}   
\maketitle

\medskip

\maketitle

\section{Probability}
\begin{enumerate}
    \item Suppose \( W \) is a Gaussian random variable with distribution \( N(\mu, \sigma^2) \) and \( U \) a uniform random variable over the interval \([a, b]\). Assuming that \( W \) and \( U \) are independent, what is the expected value \( \mathbb{E}[Z] \) and variance \( \text{Var}[Z] \) of \( Z = 3W + 2U \)?

\begin{itemize}
    \item \textbf{Expected value:}
    \[
    \mathbb{E}[Z] = \mathbb{E}[3W + 2U] = 3\mathbb{E}[W] + 2\mathbb{E}[U]
    \]
    Since \( W \) is Gaussian with mean \( \mu \) and \( U \) is uniform over \([a, b]\) with mean \( \frac{a+b}{2} \):
    \[
    \mathbb{E}[Z] = 3\mu + 2\left(\frac{a+b}{2}\right) =\mathbf{ 3\mu + (a + b)}
    \]

    \item \textbf{Variance:}
    if a random variable \(X\), is scaled by a constant \(a\), then the variance of the scaled random variable is \(a^2\) times the variance of the original random variable. Therefore:
    \[
    \text{Var}[Z] = \text{Var}[3W + 2U] = 3^2\text{Var}[W] + 2^2\text{Var}[U]
    \]
    Since \( W \) is Gaussian with variance \( \sigma^2 \) and \( U \) is uniform over \([a, b]\) with variance \( \frac{(b-a)^2}{12} \):
    \[
    \text{Var}[Z] = 9\sigma^2 + 4\left(\frac{(b-a)^2}{12}\right) = 9\sigma^2 + \frac{(b-a)^2}{3}
    \]
\end{itemize}
\item Consider the following joint distribution between the random variable \( X \), which takes values \( T \) or \( F \), and the random variable \( Y \), which takes values \( a \), \( b \), \( c \), or \( d \).

\[
\begin{array}{c|cccc}
P(X, Y) & Y=a & Y=b & Y=c & Y=d \\
\hline
X=T & 0.1 & 0.2 & 0.1 & 0.1 \\
X=F & 0.1 & 0.1 & 0.2 & 0.1 \\
\end{array}
\]

\begin{enumerate}
    \item \textbf{Marginal distribution \( P_Y \):}
    \[
    \begin{aligned}
    \Pr(Y = a) &= \Pr(X = T, Y = a) + \Pr(X = F, Y = a) = 0.1 + 0.1 =\mathbf{ 0.2} \\
    \Pr(Y = b) &= \Pr(X = T, Y = b) + \Pr(X = F, Y = b) = 0.2 + 0.1 = \mathbf{0.3} \\
    \Pr(Y = c) &= \Pr(X = T, Y = c) + \Pr(X = F, Y = c) = 0.1 + 0.2 = \mathbf{0.3} \\
    \Pr(Y = d) &= \Pr(X = T, Y = d) + \Pr(X = F, Y = d) = 0.1 + 0.1 = \mathbf{0.2} \\
    \end{aligned}
    \]

    \item \textbf{Conditional probability \( \Pr(X = T \mid Y \in \{b, c, d\}) \):}
    since we are conditioning on \( Y \) being in the set \(\{b, c, d\}\), we need to find the probability of \( X = T \) and \( Y \) being in the set \(\{b, c, d\}\) and divide it by the probability of \( Y \) being in the set \(\{b, c, d\}\):
    \\ this is the bayes theorem:
    \[
    \Pr(A \mid B) = \frac{\Pr(B \mid A) \Pr(A)}{\Pr(B)} = \frac{\Pr(A \cap B)}{\Pr(B)}
    \]
    \[
    \begin{aligned}
    \Pr(Y \in \{b, c, d\}) &= \Pr(Y = b) + \Pr(Y = c) + \Pr(Y = d) = 0.3 + 0.3 + 0.2 = 0.8 \\
    \Pr(X = T \cap Y \in \{b, c, d\}) &= \Pr(X = T, Y = b) + \Pr(X = T, Y = c) + \Pr(X = T, Y = d) = 0.2 + 0.1 + 0.1 = 0.4 \\
    \Pr(X = T \mid Y \in \{b, c, d\}) &= \frac{\Pr(X = T \cap Y \in \{b, c, d\})}{\Pr(Y \in \{b, c, d\})} = \frac{0.4}{0.8} =\mathbf{ 0.5} \\
    \end{aligned}
    \]
\end{enumerate}
\end{enumerate}
\newpage
\section{Linear Algebra}
\begin{enumerate}
    \item Let \( A_k \in \mathbb{R}^{n \times n} \) for \( k = 1, \ldots, K \) such that \( A_k = A_k^\top \), i.e., each \( A_k \) is a symmetric, \( n \)-dimensional square matrix. Suppose all \( A_k \) have the exact same set of eigenvectors \( u_1, u_2, \ldots, u_n \) with the corresponding eigenvalues \( \alpha_{k1}, \ldots, \alpha_{kn} \) for each \( A_k \). Write down the eigenvectors and their corresponding eigenvalues for the following matrices:
    \begin{enumerate}
        \item \( C = \sum_{k=1}^K A_k \)
        \[
        \text{Eigenvectors: } \mathbf{u_1, u_2, \ldots, u_n}
        \]
        \[
        \text{Eigenvalues: } \mathbf{\sum_{k=1}^K \alpha_{k1}, \sum_{k=1}^K \alpha_{k2}, \ldots, \sum_{k=1}^K \alpha_{kn}}
        \]

        \item \( D = A_i^{-1} A_j A_i \), where \( i \neq j \) and \( i, j \in \{1, 2, \ldots, K\} \). Here we assume \( A_i \) is invertible.
        A matrix \(A\) is similar to a matrix \(B\) if there exists an invertible matrix \(P\) such that \(A = P^{-1}BP\). Similar matrices have the same eigenvalues. Therefore, the eigenvalues of \(D\) are the same as the eigenvalues of \(A_j\).
        \\ Since \(A_j\) has the same eigenvectors as \(A_i\), the eigenvectors of \(D\) are the same as the eigenvectors of \(A_i\).
        \[
        \text{Eigenvectors: } \mathbf{ u_1, u_2, \ldots, u_n}
        \]
        For the eigenvalues:
        \[
        \text{Eigenvalues: } \mathbf{ \alpha_{j1}, \alpha_{j2}, \ldots, \alpha_{jn}}
        \]
        
    \end{enumerate}
\end{enumerate}

\end{document}