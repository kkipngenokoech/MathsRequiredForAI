\documentclass[a3paper,12pt]{extarticle} % Use extarticle for A3 paper size
\usepackage{graphicx} % Include this package for \includegraphics
\usepackage{amsmath}
\usepackage{hyperref}
\usepackage{algorithm}
\usepackage{algpseudocode}
\usepackage{amssymb} % Include this package for \mathbb
\usepackage[margin=1in]{geometry} % Adjust the margin as needed

\begin{document}

\author{kipngeno koech - bkoech}
\title{Homework 1 - Introduction to Machine Learning for Engineers}   
\maketitle

\medskip

\maketitle

\section{Probability}
\begin{enumerate}
    \item Suppose \( W \) is a Gaussian random variable with distribution \( N(\mu, \sigma^2) \) and \( U \) a uniform random variable over the interval \([a, b]\). Assuming that \( W \) and \( U \) are independent, what is the expected value \( \mathbb{E}[Z] \) and variance \( \text{Var}[Z] \) of \( Z = 3W + 2U \)?

\begin{itemize}
    \item \textbf{Expected value:}
    \[
    \mathbb{E}[Z] = \mathbb{E}[3W + 2U] = 3\mathbb{E}[W] + 2\mathbb{E}[U]
    \]
    Since \( W \) is Gaussian with mean \( \mu \) and \( U \) is uniform over \([a, b]\) with mean \( \frac{a+b}{2} \):
    \[
    \mathbb{E}[Z] = 3\mu + 2\left(\frac{a+b}{2}\right) =\mathbf{ 3\mu + (a + b)}
    \]

    \item \textbf{Variance:}
    if a random variable \(X\), is scaled by a constant \(a\), then the variance of the scaled random variable is \(a^2\) times the variance of the original random variable. Therefore:
    \[
    \text{Var}[Z] = \text{Var}[3W + 2U] = 3^2\text{Var}[W] + 2^2\text{Var}[U]
    \]
    Since \( W \) is Gaussian with variance \( \sigma^2 \) and \( U \) is uniform over \([a, b]\) with variance \( \frac{(b-a)^2}{12} \):
    \[
    \text{Var}[Z] = 9\sigma^2 + 4\left(\frac{(b-a)^2}{12}\right) = 9\sigma^2 + \frac{(b-a)^2}{3}
    \]
\end{itemize}
\item Consider the following joint distribution between the random variable \( X \), which takes values \( T \) or \( F \), and the random variable \( Y \), which takes values \( a \), \( b \), \( c \), or \( d \).

\[
\begin{array}{c|cccc}
P(X, Y) & Y=a & Y=b & Y=c & Y=d \\
\hline
X=T & 0.1 & 0.2 & 0.1 & 0.1 \\
X=F & 0.1 & 0.1 & 0.2 & 0.1 \\
\end{array}
\]

\begin{enumerate}
    \item \textbf{Marginal distribution \( P_Y \):}
    \[
    \begin{aligned}
    \Pr(Y = a) &= \Pr(X = T, Y = a) + \Pr(X = F, Y = a) = 0.1 + 0.1 =\mathbf{ 0.2} \\
    \Pr(Y = b) &= \Pr(X = T, Y = b) + \Pr(X = F, Y = b) = 0.2 + 0.1 = \mathbf{0.3} \\
    \Pr(Y = c) &= \Pr(X = T, Y = c) + \Pr(X = F, Y = c) = 0.1 + 0.2 = \mathbf{0.3} \\
    \Pr(Y = d) &= \Pr(X = T, Y = d) + \Pr(X = F, Y = d) = 0.1 + 0.1 = \mathbf{0.2} \\
    \end{aligned}
    \]

    \item \textbf{Conditional probability \( \Pr(X = T \mid Y \in \{b, c, d\}) \):}
    since we are conditioning on \( Y \) being in the set \(\{b, c, d\}\), we need to find the probability of \( X = T \) and \( Y \) being in the set \(\{b, c, d\}\) and divide it by the probability of \( Y \) being in the set \(\{b, c, d\}\):
    \\ this is the bayes theorem:
    \[
    \Pr(A \mid B) = \frac{\Pr(B \mid A) \Pr(A)}{\Pr(B)} = \frac{\Pr(A \cap B)}{\Pr(B)}
    \]
    \[
    \begin{aligned}
    \Pr(Y \in \{b, c, d\}) &= \Pr(Y = b) + \Pr(Y = c) + \Pr(Y = d) = 0.3 + 0.3 + 0.2 = 0.8 \\
    \Pr(X = T \cap Y \in \{b, c, d\}) &= \Pr(X = T, Y = b) + \Pr(X = T, Y = c) + \Pr(X = T, Y = d) = 0.2 + 0.1 + 0.1 = 0.4 \\
    \Pr(X = T \mid Y \in \{b, c, d\}) &= \frac{\Pr(X = T \cap Y \in \{b, c, d\})}{\Pr(Y \in \{b, c, d\})} = \frac{0.4}{0.8} =\mathbf{ 0.5} \\
    \end{aligned}
    \]
\end{enumerate}
\end{enumerate}
\newpage
\section{Linear Algebra}
\begin{enumerate}
    \item Let \( A_k \in \mathbb{R}^{n \times n} \) for \( k = 1, \ldots, K \) such that \( A_k = A_k^\top \), i.e., each \( A_k \) is a symmetric, \( n \)-dimensional square matrix. Suppose all \( A_k \) have the exact same set of eigenvectors \( u_1, u_2, \ldots, u_n \) with the corresponding eigenvalues \( \alpha_{k1}, \ldots, \alpha_{kn} \) for each \( A_k \). Write down the eigenvectors and their corresponding eigenvalues for the following matrices:
    \begin{enumerate}
        \item \( C = \sum_{k=1}^K A_k \)
        \[
        \text{Eigenvectors: } \mathbf{u_1, u_2, \ldots, u_n}
        \]
        \[
        \text{Eigenvalues: } \mathbf{\sum_{k=1}^K \alpha_{k1}, \sum_{k=1}^K \alpha_{k2}, \ldots, \sum_{k=1}^K \alpha_{kn}}
        \]

        \item \( D = A_i^{-1} A_j A_i \), where \( i \neq j \) and \( i, j \in \{1, 2, \ldots, K\} \). Here we assume \( A_i \) is invertible.
        A matrix \(A\) is similar to a matrix \(B\) if there exists an invertible matrix \(P\) such that \(A = P^{-1}BP\). Similar matrices have the same eigenvalues. Therefore, the eigenvalues of \(D\) are the same as the eigenvalues of \(A_j\).
        \\ Since \(A_j\) has the same eigenvectors as \(A_i\), the eigenvectors of \(D\) are the same as the eigenvectors of \(A_i\).
        \[
        \text{Eigenvectors: } \mathbf{ u_1, u_2, \ldots, u_n}
        \]
        For the eigenvalues:
        \[
        \text{Eigenvalues: } \mathbf{ \alpha_{j1}, \alpha_{j2}, \ldots, \alpha_{jn}}
        \]
        
    \end{enumerate}
    \item Let \( A \in \mathbb{R}^{m \times n} \), \( b \in \mathbb{R}^m \) be given, and \(\text{col}(A)\) be the column space of \( A \). For a given value of \( m \), under what conditions on \( b \), \(\text{col}(A)\), and \(\text{rank}(A)\) will the equation \( Ax = b \) have:
    \\ The column space of a matrix \(A\) is the set of all possible linear combinations of the columns of \(A\).:
    \[
    \text{col}(A) = \{ A \mathbf{x} \mid \mathbf{x} \in \mathbb{R}^n \}
    \]
     The rank of a matrix is the dimension of the column space of the matrix.
    \begin{enumerate}
        \item \textbf{No solution:} 
        \[
        \text{The equation } Ax = b \text{ has no solution if } b \notin \text{col}(A). \text{ This means that } b \text{ is not a linear combination of the columns of } A.
        \]
        
        \item \textbf{Exactly one solution:}
        \[
        \text{The equation } Ax = b \text{ has exactly one solution if } b \in \text{col}(A) \text{ and } \text{rank}(A) = n.
        \]
        
        \item \textbf{Infinitely many solutions:}
        \[
        \text{The equation } Ax = b \text{ has infinitely many solutions if } b \in \text{col}(A) \text{ and } \text{rank}(A) < n. 
        \]
    \end{enumerate}
\end{enumerate}
\newpage
\section{Matrix Calculus}
\begin{enumerate}
    \item Find the first derivative of the following functions with respect to \( X \). Before you attempt the questions below, you are encouraged to review the  \href{https://en.wikipedia.org/wiki/Matrix_calculus}{Matrix Calculus Wikipedia page}. Note that in this problem, we follow the convention that the derivative of a scalar function \( f(X) \) with respect to \( X \) should have the same dimension as \( X \).
    \begin{enumerate}
        \item \( f(X) = \text{tr}(XX^\top) \), where \( X \in \mathbb{R}^{n \times n} \) and \(\text{tr}\) is the trace of a square matrix.
        \\ The derivative of the trace of a matrix is the matrix itself:
        \[
        \frac{\partial f}{\partial X} = \mathbf{2X}
        \]

        \item \( f(X) = a^\top X b \), where \( X \in \mathbb{R}^{m \times n} \) and \( a \in \mathbb{R}^m \) and \( b \in \mathbb{R}^n \).
        \\ when you multiply a row vector by a matrix, the derivative is the row vector itself, here we have two row vectors, so the derivative is the outer product of the two row vectors.
        \[
        \frac{\partial f}{\partial X} = \mathbf{ab^\top}
        \]

        \item \( f(X) = \|Xb\|^2 \), where \( X \in \mathbb{R}^{m \times n} \) and \( b \in \mathbb{R}^n \).
        \[
        \frac{\partial f}{\partial X} = \mathbf{2Xbb^\top}
        \]
    \end{enumerate}
\end{enumerate}
\newpage
\section{MLE/MAP}
\begin{enumerate}
    \item Suppose \( x \in \mathbb{R} \) is fixed and given. Assume \( \lambda > 0 \) is a scalar parameter, and \( y \sim \text{Poisson}(\lambda) \), where \(\text{Poisson}(\lambda)\) denotes the Poisson distribution with rate \(\lambda\).

    \begin{enumerate}
        \item \textbf{Maximum likelihood estimation:}
        \begin{enumerate}
            \item Write down the PMF (probability mass function) of \( y \mid \lambda \).
            \[
            P(y \mid \lambda) = \frac{\lambda^y e^{-\lambda}}{y!}, \quad y = 0, 1, 2, \ldots
            \]

            \item Assume \( N \) independent observations \( y_1, y_2, \ldots, y_N \) are drawn, where \( y_n \sim \text{Poisson}(\lambda) \) for \( n = 1, 2, \ldots, N \). Derive the joint PMF \( P(y_1, \ldots, y_N \mid \lambda) \).
            \[
            P(y_1, \ldots, y_N \mid \lambda) = \prod_{n=1}^N P(y_n \mid \lambda) = \prod_{n=1}^N \frac{\lambda^{y_n} e^{-\lambda}}{y_n!} = \frac{\lambda^{\sum_{n=1}^N y_n} e^{-N\lambda}}{\prod_{n=1}^N y_n!}
            \]

            \item Write down the negative log-likelihood function of the joint PMF derived in the previous problem and simplify it into a form that can be minimized. (Hint: remove the terms that do not have \(\lambda\)).
            \[
            \mathcal{L}(\lambda) = -\log P(y_1, \ldots, y_N \mid \lambda) = -\log \left( \frac{\lambda^{\sum_{n=1}^N y_n} e^{-N\lambda}}{\prod_{n=1}^N y_n!} \right)
            \]
            \[
            \mathcal{L}(\lambda) = -\left( \sum_{n=1}^N y_n \log \lambda - N\lambda - \log \left( \prod_{n=1}^N y_n! \right) \right)
            \]
            \[
            \mathcal{L}(\lambda) = -\sum_{n=1}^N y_n \log \lambda + N\lambda + \text{const}
            \]
        \end{enumerate}

        \item \textbf{Maximum-a-posteriori (MAP) estimation:} Suppose \( \lambda \sim \text{Gamma}(\alpha, \beta) \), with PMF:
        \[
        P(\lambda) = \frac{\beta^\alpha}{\Gamma(\alpha)} \lambda^{\alpha-1} e^{-\beta \lambda}, \quad \lambda > 0
        \]
        \begin{enumerate}
            \item Write the joint distribution \( P(y_1, \ldots, y_N, \lambda) \) where each \( y_n \sim \text{Poisson}(\lambda) \) is drawn independently as above.
            \[
            P(y_1, \ldots, y_N, \lambda) = P(y_1, \ldots, y_N \mid \lambda) P(\lambda)
            \]
            \[
            P(y_1, \ldots, y_N, \lambda) = \left( \frac{\lambda^{\sum_{n=1}^N y_n} e^{-N\lambda}}{\prod_{n=1}^N y_n!} \right) \left( \frac{\beta^\alpha}{\Gamma(\alpha)} \lambda^{\alpha-1} e^{-\beta \lambda} \right)
            \]
            \[
            P(y_1, \ldots, y_N, \lambda) = \frac{\beta^\alpha}{\Gamma(\alpha)} \frac{\lambda^{\sum_{n=1}^N y_n + \alpha - 1} e^{-(N + \beta)\lambda}}{\prod_{n=1}^N y_n!}
            \]

            \item Write the negative logarithm of \( P(y_1, \ldots, y_N, \lambda) \) and simplify it into a form that can be minimized to find the MAP estimate \( \hat{\lambda} \). (Hint: remove the terms that do not have \(\lambda\)).
            \[
            \mathcal{L}_{\text{MAP}}(\lambda) = -\log P(y_1, \ldots, y_N, \lambda) = -\log \left( \frac{\beta^\alpha}{\Gamma(\alpha)} \frac{\lambda^{\sum_{n=1}^N y_n + \alpha - 1} e^{-(N + \beta)\lambda}}{\prod_{n=1}^N y_n!} \right)
            \]
            \[
            \mathcal{L}_{\text{MAP}}(\lambda) = -\left( \sum_{n=1}^N y_n + \alpha - 1 \right) \log \lambda + (N + \beta)\lambda + \text{const}
            \]
        \end{enumerate}
    \end{enumerate}
\end{enumerate}
\newpage
\section{Linear Regression with Regularization}
\begin{enumerate}
    \item Consider a dataset with \( N \) samples \((x_i, y_i)\), where:
    \[
    y_i = x_i^\top w + \epsilon_i, \quad \epsilon_i \sim N(0, \sigma^2),
    \]
    and \( w \in \mathbb{R}^d \) is the weight vector, \(\epsilon_i\) is Gaussian noise, and \( x_i \in \mathbb{R}^d \).
    To enforce smoothness in the weights \( w \), we introduce a regularizer based on the differences between adjacent weights. This results in the following optimization problem:
    \[
    \min_w L(w) = \|y - Xw\|_2^2 + \lambda \|w\|_2^2 + \mu \|Dw\|_2^2,
    \]
    where \(\lambda, \mu > 0\) are regularization parameters and \(\|Dw\|_2^2 = \sum_{i=2}^{d-1} (2w_i - w_{i-1} - w_{i+1})^2\).

    \begin{enumerate}
        \item Find \( D \in \mathbb{R}^{(d-2) \times d} \) such that \(\|Dw\|_2^2 = \sum_{i=2}^{d-1} (2w_i - w_{i-1} - w_{i+1})^2\). Write \( D \) explicitly. Note that \(\|\cdot\|_2\) denotes the usual \(\ell_2\), or Euclidean, norm.
        \\ we are tasked with finding a matrix \(D\) such that \(\|Dw\|_2^2 = \sum_{i=2}^{d-1} (2w_i - w_{i-1} - w_{i+1})^2\). 
        \\ if weight vector \(w = [w_1, w_2, \ldots, w_d]^T\), then :
        \[
        Dw = \begin{bmatrix}
        2w_2 - w_1 - w_3 \\
        2w_3 - w_2 - w_4 \\
        \vdots \\
        2w_{d-1} - w_{d-2} - w_d
        \end{bmatrix}
        \]
        From the above, we can see that:
        \[
        D = \begin{bmatrix}
        0 & 2 & -1 & 0 & \ldots & 0 \\
        0 & -1 & 2 & -1 & \ldots & 0 \\
        0 & 0 & -1 & 2 & \ldots & 0 \\
        \vdots & \vdots & \vdots & \vdots & \ddots & \vdots \\
        0 & 0 & \ldots & -1 &2 & 1\\
        0 & \ldots & 0 & 0 & -1 & 2
        \end{bmatrix}_{(d-2) \times d}
        \]
        In the first row, the first element is 0, the second element is 2, and the third element is -1. The rest of the elements are 0 because we only care about the relationship between adjancent weights \(w_i, w_{i-1}, w_{i+1}\). The same applies to the rest of the rows.
        \item Derive the closed-form solution for \( w^* \), the minimizer of \( L(w) \).
        \\ this equation can be divided into three parts:
        \\ Residual sum of errors:
        \[
        \text{RSS loss} = \|y - Xw\|_2^2 = (y - Xw)^\top (y - Xw) = y^\top y - 2w^\top X^\top y + w^\top X^\top Xw
        \]
        \textbf{N/B:}scalars are symmetric, so \(w^\top X^\top y = y^\top Xw\).
        \\ Let us find the derivative of the RSS loss w.r.t \(w\):
        \[
        \frac{\partial \text{RSS loss}}{\partial w} = -2X^\top y + 2X^\top Xw = 0
        \]
        L2 regularization:
        \[
        \text{L2 loss} = \lambda \|w\|_2^2 = \lambda w^\top w
        \]
        to find the derivative of the L2 loss w.r.t \(w\):
        \[
        \frac{\partial \text{L2 loss}}{\partial w} = 2\lambda w
        \]
        Smooth regularization:
        \[
        \text{Smooth loss} = \mu \|Dw\|_2^2 = \mu w^\top D^\top Dw = \mu w^\top D^\top Dw
        \]
        to find the derivative of the smooth loss w.r.t \(w\):
        \[
        \frac{\partial \text{Smooth loss}}{\partial w} = 2\mu D^\top Dw
        \]
        Combining the three derivatives:
        \[
        -2X^\top y + 2X^\top Xw + 2\lambda w + 2\mu D^\top Dw = 0
        \]
        \[
        X^\top Xw + \lambda w + \mu D^\top Dw = X^\top y
        \]
        \[
        (X^\top X + \lambda I + \mu D^\top D)w = X^\top y
        \]
        \[
        w^* =  \mathbf{(X^\top X + \lambda I + \mu D^\top D)^{-1} X^\top y}
        \]


        \item Denote the minimizer of problem (1) by \( w^* \). Consider the same problem without the “smooth” regularization:
        \[
        \min_w L_{\text{ridge}}(w) = \|y - Xw\|_2^2 + \lambda \|w\|_2^2.
        \]
        Derive the closed-form solution for \( w_{\text{ridge}}^* \), the minimizer of \( L_{\text{ridge}}(w) \).
        \\ The loss function is:
        \[
        L_{\text{ridge}}(w) = \|y - Xw\|_2^2 + \lambda \|w\|_2^2
        \]
        The derivative of the loss function w.r.t \(w\) is:
        \[
        \frac{\partial L_{\text{ridge}}(w)}{\partial w} = -2X^\top y + 2X^\top Xw + 2\lambda w = 0
        \]
        \[
        X^\top Xw + \lambda w = X^\top y
        \]
        \[
        (X^\top X + \lambda I)w = X^\top y
        \]
        \[
        w_{\text{ridge}}^* = \mathbf{(X^\top X + \lambda I)^{-1} X^\top y}
        \]
        Let us compare the two solutions:   
        \[
        w^* =  \mathbf{(X^\top X + \lambda I + \mu D^\top D)^{-1} X^\top y}
        \]
        \[
        w_{\text{ridge}}^* = \mathbf{(X^\top X + \lambda I)^{-1} X^\top y}
        \]
        We need to prove that \(w^* \leq  w_{\text{ridge}}^*\) by showing that \((X^\top X + \lambda I + \mu D^\top D)^{-1} \preceq (X^\top X + \lambda I)^{-1}\).:
        \[
        (X^\top X + \lambda I + \mu D^\top D)^{-1} - (X^\top X + \lambda I)^{-1} = (X^\top X + \lambda I)^{-1} \mu D^\top D (X^\top X + \lambda I)^{-1}
        \]
        Since \(D^\top D\) is positive semi-definite, then \((X^\top X + \lambda I + \mu D^\top D)^{-1} \preceq (X^\top X + \lambda I)^{-1}\).
        \\ this shows that:
        \[
        w^* \leq  w_{\text{ridge}}^*
        \]
        \item Briefly (in 1-3 sentences) explain how smooth regularization affects the bias and variance of the learned model \( w \).
        \\
       \\  Smooth regularization increases the bias of the learned model by penalizing large differences between adjacent weights, leading to smoother weight vectors. However, it reduces the variance by preventing overfitting to the noise in the data. This is the trade-off between bias and variance in machine learning models, where increasing smoothness reduces variance but increases bias.
    \end{enumerate}
\end{enumerate}
\newpage
\section{Online Update}
    \begin{enumerate}
        \item In the standard linear regression model, we consider a model for which the observed response variable \( y \) is the prediction \( x^\top w \) perturbed by noise, namely
        \[ 
        y = w^\top x + \epsilon
        \]
        where \( \epsilon \) is a Gaussian random variable with mean 0 and variance \( \sigma^2 \) and \( x, w \in \mathbb{R}^d \). Suppose our training dataset contains \( N \) observations of \( d \)-dimensional features. We denote the corresponding feature matrix by \( X \in \mathbb{R}^{N \times d} \), and the labels associated with each observation are denoted by \( y \in \mathbb{R}^N \). We have shown in the class that the MLE (maximum likelihood estimate) of \( w \) is given by
        \[
        \hat{w} = (X^\top X)^{-1} X^\top y.
        \]
        Suppose that we get a new observation \( (\tilde{x}, \tilde{y}) \) and would like to compute the updated MLE \( \hat{w}_{\text{new}} \) after adding this observation to our dataset. Now \( X_{\text{new}} = \begin{pmatrix} X \\ \tilde{x}^\top \end{pmatrix} \in \mathbb{R}^{(N+1) \times d} \) and \( y_{\text{new}} = \begin{pmatrix} y \\ \tilde{y} \end{pmatrix} \in \mathbb{R}^{N+1} \). If we simply apply \( \hat{w}_{\text{new}} = (X_{\text{new}}^\top X_{\text{new}})^{-1} X_{\text{new}}^\top y_{\text{new}} \), we will need to solve the normal equations for \( \hat{w}_{\text{new}} \) again, which is inefficient. In this problem, we will derive a better algorithm for updating the parameters. The main idea is that the inverse of \( X_{\text{new}}^\top X_{\text{new}} \), which is a low rank correction of \( X^\top X \), can be computed by doing a low rank correction to the inverse of \( X^\top X \).

        Note that \( X_{\text{new}}^\top X_{\text{new}} = X^\top X + \tilde{x} \tilde{x}^\top \). Using the Sherman–Morrison–Woodbury formula, we can show that
        \[
        (X_{\text{new}}^\top X_{\text{new}})^{-1} = (X^\top X)^{-1} - \frac{(X^\top X)^{-1} \tilde{x} \tilde{x}^\top (X^\top X)^{-1}}{1 + \tilde{x}^\top (X^\top X)^{-1} \tilde{x}}.
        \]

        Since \( X_{\text{new}}^\top y_{\text{new}} = X^\top y + \tilde{x} \tilde{y} \), we have
        \[
        \hat{w}_{\text{new}} = (X_{\text{new}}^\top X_{\text{new}})^{-1} X_{\text{new}}^\top y_{\text{new}} = \left( (X^\top X)^{-1} - \frac{(X^\top X)^{-1} \tilde{x} \tilde{x}^\top (X^\top X)^{-1}}{1 + \tilde{x}^\top (X^\top X)^{-1} \tilde{x}} \right) (X^\top y + \tilde{x} \tilde{y}).
        \]

        \begin{enumerate}
            \item Derive the formula for \( \hat{w}_{\text{new}} \) using the Sherman–Morrison formula. Your formula should express \( \hat{w}_{\text{new}} \) explicitly in terms of \( \hat{w} \).
            The \(\hat{w_{new}}\) can be expressed as:
            \[
            \hat{w}_{\text{new}} = \left( (X^\top X)^{-1} - \frac{(X^\top X)^{-1} \tilde{x} \tilde{x}^\top (X^\top X)^{-1}}{1 + \tilde{x}^\top (X^\top X)^{-1} \tilde{x}} \right) (X^\top y + \tilde{x} \tilde{y})
            \]
            Expanding the equation:
            \[
            \hat{w}_{\text{new}} = (X^\top X)^{-1} (X^\top y + \tilde{x} \tilde{y})  - \frac{(X^\top X)^{-1} \tilde{x} \tilde{x}^\top (X^\top X)^{-1} (X^\top y + \tilde{x} \tilde{y})}{1 + \tilde{x}^\top (X^\top X)^{-1} \tilde{x}}
            \]
            \[
            \hat{w}_{\text{new}} = (X^\top X)^{-1} X^\top y + (X^\top X)^{-1} \tilde{x} \tilde{y} - \frac{(X^\top X)^{-1} \tilde{x} \tilde{x}^\top (X^\top X)^{-1} X^\top y + (X^\top X)^{-1} \tilde{x} \tilde{x}^\top (X^\top X)^{-1} \tilde{x} \tilde{y}}{1 + \tilde{x}^\top (X^\top X)^{-1} \tilde{x}}
            \]
            \[
            \hat{w}_{\text{new}} = \hat{w} + (X^\top X)^{-1} \tilde{x} (\tilde{y} - \tilde{x}^\top \hat{w}) - \frac{(X^\top X)^{-1} \tilde{x} \tilde{x}^\top \hat{w} + (X^\top X)^{-1} \tilde{x} \tilde{x}^\top \tilde{x} (\tilde{y} - \tilde{x}^\top \hat{w})}{1 + \tilde{x}^\top (X^\top X)^{-1} \tilde{x}}
            \]
            
           
            \begin{algorithm}
            \caption{Online update}
            \begin{algorithmic}[1]
                \State Receive observation \( (\tilde{x}, \tilde{y}) \).
                \State Set \( \xi \leftarrow K \tilde{x} / (1 + \tilde{x}^\top K \tilde{x}) \).
                \State Set \( \alpha \leftarrow \alpha + \xi (\tilde{y} - \tilde{x}^\top \alpha) \).
                \State Set \( K \leftarrow K - \xi \tilde{x}^\top K \).
            \end{algorithmic}
            \end{algorithm}
            \item Assume that \( n \) new data points arrive in a sequential order and we would like to update \( \hat{w} \) every time a new data point arrives. Consider the following two methods:
            \begin{enumerate}
                \item First, initialize \( \hat{w} \) using the original \( N \) data points. Then run Algorithm 1 once for each new data point.
                \item Directly compute \( \hat{w}_{\text{new}} = (X_{\text{new}}^\top X_{\text{new}})^{-1} X_{\text{new}}^\top y_{\text{new}} \) every time a new data point arrives.
            \end{enumerate}
            Compare the total computation complexity (in terms of number of multiplications) of methods (a) and (b) by showing the computational complexity of initialization plus the computational complexity of \( n \) updates. Your answers should be in terms of \( N \), \( d \), and \( n \). Which method is more efficient?

            \textbf{Solution:}
            \begin{itemize}
                \item Method (a):
                \begin{itemize}
                    \item Initialization: \( \mathcal{O}(Nd^2 + d^3) \)
                    initialization involves computing \( (X^\top X)^{-1} \) and \( X^\top y \).
                    \\ where \( X \in \mathbb{R}^{N \times d} \) and \( y \in \mathbb{R}^N \). so:
                    \[
                    X^\top X = \mathcal{O}(Nd^2), \quad X^\top y = \mathcal{O}(Nd)
                    \]
                    You also need to invert \( X^\top X \) which is \( \mathcal{O}(d^3) \).
                    \\ To multiply \( (X^\top X)^{-1} \) by \( X^\top y \) is \( \mathcal{O}(Nd^2) \).  
                    \item Each update: \( \mathcal{O}(d^2 + d^3) \) for \( n \)-th update
                    \item Total for \( n \) updates: \( \mathcal{O}(Nd^2 + d^3 + nd^2 + nd^3) = \mathcal{O}(Nd^2 + nd^2 + d^3 + nd^3) \)
                \end{itemize}
                \item Method (b):
                \begin{itemize}
                    \item Each update: \( \mathcal{O}((N+k)d^2 + d^3) \) for \( k \)-th update
                    \item Total for \( n \) updates: \( \mathcal{O}(\sum_{k=1}^n ((N+k)d^2 + d^3)) = \mathcal{O}(nNd^2 + nd^3 + \frac{n(n+1)}{2}d^2) = \mathcal{O}(nNd^2 + nd^3 + n^2d^2) \)
                \end{itemize}
            \end{itemize}
            Method (a) is more efficient.
        \end{enumerate}
\end{enumerate}

\end{document}