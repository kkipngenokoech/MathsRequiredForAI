\documentclass[a3paper,12pt]{extarticle} % Use extarticle for A3 paper size
\usepackage{graphicx} % Include this package for \includegraphics
\usepackage{amsmath}
\usepackage{amssymb} % Include this package for \mathbb
\usepackage[margin=1in]{geometry} % Adjust the margin as needed

\begin{document}

\author{kipngeno koech - bkoech}
\title{Homework 0 - Introduction to Probabilistic Graphical Models}   
\maketitle

\medskip

\maketitle

\section{ 3-Dimensional Principal Component Analysis [20 points]}

In this problem, we will perform PCA on 3-dimensional data step by step. We are given three data points:
\begin{align}
\mathbf{x}_1 &= [0, -1, -2]\\
\mathbf{x}_2 &= [1, 1, 1]\\
\mathbf{x}_3 &= [2, 0, 1]
\end{align}

and we want to find 2 principal components of the given data.

\begin{enumerate}
\item First, find the covariance matrix $\mathbf{C}_X = \mathbf{X}^T \mathbf{X}$ where $\mathbf{X} = \begin{bmatrix} \mathbf{x}_1 - \bar{\mathbf{x}} \\ \mathbf{x}_2 - \bar{\mathbf{x}} \\ \mathbf{x}_3 - \bar{\mathbf{x}} \end{bmatrix}$, where $\bar{\mathbf{x}} = \frac{1}{3}(\mathbf{x}_1 + \mathbf{x}_2 + \mathbf{x}_3)$ is the mean of the data samples. Then, find the eigenvalues and the corresponding eigenvectors of $\mathbf{C}_X$. Feel free to use any numerical analysis program such as numpy, e.g., \texttt{numpy.linalg.eig} can be useful. However, you should explain what you inputted into this program.
\\ Finding the mean $\bar{\mathbf{x}}$:
\begin{align}
\bar{\mathbf{x}} &= \frac{1}{3}(\mathbf{x}_1 + \mathbf{x}_2 + \mathbf{x}_3)\\
&= \frac{1}{3}([0, -1, -2] + [1, 1, 1] + [2, 0, 1])\\
&= \frac{1}{3}([3, 0, 0])\\
&= [1, 0, 0]
\end{align}
let us find $\mathbf{X}$:
\begin{align}
\mathbf{X} &= \begin{bmatrix} \mathbf{x}_1 - \bar{\mathbf{x}} \\ \mathbf{x}_2 - \bar{\mathbf{x}} \\ \mathbf{x}_3 - \bar{\mathbf{x}} \end{bmatrix}\\
&= \begin{bmatrix} [0, -1, -2] - [1, 0, 0] \\ [1, 1, 1] - [1, 0, 0] \\ [2, 0, 1] - [1, 0, 0] \end{bmatrix}\\
&= \begin{bmatrix} [-1, -1, -2] \\ [0, 1, 1] \\ [1, 0, 1] \end{bmatrix}
\end{align}
Now, we can find the covariance matrix $\mathbf{C}_X$:
\begin{align}
\mathbf{C}_X &=  \mathbf{X}^T \mathbf{X}\\
&=  \begin{bmatrix} [-1, 0, 1] \\ [-1, 1, 0] \\ [-2, 1, 1] \end{bmatrix} \begin{bmatrix} [-1, -1, -2] \\ [0, 1, 1] \\ [1, 0, 1] \end{bmatrix}\\
&= \begin{bmatrix} 2 & 1 & 3 \\ 1 & 2 & 3 \\ 3 & 3 & 6 \end{bmatrix}
\end{align}
Now, we can find the eigenvalues and eigenvectors of $\mathbf{C}_X$ using numpy:
\begin{verbatim}
import numpy as np
C_X = np.array([[2, 1, 3], [1, 2, 3], [3, 3, 6]])
eigenvalues, eigenvectors = np.linalg.eig(C_X)
print("Eigenvalues:", eigenvalues)
print("Eigenvectors:", eigenvectors)
\end{verbatim}
The output will give us the eigenvalues and eigenvectors of the covariance matrix. The eigenvalues are:
\begin{align}
\lambda_1 &= 9.00\\
\lambda_2 &= 1.00\\
\lambda_3 &= 0.00
\end{align}
The corresponding eigenvectors are:
\begin{align}
    \mathbf{u}_1  &= \begin{pmatrix}
    1\\
    1\\
    2
    \end{pmatrix}
    \mathbf{u}_2  &= \begin{pmatrix}
    -1\\
    1\\
    0
    \end{pmatrix}
    \mathbf{u}_3  &= \begin{pmatrix}
    -1\\
    -1\\
    1
    \end{pmatrix}
\end{align}
I used the numpy function \texttt{numpy.linalg.eig} to compute the eigenvalues and eigenvectors of the covariance matrix. The input to this function was the covariance matrix $\mathbf{C}_X$ that we computed above.   
\item 
Using the result above, find the first two principal components of the given data.
\\ The first two principal components are the eigenvectors corresponding to the two largest eigenvalues. In this case, the first two principal components are:
\begin{align}
\mathbf{u}_1 &= \begin{pmatrix}
1\\
1\\
2
\end{pmatrix} \quad \text{(corresponding to } \lambda_1 = 9.00\text{)}\\
\mathbf{u}_2 &= \begin{pmatrix}
-1\\
1\\
0
\end{pmatrix} \quad \text{(corresponding to } \lambda_2 = 1.00\text{)}
\end{align}

\item
Now we want to represent the data $\mathbf{x}_1, \cdots, \mathbf{x}_3$ using a 2-dimensional subspace instead of a 3-dimensional one. PCA gives us the 2-D plane which minimizes the difference between the original data and the data projected to the 2-dimensional plane. In other words, $\mathbf{x}_i$ can be approximated as:
\begin{align}
\tilde{\mathbf{x}}_i = a_{i1}\mathbf{u}_1 + a_{i2}\mathbf{u}_2 + \bar{\mathbf{x}},
\end{align}

where $\mathbf{u}_1$ and $\mathbf{u}_2$ are the principal components we found in 3.b. Figure 1 gives an example of what this might look like.

\begin{figure}[h]
\centering
\caption{Example of 2-D plane spanned by the first two principal components.}
\end{figure}

Find $a_{i1}, a_{i2}$ for $i = 1, 2, 3$. Then, find the $\tilde{\mathbf{x}}_i$'s and the difference between $\tilde{\mathbf{x}}_i$ and $\mathbf{x}_i$, i.e., $||\tilde{\mathbf{x}}_i - \mathbf{x}_i||_2$ for $i = 1, 2, 3$. (Again, feel free to use any numerical analysis program to get the final answer. But, show your calculation process.)

\end{enumerate}

\end{document}