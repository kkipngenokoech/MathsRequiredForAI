\documentclass[a3paper,12pt]{extarticle} % Use extarticle for A3 paper size
\usepackage{graphicx} % Include this package for \includegraphics
\usepackage{amsmath}
\usepackage{amssymb} % Include this package for \mathbb
\usepackage[margin=1in]{geometry} % Adjust the margin as needed
\usepackage{placeins} % Include this package for \Floatbarrier

\begin{document}

\author{kipngeno koech - bkoech}
\title{Homework 2 - Introduction to Probabilistic Graphical Models}   
\maketitle

\medskip

\maketitle
\section{Conditional Independence}
\begin{enumerate}
    \item (10 points) State True or False, and briefly justify your answer within 3 lines. The statements are either
    direct consequences of theorems in Koller and Friedman (2009, Ch. 3), or have a short proof. In the
    follows, P is a distribution and G is a BN structure.
    \begin{figure}[h!]
        \centering
        \includegraphics[width=0.5\textwidth]{"conditional_independence.png"}
        \caption{Caption for the image}
        \label{fig:example_image}
    \end{figure}
    \item Recall the definitions of local and global independences of G and independences of P.
    \begin{align}
        I_l(G) &= \{(X \perp \text{NonDescendants}_G(X) \mid \text{Parents}_G(X))\} \\
        I(G) &= \{(X \perp Y \mid Z) : \text{d-separated}_G(X, Y \mid Z)\} \\
        I(P) &= \{(X \perp Y \mid Z) : P(X, Y \mid Z) = P(X \mid Z)P(Y \mid Z)\}
    \end{align}
    \begin{enumerate}
        \item[(a)] In Figure \ref{fig:example_image}, relation 1 is true.
        \item[(b)] In Figure \ref{fig:example_image}, relation 2 is true.
        \item[(c)] In Figure \ref{fig:example_image}, relation 3 is true.
        \item[(d)] If G is an I-map for P, then P may have extra conditional independencies than G.
        \item[(e)] Two BN structures $G_1$ and $G_2$ are I-equivalent if they have the same skeleton and the same set of v-structures.
        \item[(f)] If $G_1$ is an I-map of distribution P, and $G_1$ has fewer edges than $G_2$, then $G_2$ is not a minimal I-map of P.
        \item[(g)] The P-map of a distribution, if it exists, is unique.
    \end{enumerate}
\end{enumerate}
\newpage
\section{Exact Inference (Junction Tree a.k.a Clique Tree)}
\begin{enumerate}
    \item Consider the following Bayesian network G:
    \begin{figure}[h!]
        \centering
        \includegraphics[width=0.5\textwidth]{junction_tree.png}
        \caption{Caption for the image}
        \label{fig:example_image}
    \end{figure}
    \item We are going to construct a junction tree T from G. Please sketch the generated objects in each step.
    \begin{enumerate}
        \item (4 points) Moralize G to construct an undirected graph H.
        \item (7 points) Triangulate H to construct a chordal graph H*.
        (Although there are many ways to triangulate a graph, for the ease of grading, please try adding fewest
        additional edges possible.)
        \item (7 points) Construct a cluster graph U where each node is a maximal clique Ci from H* and each edge is the sepset \(S_{i,j} = C_i \cap C_j\) between adjacent cliques \(C_i\) and \(C_j\).
        \item (7 points) The junction tree T is the maximum spanning tree of U.
        (The cluster graph is small enough to calculate maximum spanning tree in one’s head.)
    \end{enumerate}

\end{enumerate}
\newpage
\section{Parameter Estimation (HMM - EM Algorithm)}
\begin{enumerate}
    \item Consider an HMM with $T$ time steps, $M$ discrete states, and $K$-dimensional observations as in Figure 3, where $z_t \in \{0, 1\}^M$, $\sum_{s} z_{ts} = 1$, $x_t \in \mathbb{R}^K$ for $t \in [T]$.
    \begin{figure}[h!]
        \centering
        \includegraphics[width=0.5\textwidth]{hmm.png}
        \caption{Caption for the image}
        \label{fig:example_image}
    \end{figure}
\item The joint distribution factorizes over the graph:
\begin{align}
    p(x_{1:T}, z_{1:T}) &= p(z_1) \prod_{t=2}^{T} p(z_t \mid z_{t-1}) \prod_{t=1}^{T} p(x_t \mid z_t). \tag{4}
\end{align}
Now consider the parameterization of CPDs. Let $\pi \in \mathbb{R}^M$ be the initial state distribution and $A \in \mathbb{R}^{M \times M}$ be the transition matrix. The emission density $f(\cdot)$ is parameterized by $\phi_i$ at state $i$. In other words,
\begin{align}
    p(z_{1i} = 1) &= \pi_i, \quad p(z_1) = \prod_{i=1}^{M} \pi_i^{z_{1i}}, \tag{5} \\
    p(z_{tj} = 1 \mid z_{t-1,i} = 1) &= a_{ij}, \quad p(z_t \mid z_{t-1}) = \prod_{i=1}^{M} \prod_{j=1}^{M} a_{ij}^{z_{t-1,i} z_{tj}}, \quad t = 2, \ldots, T, \tag{6} \\
    p(x_t \mid z_{ti} = 1) &= f(x_t; \phi_i), \quad p(x_t \mid z_t) = \prod_{i=1}^{M} f(x_t; \phi_i)^{z_{ti}}, \quad t = 1, \ldots, T. \tag{7}
\end{align}
Let $\theta = (\pi, A, \{\phi_i\}_{i=1}^{M})$ be the set of parameters of the HMM. Given the empirical distribution $\hat{p}$ of $x_{1:T}$, we would like to find the MLE of $\theta$ by solving the following problem:
\begin{align}
    \max_{\theta} \mathbb{E}_{x_{1:T} \sim \hat{p}} [\log p_{\theta}(x_{1:T})]. \tag{8}
\end{align}
However, the marginal likelihood is intractable due to summation over $M^T$ terms:
\begin{align}
    p_{\theta}(x_{1:T}) = \sum_{z_{1:T}} p_{\theta}(x_{1:T}, z_{1:T}). \tag{9}
\end{align}
An alternative is to use the EM algorithm as we saw in the class.
\begin{enumerate}
    \item (10 points) Show that the EM updates can take the following form:
    \begin{align}
        \theta^* &\leftarrow \arg\max_{\theta} \mathbb{E}_{x_{1:T} \sim \hat{p}} [F(x_{1:T}; \theta)] \tag{10}
    \end{align}
    where
    \begin{align}
        F(x_{1:T}; \theta) := &\sum_{i=1}^{M} \gamma(z_{1i}) \log \pi_i + \sum_{t=2}^{T} \sum_{i=1}^{M} \sum_{j=1}^{M} \xi(z_{t-1,i}, z_{tj}) \log a_{ij} \notag \\
        &+ \sum_{t=1}^{T} \sum_{i=1}^{M} \gamma(z_{ti}) \log f(x_t; \phi_i) \tag{11}
    \end{align}
    and $\gamma$ and $\xi$ are the posterior expectations over current parameters $\hat{\theta}$:
    \begin{align}
        \gamma(z_{ti}) &:= \mathbb{E}_{z_{1:T} \sim p_{\hat{\theta}}(z_{1:T} \mid x_{1:T})} [z_{ti}] = p_{\hat{\theta}}(z_{ti} = 1 \mid x_{1:T}), \quad t = 1, \ldots, T \tag{12} \\
        \xi(z_{t-1,i}, z_{tj}) &:= \mathbb{E}_{z_{1:T} \sim p_{\hat{\theta}}(z_{1:T} \mid x_{1:T})} [z_{t-1,i} z_{tj}] = p_{\hat{\theta}}(z_{t-1,i} z_{tj} = 1 \mid x_{1:T}), \quad t = 2, \ldots, T \tag{13}
    \end{align}
    \item (0 points) (No need to answer.) Suppose $\gamma$ and $\xi$ are given, and we use isotropic Gaussian $x_t \mid z_{ti} = 1 \sim \mathcal{N}(\mu_i, \sigma_i^2 I)$ as the emission distribution. Then the parameter updates have the following closed form:
    \begin{align}
        \pi_i^* &\propto \mathbb{E}_{x_{1:T} \sim \hat{p}} [\gamma(z_{1i})] \tag{14} \\
        a_{ij}^* &\propto \mathbb{E}_{x_{1:T} \sim \hat{p}} \left[ \sum_{t=2}^{T} \xi(z_{t-1,i}, z_{tj}) \right] \tag{15} \\
        \mu_{ik}^* &= \frac{\mathbb{E}_{x_{1:T} \sim \hat{p}} \left[ \sum_{t=1}^{T} \gamma(z_{ti}) x_t \right]}{\mathbb{E}_{x_{1:T} \sim \hat{p}} \left[ \sum_{t=1}^{T} \gamma(z_{ti}) \right]} \tag{16} \\
        \sigma_i^{2*} &= \frac{\mathbb{E}_{x_{1:T} \sim \hat{p}} \left[ \sum_{t=1}^{T} \gamma(z_{ti}) \|x_t - \mu_i\|_2^2 \right]}{\mathbb{E}_{x_{1:T} \sim \hat{p}} \left[ \sum_{t=1}^{T} \gamma(z_{ti}) \right] K} \tag{17}
    \end{align}
    \item (10 points) We will use the belief propagation algorithm (Koller and Friedman, 2009, Alg. 10.2) to perform inference for all marginal queries:
    \begin{align}
        \gamma(z_t) &= p_{\hat{\theta}}(z_t \mid x_{1:T}), \quad t = 1, \ldots, T \tag{18} \\
        \xi(z_{t-1}, z_t) &= p_{\hat{\theta}}(z_{t-1}, z_t \mid x_{1:T}), \quad t = 2, \ldots, T \tag{19}
    \end{align}
    For convenience, the notation $\hat{\theta}$ will be omitted from now on. Derive the following BP updates:
    \begin{align}
        \gamma(z_t) &= \frac{1}{Z(x_{1:T})} \cdot s(z_t) \tag{20} \\
        \xi(z_{t-1}, z_t) &= \frac{1}{Z(x_{1:T})} \cdot c(z_{t-1}, z_t) \tag{21}
    \end{align}
    where
    \begin{align}
        s(z_t) &= \alpha(z_t) \beta(z_t), \quad t = 1, \ldots, T \tag{23} \\
        c(z_{t-1}, z_t) &= p(z_t \mid z_{t-1}) p(x_t \mid z_t) \alpha(z_{t-1}) \beta(z_t), \quad t = 2, \ldots, T \tag{24} \\
        Z(x_{1:T}) &= \sum_{z_t} s(z_t) \tag{25}
    \end{align}
    and
    \begin{align}
        \alpha(z_1) &= p(z_1) p(x_1 \mid z_1) \tag{26} \\
        \alpha(z_t) &= p(x_t \mid z_t) \sum_{z_{t-1}} p(z_t \mid z_{t-1}) \alpha(z_{t-1}), \quad t = 2, \ldots, T \tag{27} \\
        \beta(z_{t-1}) &= \sum_{z_t} p(z_t \mid z_{t-1}) p(x_t \mid z_t) \beta(z_t), \quad t = 2, \ldots, T \tag{28} \\
        \beta(z_T) &= 1 \tag{29}
    \end{align}
    \item (0 points) (No need to answer.) Implemented as above, the $(\alpha, \beta)$-recursion is likely to encounter numerical instability due to repeated multiplication of small values. One way to mitigate the numerical issue is to scale $(\alpha, \beta)$ messages at each step $t$, so that the scaled values are always in some appropriate range, while not affecting the inference result for $(\gamma, \xi)$.

    Recall that the forward message is in fact a joint distribution
    \begin{align}
        \alpha(z_t) = p(x_{1:t}, z_t). \tag{30}
    \end{align}
    Define scaled messages by re-normalizing $\alpha$ w.r.t. $z_t$:
    \begin{align}
        \hat{\alpha}(z_t) &= \frac{1}{Z(x_{1:t})} \cdot \alpha(z_t), \tag{31} \\
        Z(x_{1:t}) &= \sum_{z_t} \alpha(z_t). \tag{32}
    \end{align}
    Furthermore, define
    \begin{align}
        r_1 &:= Z(x_1), \tag{33} \\
        r_t &:= \frac{Z(x_{1:t})}{Z(x_{1:t-1})}, \quad t = 2, \ldots, T. \tag{34}
    \end{align}
    Notice that $Z(x_{1:t}) = r_1 \cdot \ldots \cdot r_t$, hence
    \begin{align}
        \hat{\alpha}(z_t) = \frac{1}{r_1 \cdot \ldots \cdot r_t} \cdot \alpha(z_t). \tag{35}
    \end{align}
    Plugging $\hat{\alpha}$ into forward messages, the new $\hat{\alpha}$-recursion is
    \begin{align}
        \hat{\alpha}(z_1) &= \frac{1}{r_1} \cdot p(z_1)p(x_1 \mid z_1), \tag{36} \\
        \hat{\alpha}(z_t) &= \frac{1}{r_t} \cdot p(x_t \mid z_t) \sum_{z_{t-1}} p(z_t \mid z_{t-1}) \hat{\alpha}(z_{t-1}), \quad t = 2, \ldots, T. \tag{37}
    \end{align}
    Since $\hat{\alpha}$ is normalized, each $r_t$ serves as the normalizing constant:
    \begin{align}
        r_t = \sum_{z_t} \tilde{\alpha}(z_t). \tag{38}
    \end{align}
    Now switch focus to $\beta$. In order to make the inference for $(\gamma, \xi)$ invariant of scaling, $\beta$ has to be scaled in a way that counteracts the scaling on $\alpha$. Plugging $\hat{\alpha}$ into the marginal queries,
    \begin{align}
        \gamma(z_t) &= \frac{1}{Z(x_{1:T})} \cdot r_1 \cdot \ldots \cdot r_t \cdot \hat{\alpha}(z_t) \beta(z_t), \tag{39} \\
        \xi(z_{t-1}, z_t) &= \frac{1}{Z(x_{1:T})} \cdot p(z_t \mid z_{t-1}) p(x_t \mid z_t) \cdot r_1 \cdot \ldots \cdot r_{t-1} \cdot \hat{\alpha}(z_{t-1}) \beta(z_t). \tag{40}
    \end{align}
    Since $Z(x_{1:T}) = r_1 \cdot \ldots \cdot r_T$, a natural scaling scheme for $\beta$ is
    \begin{align}
        \hat{\beta}(z_{t-1}) &= \frac{1}{r_t \cdot \ldots \cdot r_T} \cdot \beta(z_{t-1}), \quad t = 2, \ldots, T, \tag{41} \\
        \hat{\beta}(z_T) &:= \beta(z_T). \tag{42}
    \end{align}
    which simplifies the expression for marginals $(\gamma, \xi)$ to
    \begin{align}
        \gamma(z_t) &= \hat{\alpha}(z_t) \hat{\beta}(z_t), \tag{43} \\
        \xi(z_{t-1}, z_t) &= \frac{1}{r_t} \cdot p(z_t \mid z_{t-1}) p(x_t \mid z_t) \hat{\alpha}(z_{t-1}) \hat{\beta}(z_t). \tag{44}
    \end{align}
    The new $\hat{\beta}$-recursion can be obtained by plugging $\hat{\beta}$ into backward messages:
    \begin{align}
        \hat{\beta}(z_{t-1}) &= \frac{1}{r_t} \cdot \sum_{z_t} p(z_t \mid z_{t-1}) p(x_t \mid z_t) \hat{\beta}(z_t), \quad t = 2, \ldots, T, \tag{45} \\
        \hat{\beta}(z_T) &= 1. \tag{46}
    \end{align}
    In other words, $\hat{\beta}(z_{t-1})$ is scaled by $1/r_t$, the normalizer of $\hat{\alpha}(z_t)$.

    The full algorithm is summarized below.

    \textbf{Algorithm 1: Exact inference for $(\gamma, \xi)$}
    \begin{enumerate}
        \item Scaled forward message for $t = 1$:
        \begin{align}
            \tilde{\alpha}(z_1) &= p(z_1)p(x_1 \mid z_1), \tag{47} \\
            r_1 &= \sum_{z_1} \tilde{\alpha}(z_1), \tag{48} \\
            \hat{\alpha}(z_1) &= \frac{1}{r_1} \cdot \tilde{\alpha}(z_1). \tag{49}
        \end{align}
        \item Scaled forward message for $t = 2, \ldots, T$:
        \begin{align}
            \tilde{\alpha}(z_t) &= p(x_t \mid z_t) \sum_{z_{t-1}} p(z_t \mid z_{t-1}) \hat{\alpha}(z_{t-1}), \tag{50} \\
            r_t &= \sum_{z_t} \tilde{\alpha}(z_t), \tag{51} \\
            \hat{\alpha}(z_t) &= \frac{1}{r_t} \cdot \tilde{\alpha}(z_t). \tag{52}
        \end{align}
        \item Scaled backward message for $t = T + 1$:
        \begin{align}
            \hat{\beta}(z_T) = 1. \tag{53}
        \end{align}
        \item Scaled backward message for $t = T, \ldots, 2$:
        \begin{align}
            \hat{\beta}(z_{t-1}) &= \frac{1}{r_t} \cdot \sum_{z_t} p(z_t \mid z_{t-1}) p(x_t \mid z_t) \hat{\beta}(z_t). \tag{54}
        \end{align}
        \item Singleton marginal for $t = 1, \ldots, T$:
        \begin{align}
            \gamma(z_t) = \hat{\alpha}(z_t) \hat{\beta}(z_t). \tag{55}
        \end{align}
        \item Pairwise marginal for $t = 2, \ldots, T$:
        \begin{align}
            \xi(z_{t-1}, z_t) &= \frac{1}{r_t} \cdot p(z_t \mid z_{t-1}) p(x_t \mid z_t) \hat{\alpha}(z_{t-1}) \hat{\beta}(z_t). \tag{56}
        \end{align}

    \end{enumerate}
        \item (15 points) We will implement the EM algorithm (also known as Baum-Welch algorithm), where the E-step performs exact inference and the M-step updates parameter estimates. Please complete the TODO blocks in the provided template \texttt{baum\_welch.py} and submit it to Gradescope. The template contains a toy problem to play with. The submitted code will be tested against randomly generated problem instances.
\end{enumerate}
\end{enumerate}
\newpage
\section{ D-Separation}
\begin{enumerate}
    \item (5 points) Given that the gray nodes are observed, are the variables in A independent to those in B?
    \begin{figure}[h!]
        \centering
        \includegraphics[width=0.5\textwidth]{dseperation_1.png}
        \label{fig:example_image}
    \end{figure}
    \\\\\\\\\\\\\\\\\\
    \item (5 points) Given that the gray nodes are observed, are the nodes 2 and 3 d-separated?
    \begin{figure}[h!]
        \centering
        \includegraphics[width=0.5\textwidth]{dseperation_2.png}
        \label{fig:example_image}
    \end{figure}
    \item (5 points) Given that the gray nodes are observed, are the nodes 5 and 6 d-separated?
    \begin{figure}[h!]
        \centering
        \includegraphics[width=0.5\textwidth]{dseperation_3.png}
        \label{fig:example_image}
    \end{figure} 
    \item (15 points) Write a python program to check d-separation. Three files have been provided. You have to modify only the \texttt{BN.py} file. The instructions on how to run the code are in the \texttt{check\_dsep.py} file. All the files are provided as a zip file named \texttt{code\_files.zip}.
\end{enumerate}

\end{document}