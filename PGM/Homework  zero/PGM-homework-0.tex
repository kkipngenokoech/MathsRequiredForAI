\documentclass[a3paper,12pt]{extarticle} % Use extarticle for A3 paper size
\usepackage{amsmath}
\usepackage{amssymb} % Include this package for \mathbb
\usepackage[margin=1in]{geometry} % Adjust the margin as needed

\begin{document}

\author{kipngeno koech - bkoech}
\title{Homework 0 - Introduction to Probabilistic Graphical Models}   
\maketitle

\medskip

\maketitle

\section{Probability}
\begin{enumerate}
    \item A fair coin is tossed 10 times. The sample space for each trial is Head, Tail and the trials are independent. What is the probability of having:
    \[
        P(H) = \frac{1}{2}, \quad P(T) = \frac{1}{2}
    \]
    The probability of getting exactly \( k \) heads
    is given by the binomial distribution:
    \[
        P(X = k) = \binom{n}{k} p^k (1-p)^{n-k}
    \]
    where \( n = 10 \) and \( p = \frac{1}{2} \).
    \begin{enumerate}
        \item Zero Tail
        \[
            P(0T) = \binom{10}{0} \left(\frac{1}{2}\right)^0 \left(\frac{1}{2}\right)^{10} = \frac{1}{1024}
        \]
        \item 6 heads
        \[
            P(6H) = \binom{10}{6} \left(\frac{1}{2}\right)^6 \left(\frac{1}{2}\right)^4 = \frac{210}{1024}
        \]
        \item At least three heads
        \[
            P(X \geq 3) = 1 - P(X < 3) = 1 - P(X = 0) - P(X = 1) - P(X = 2) = 1 - \left(\frac{1}{1024} + \frac{10}{1024} + \frac{45}{1024}\right) = \frac{968}{1024}
        \]
        \item At least three Heads given the first trail was a Head!
        \[
            P(X \geq 3 | X_1 = H) = 1 - P(X < 3 | X_1 = H) = 1 - P(X = 0 | X_1 = H) - P(X = 1 | X_1 = H) - P(X = 2 | X_1 = H) = 1 - \left(\frac{1}{512} + \frac{9}{512} + \frac{36}{512}\right) = \frac{466}{512}
        \]
        \[
         = 1 - \left(\frac{1}{512} + \frac{9}{512} + \frac{36}{512}\right) = \frac{466}{512}
        \]
    \end{enumerate}
    \item Assuming the probability that it rains on Monday is 0.45; the probability that it rains on Wednesday is 0.4; and the probability that it rains on Wednesday given that rained on Monday is 0.6. What is the probability that:
    \begin{enumerate}
        \item It rains on both days
        \[
            P(M \cap W) = P(W | M)P(M) = 0.6 \times 0.45 = 0.27
        \]
        \item Rain will come next Monday, given that it has just finished raining today (Wednesday)
        \[
            P(M | W) = \frac{P(W | M)P(M)}{P(W)} = \frac{0.6 \times 0.45}{0.4} = 0.675
        \]
    \end{enumerate}
    \item Let \(X\) denote the outcome of a random experiment with possible values \{-4, -3, -2, -1, 0, 1, 2, 3, 4\} according to the following probability law:
    \[
        P(X = x) = 
    \begin{cases}
        ck^2 & \text{if } x \in \{-4, -3, -2, -1, 0, 1, 2, 3, 4\} \\
        0 & \text{otherwise}
    \end{cases}
    \]
    \begin{enumerate}
        \item What is the value of \(c\)?
        \[
            \sum_{x \in \{-4, -3, -2, -1, 0, 1, 2, 3, 4\}} P(X = x) = 1
        \]
        \[
            \sum_{x \in \{-4, -3, -2, -1, 0, 1, 2, 3, 4\}} ck^2 = 1
        \]
        \[
            c \sum_{x \in \{-4, -3, -2, -1, 0, 1, 2, 3, 4\}} k^2 = 1
        \]
        \[
            c \sum_{k = -4}^{4} k^2 = 1
        \]
        To find the value of \(c\), we can use the formula for the sum of squares of the first \(n\) natural numbers:
        \[
            \sum_{k = 1}^{n} k^2 = \frac{n(n+1)(2n+1)}{6}
        \]
        \[
            \sum_{k = -4}^{4} k^2 = \sum_{k = 1}^{4} k^2 + \sum_{k = 1}^{4} k^2 = \frac{4(4+1)(2 \times 4 + 1)}{6} + \frac{4(4+1)(2 \times 4 + 1)}{6} = 30
        \]
        \[
            c \times 30 = 1
        \]
        \[
            c = \frac{1}{30}
        \]
        \item Compute the expectation and variance of \(X\).
        \[
            E[X] = \sum_{x \in \{-4, -3, -2, -1, 0, 1, 2, 3, 4\}} xP(X = x) = \sum_{x \in \{-4, -3, -2, -1, 0, 1, 2, 3, 4\}} x \times \frac{1}{30}x^2 = \frac{1}{30} \sum_{x \in \{-4, -3, -2, -1, 0, 1, 2, 3, 4\}} x^3
        \]
        \[
            E[X] = \frac{1}{30} \sum_{x \in \{-4, -3, -2, -1, 0, 1, 2, 3, 4\}} x^3 = \frac{1}{30} \times 0 = 0
        \]
        \[
            E[X^2] = \sum_{x \in \{-4, -3, -2, -1, 0, 1, 2, 3, 4\}} x^2P(X = x) = \sum_{x \in \{-4, -3, -2, -1, 0, 1, 2, 3, 4\}} x^2 \times \frac{1}{30}x^2 = \frac{1}{30} \sum_{x \in \{-4, -3, -2, -1, 0, 1, 2, 3, 4\}} x^4
        \]
        \[
            E[X^2] = \frac{1}{30} \sum_{x \in \{-4, -3, -2, -1, 0, 1, 2, 3, 4\}} x^4 = \frac{1}{30} \times 0 = 0
        \]
        \[
            Var[X] = E[X^2] - E[X]^2 = 0 - 0 = 0
        \]
    \end{enumerate}
    \item ou just built a new COVID test with the following properties:
    \begin{itemize}
        \item If a person has COVID, the test is positive with  0.95 probability.
        \item If a person does not have COVID, the test can still be positive with 0.05 probability.
    \end{itemize}
    You are told that a random person has COVID with probability 0.001. You just use your test on a random person and it turns out to be positive. What is the probability that the person really has COVID?
    \[
        P(C) = 0.001, \quad P(\bar{C}) = 0.999
    \]
    \[
        P(T | C) = 0.95, \quad P(T | \bar{C}) = 0.05
    \]
    Bayes theorem:
    \[
        P(C | T) = \frac{P(T | C)P(C)}{P(T)}
    \]
    Probability of a positive test( Total Probability):
    \[
        P(T) = P(T | C)P(C) + P(T | \bar{C})P(\bar{C}) = 0.95 \times 0.001 + 0.05 \times 0.999 = 0.0509
    \]
    so:
    \[
        P(C | T) = \frac{0.95 \times 0.001}{0.0509} = 0.0186
    \]
\end{enumerate}


\end{document}