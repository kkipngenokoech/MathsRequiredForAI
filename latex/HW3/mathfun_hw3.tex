\documentclass[12pt]{extarticle} % Use extarticle for 12pt font size
\usepackage{amsmath}
\usepackage{amssymb} % Include this package for \mathbb
\usepackage[margin=1in]{geometry} % Adjust the margin as needed

\begin{document}

\author{kipngeno koech - bkoech}
\title{Homework 3 - Maths Foundation for Machine Learning}   
\maketitle

\medskip

\maketitle
\begin{center}
    \large \textbf{Question 1: Eigen Values and Eigen Vectors}
\end{center}

Consider the matrix $A = \begin{bmatrix} 9 & 1 & -1\\ -1 & 11 & 1 \\ -2 & 2 & 10 \end{bmatrix}$.
\begin{enumerate}
    \item Obtain the characteristic polynomial of $A$. (Hint: find the determinant of $A - \lambda I$)
    \[
    A - \lambda I = \begin{bmatrix} 9-\lambda & 1 & -1\\ -1 & 11-\lambda & 1 \\ -2 & 2 & 10-\lambda \end{bmatrix}
    \]
    \item From the characteristic polynomial, obtain the 3 eigenvalues of $\lambda_1, \lambda_2, \lambda_3$, such that $\lambda_1 <  \lambda_2 < \lambda_3$.
    \[
    \text{det}(A - \lambda I) = 0
    \]
    \[
    \text{det}\begin{bmatrix} 9-\lambda & 1 & -1\\ -1 & 11-\lambda & 1 \\ -2 & 2 & 10-\lambda \end{bmatrix} = 0
    \]
    \[
    (9-\lambda)((11-\lambda)(10-\lambda) - (2 \times 1)) - 1 ((-1)(10-\lambda) - (-2 \times 1)) -1 ((-1)(2) - (-2)(11-\lambda)) = 0
    \]
    \[
    (9-\lambda)((11-\lambda)(10-\lambda) - 2) - 1 ((-10+\lambda) + 2) -1 (-2 + 22 - 2\lambda) = 0
    \]
    \[
    (9-\lambda)(110 - 21\lambda  + \lambda^2 - 2) - 1 (-8 + \lambda) -1 (20 - 2\lambda) = 0
    \]
    \[
    (9-\lambda)(108 - 21\lambda  + \lambda^2) - 1 (-8 + \lambda) -1 (20 - 2\lambda) = 0
    \]
    \[
    9(108 - 21\lambda  + \lambda^2) - \lambda(108 - 21\lambda  + \lambda^2) + 8 - \lambda - 20 + 2\lambda = 0
    \]
    \[
    972 - 189\lambda  + 9\lambda^2 - 108\lambda + 21\lambda^2 - \lambda^3 + 8 - \lambda - 20 + 2\lambda = 0
    \]
    \[
    -\lambda^3 + 30\lambda^2 - 296\lambda + 960 = 0
    \]
    \[
    \lambda^3 - 30\lambda^2 + 296\lambda - 960 = 0
    \]
    \[
    (\lambda - 10)(\lambda^2 - 20\lambda + 96) = 0
    \]
    \[
    (\lambda - 10)(\lambda - 8)(\lambda - 12) = 0
    \]
    \[
    \lambda_1 = 8, \lambda_2 = 10, \lambda_3 = 12
    \]
    \item Using Gaussian elimination, find the eigenvectors of $A$ corresponding to the eigenvalues $\lambda_1, \lambda_2, \lambda_3$.
    \\ for eigen value $\lambda_1 = 8$
    \[
    A - 8I = \begin{bmatrix} 1 & 1 & -1\\ -1 & 3 & 1 \\ -2 & 2 & 2 \end{bmatrix}
    \]
    \[
    \begin{bmatrix} 1 & 1 & -1\\ -1 & 3 & 1 \\ -2 & 2 & 2 \end{bmatrix} \begin{bmatrix} x_1 \\ x_2 \\ x_3 \end{bmatrix} = \begin{bmatrix} 0 \\ 0 \\ 0 \end{bmatrix}
    \]
    Augumented matrix:
    \[
    \begin{bmatrix} 1 & 1 & -1 & | & 0\\ -1 & 3 & 1 & | & 0 \\ -2 & 2 & 2 & | & 0 \end{bmatrix} \xrightarrow{R_2 = R_2 + R_1} \begin{bmatrix} 1 & 1 & -1 & | & 0\\ 0 & 4 & 0 & | & 0 \\ -2 & 2 & 2 & | & 0 \end{bmatrix}
    \]
    \[
    \xrightarrow{R_3 = R_3 + 2R_1} \begin{bmatrix} 1 & 1 & -1 & | & 0\\ 0 & 4 & 0 & | & 0 \\ 0 & 4 & 0 & | & 0 \end{bmatrix} \xrightarrow{R_3 = R_3 - R_2} \begin{bmatrix} 1 & 1 & -1 & | & 0\\ 0 & 4 & 0 & | & 0 \\ 0 & 0 & 0 & | & 0 \end{bmatrix}
    \]
    \[
    \xrightarrow{R_2 = \frac{1}{4}R_2} \begin{bmatrix} 1 & 1 & -1 & | & 0\\ 0 & 1 & 0 & | & 0 \\ 0 & 0 & 0 & | & 0 \end{bmatrix} \xrightarrow{R_1 = R_1 - R_2} \begin{bmatrix} 1 & 0 & -1 & | & 0\\ 0 & 1 & 0 & | & 0 \\ 0 & 0 & 0 & | & 0 \end{bmatrix}
    \]
    \[
    x_1 - x_3 = 0 \Rightarrow x_1 = x_3
    \]
    \[
    x_2 = 0
    \]
    Let $x_3 = 1$, then $x_1 = 1$ and $x_2 = 0$. Therefore, the eigenvector corresponding to $\lambda_1 = 8$ is $\begin{bmatrix} 1 \\ 0 \\ 1 \end{bmatrix}$.
    \\ for eigen value $\lambda_2 = 10$
    \[
    A - 10I = \begin{bmatrix} -1 & 1 & -1\\ -1 & 1 & 1 \\ -2 & 2 & 0 \end{bmatrix}
    \]
    \[
    \begin{bmatrix} -1 & 1 & -1\\ -1 & 1 & 1 \\ -2 & 2 & 0 \end{bmatrix} \begin{bmatrix} x_1 \\ x_2 \\ x_3 \end{bmatrix} = \begin{bmatrix} 0 \\ 0 \\ 0 \end{bmatrix}
    \]
    Augumented matrix:
    \[
    \begin{bmatrix} -1 & 1 & -1 & | & 0\\ -1 & 1 & 1 & | & 0 \\ -2 & 2 & 0 & | & 0 \end{bmatrix} \xrightarrow{R_1 = -R_1} \begin{bmatrix} 1 & -1 & 1 & | & 0\\ -1 & 1 & 1 & | & 0 \\ -2 & 2 & 0 & | & 0 \end{bmatrix}
    \]
    \[
    \xrightarrow{R_2 = R_2 + R_1} \begin{bmatrix} 1 & -1 & 1 & | & 0\\ 0 & 0 & 2 & | & 0 \\ -2 & 2 & 0 & | & 0 \end{bmatrix} \xrightarrow{R_3 = R_3 + 2R_1} \begin{bmatrix} 1 & -1 & 1 & | & 0\\ 0 & 0 & 2 & | & 0 \\ 0 & 0 & 2 & | & 0 \end{bmatrix}
    \]
    \[
    \xrightarrow{R_3 = R_3 - R_2} \begin{bmatrix} 1 & -1 & 1 & | & 0\\ 0 & 0 & 2 & | & 0 \\ 0 & 0 & 0 & | & 0 \end{bmatrix} \xrightarrow{R_2 = \frac{1}{2}R_2} \begin{bmatrix} 1 & -1 & 1 & | & 0\\ 0 & 0 & 1 & | & 0 \\ 0 & 0 & 0 & | & 0 \end{bmatrix}
    \]
    \[
    \xrightarrow{R_1 = R_1 + R_2} \begin{bmatrix} 1 & -1 & 0 & | & 0\\ 0 & 0 & 1 & | & 0 \\ 0 & 0 & 0 & | & 0 \end{bmatrix}
    \]
    \[
    x_1 - x_2 = 0, x_3 = 0 \Rightarrow x_1 = x_2
    \]
    Let $x_2 = 1$, then $x_1 = 1$ and $x_3 = 0$. Therefore, the eigenvector corresponding to $\lambda_2 = 10$ is $\begin{bmatrix} 1 \\ 1 \\ 0 \end{bmatrix}$.
    \\ for eigen value $\lambda_3 = 12$
    \[
    A - 12I = \begin{bmatrix} -3 & 1 & -1\\ -1 & -1 & 1 \\ -2 & 2 & -2 \end{bmatrix}
    \]
    \[
    \begin{bmatrix} -3 & 1 & -1\\ -1 & -1 & 1 \\ -2 & 2 & -2 \end{bmatrix} \begin{bmatrix} x_1 \\ x_2 \\ x_3 \end{bmatrix} = \begin{bmatrix} 0 \\ 0 \\ 0 \end{bmatrix}
    \]
    \[
    \begin{bmatrix} -3 & 1 & -1 & | & 0\\ -1 & -1 & 1 & | & 0 \\ -2 & 2 & -2 & | & 0 \end{bmatrix} \xrightarrow{R_1 = -\frac{1}{3}R_1} \begin{bmatrix} 1 & -\frac{1}{3} & \frac{1}{3} & | & 0\\ -1 & -1 & 1 & | & 0 \\ -2 & 2 & -2 & | & 0 \end{bmatrix}
    \]
    \[
    \xrightarrow{R_2 = R_2 + R_1} \begin{bmatrix} 1 & -\frac{1}{3} & \frac{1}{3} & | & 0\\ 0 & -\frac{4}{3} & \frac{4}{3} & | & 0 \\ -2 & 2 & -2 & | & 0 \end{bmatrix} \xrightarrow{R_3 = R_3 + 2R_1} \begin{bmatrix} 1 & -\frac{1}{3} & \frac{1}{3} & | & 0\\ 0 & -\frac{4}{3} & \frac{4}{3} & | & 0 \\ 0 & \frac{4}{3} & -\frac{4}{3} & | & 0 \end{bmatrix}
    \]
    \[
    \xrightarrow{R_3 = R_3 + R_2} \begin{bmatrix} 1 & -\frac{1}{3} & \frac{1}{3} & | & 0\\ 0 & -\frac{4}{3} & \frac{4}{3} & | & 0 \\ 0 & 0 & 0 & | & 0 \end{bmatrix} \xrightarrow{R_2 = -\frac{3}{4}R_2} \begin{bmatrix} 1 & -\frac{1}{3} & \frac{1}{3} & | & 0\\ 0 & 1 & -1 & | & 0 \\ 0 & 0 & 0 & | & 0 \end{bmatrix}
    \]
    \[
    \xrightarrow{R_1 = R_1 + \frac{1}{3}R_2} \begin{bmatrix} 1 & 0 & 0 & | & 0\\ 0 & 1 & -1 & | & 0 \\ 0 & 0 & 0 & | & 0 \end{bmatrix}
    \]
    \[
    x_1 = 0, x_2 = x_3
    \]
    Let $x_2 = 1$, then $x_1 = 0$ and $x_3 = 1$. Therefore, the eigenvector corresponding to $\lambda_3 = 12$ is $\begin{bmatrix} 0 \\ 1 \\ 1 \end{bmatrix}$.
    \item Using np.linalg.eig, compute the eigenvalues and eigenvectors of A
    eigen values:
    \[
    \begin{pmatrix}
        12. & 8. & 10.
    \end{pmatrix}
    \]
    eigen vectors:
    \[
        \begin{bmatrix}
            9.00258517 \times 10^{-16} & -7.07106781 \times 10^{-1} & -7.07106781 \times 10^{-1} \\
            -7.07106781 \times 10^{-1} & -3.36518470 \times 10^{-16} & -7.07106781 \times 10^{-1} \\
            -7.07106781 \times 10^{-1} & -7.07106781 \times 10^{-1} & -5.62025848 \times 10^{-15}
          \end{bmatrix}
    \]
    \item Compare the eigen vectors computed using np.linalg.eig and that obtained using Gaussian elimination in (3) above. Are they the same? If they are the same, explain why this is the case. Otherwise, explain why they are different
    \\ eigen vectors using np.linalg.eig:
    \[
        \begin{pmatrix}
            9.00258517e-16 & -7.07106781e-01 & 7.07106781e-01 \\
            -7.07106781e-01 & -3.36518470e-16 & -7.07106781e-01 \\
             -7.07106781e-01 & -7.07106781e-01 & -5.62025848e-15
        \end{pmatrix}
    \]
    \\ eigen vectors using Gaussian elimination:
    \[
        \begin{pmatrix}
            1 & 0 & 1 \\
            1 & 1 & 0 \\
            0 & 1 & 1
        \end{pmatrix}
    \]
    The eigen vectors are different. This is because the eigen vectors obtained using np.linalg.eig normalizes the eigen vectors to have a norm of 1. The eigen vectors obtained using Gaussian elimination are not normalized.
    \item Now, given a diagonal matrix \(D\), whose diagonal entries are the eigenvalues of \(A\), and a matrix \(P\) whose columns are the eigenvectors of \(A\), use the eigen values you obtained in (2) and the eigenvectors you obtained in (3) to show that \(A\) as \(PDP^{-1}\).
    \[
    D = \begin{bmatrix} 8 & 0 & 0\\ 0 & 10 & 0 \\ 0 & 0 & 12 \end{bmatrix}, P = \begin{bmatrix} 1 & 1 & 0\\ 0 & 1 & 1 \\ 1 & 0 & 1 \end{bmatrix}, A = \begin{bmatrix} 9 & 1 & -1\\ -1 & 11 & 1 \\ -2 & 2 & 10 \end{bmatrix}
    \]
    \[
    PDP^{-1} = \begin{bmatrix} 1 & 1 & 0\\ 0 & 1 & 1 \\ 1 & 0 & 1 \end{bmatrix} \begin{bmatrix} 8 & 0 & 0\\ 0 & 10 & 0 \\ 0 & 0 & 12 \end{bmatrix} \begin{bmatrix} 1 & 1 & 0\\ 0 & 1 & 1 \\ 1 & 0 & 1 \end{bmatrix}^{-1}
    \]
    \[
    P^{-1} = \begin{pmatrix}
        0.5 & -0.5 & 0.5 \\
        0.5 & 0.5 & -0.5 \\
        -0.5 & 0.5 & 0.5
    \end{pmatrix}
    \]
    \[
    PDP^{-1} = \begin{bmatrix} 1 & 1 & 0\\ 0 & 1 & 1 \\ 1 & 0 & 1 \end{bmatrix} \begin{bmatrix} 8 & 0 & 0\\ 0 & 10 & 0 \\ 0 & 0 & 12 \end{bmatrix} \begin{bmatrix} 0.5 & -0.5 & 0.5 \\ 0.5 & 0.5 & -0.5 \\ -0.5 & 0.5 & 0.5 \end{bmatrix}
    \]
    \[
    DP^{-1} = \begin{bmatrix} 8 & 0 & 0\\ 0 & 10 & 0 \\ 0 & 0 & 12 \end{bmatrix}\begin{bmatrix} 0.5 & -0.5 & 0.5 \\ 0.5 & 0.5 & -0.5 \\ -0.5 & 0.5 & 0.5 \end{bmatrix} = \begin{bmatrix} 4 & -4 & 4\\ 5 & 5  & -5 \\ -6 & 6 & 6 \end{bmatrix}
    \]
    \[
    P(DP^{-1}) = \begin{bmatrix} 1 & 1 & 0\\ 0 & 1 & 1 \\ 1 & 0 & 1 \end{bmatrix} \begin{bmatrix} 4 & -4 & 4\\ 5 & 5  & -5 \\ -6 & 6 & 6 \end{bmatrix} = \begin{bmatrix} 9 & 1 & -1\\ -1 & 11 & 1 \\ -2 & 2 & 10 \end{bmatrix}
    \]
    Therefore, \(A = PDP^{-1}\).
    \item Calculate the determinant of A, \(det(A)\):
    \[
    \text{det}(A) = 9((11 \times 10) - (2 \times 1)) - 1((-1 \times 10) - (-2 \times 1)) -1((-1 \times 2) - (11 \times -2))
    \]
    \[
    = 9(110 - 2) - 1(-10 + 2) -1(-2 + 22) = 9(108) - 1(-8) -1(20) = 972 + 8 - 20 = \textbf{960}
    \]
    
    \item Calculate the trace of A, \(tr(A)\):
    \[
    \text{tr}(A) = 9 + 11 + 10 = \textbf{30}
    \]
    \item  confirm the following holds: 
    \\ \(det(A) = \lambda_1 \lambda_2 \lambda_3\)
    \[
    960 = 8 \times 10 \times 12 = 960
    \]
    \(tr(A) = \lambda_1 + \lambda_2 + \lambda_3\)
    \[
    30 = 8 + 10 + 12 = 30
    \]
\end{enumerate}
\newpage
\begin{center}
    \large \textbf{Question 2: Image Compression using SVD}
\end{center}
In class we explored the concept of low rank approximation where we approximated a matrix as the product
of two low rank matrices. With Singular Value Decomposition (SVD), you can approximate a matrix as the
sum of several rank-1 matrices, and this was also demonstrated.
\begin{enumerate}
    \item What are the dimensions of the provided image?
    \[
    \text{Dimensions of the image} = \textbf{902 x 602}
    \]
    \item Obtain the SVD decomposition of the image and visualize the rank-1 matrix corresponding to the
    largest singular value.
    \[
    \text{Largest singular value} = \textbf{in the notebook}
    \]
    \item Add up the rank-1 matrices for the top 3 singular values and visualize your result. What do you
    observe?
    \\ The more I add up the rank-1 matrices, our rank-3 approximated image approaches the original image, more features are covered
    \item Add up the rank-1 matrices for the top 10 singular values and visualize your result. What do you
    observe?
    \\ The more I add up the rank-1 matrices, our rank-10 approximated image approaches the original image, more features are covered
    \item Assuming 1 byte of memory is required for one element of the image matrix, determine the top k rank-1 matrices required to represent the image if a compression ratio of 13:1 is desired. What is the value of k? Add up the rank-1 matrices for the top K singular values and visaulize your results. Remember to show your workings
    \[
        \text{Compression ratio} = \frac{\text{Original size}}{\text{Compressed size}} = 13
        \]
        Original size:
        \[
            \text{Original size} = \text{Number of elements in the image} = 902 \times 602 = \textbf{543004} \text{ bytes}
            \]
            \[
                \text{Compressed size} = \text{Number of elements in the top k rank-1 matrices} = k \times 902 + k \times 602 = 1504k \text{ bytes}
                \]
                \[
                    \frac{543004}{1504k} = 13 = \frac{543004}{13} = 1504k
                    \]
                    \[
                        k = \frac{543004}{13 \times 1504} = \textbf{28}
                        \]
    \end{enumerate}

\newpage

\begin{center}
    \large \textbf{Question 3: Linear Regression}
\end{center}
In linear regression, SVD can help by decomposing the design matrix, identifying dependencies between predictor variables, and enabling regularization techniques like ridge regression. By reducing the rank or
eliminating singular values, SVD assists in stabilizing regression models and improving their predictive accuracy. In this question you will solve a linear regression problem using SVD.
Complete the cells in the provided starter code and answer the following:

\begin{enumerate}
    \item What is the size of the dataset?
    \[
    \text{Size of the dataset} = \textbf{489}
    \]
    \item How many singular values were obtained during the calculation of the co-efficients? Why is this number exact - neither more nor less?
    \\\\ The number of singular values obtained during the calculation of the coefficients is \(\textbf{3}\). This number is exact because it is equal to the number of features in the dataset. 
    \item Given the SVD decomposition of a matrix, \(A = U \Sigma V^T\), show that \(A^{-1} = U\Sigma^{-1}V^T\)
    \[
    A^{-1} = (U \Sigma V^T)^{-1}
    \]
    \[
    A^{-1} = (V^T)^{-1} \Sigma^{-1} U^{-1}
    \]
    \[
    A^{-1} = V \Sigma^{-1} U^T
    \]
    \[
    A^{-1} = U \Sigma^{-1} V^T
    \]
    \item After training the model, what coefficients did you obtain? Interpret the significance of this coefficients.
    \[
        \begin{bmatrix}
            9.50068071 \\
            -1.29862203 \\
            -0.56460349
        \end{bmatrix}
    \] 
    \\ The coefficients obtained are the weights assigned to each feature in the linear regression model. The coefficient of the first feature is 9.50068071, the coefficient of the second feature is -1.29862203, and the coefficient of the third feature is -0.56460349. These coefficients indicate the impact of each feature on the target variable. A positive coefficient indicates that the feature has a positive impact on the target variable, while a negative coefficient indicates that the feature has a negative impact on the target variable. The magnitude of the coefficient indicates the strength of the impact of the feature on the target variable. Therefore, the coefficients obtained provide information on the relationship between the features and the target variable in the linear regression model.
    \item What is the mean squared error of the train split and test split?
    \\\\ Mean squared error for training data:  \(\textbf{1.1849172161729788}\)
    \\ Mean squared error for testing data:  \(\textbf{1.027395075540607}\)

    \item Visualization: complete the visualization tasks in the starter notebook and report your observation.
    \\\\ There is a linear relationship between feature 1 and the true value. As feature 1 increases, the true value also increases. There are some outliers in the data, but the overall trend is linear. 
    \\\\ There is a negative linear relationship between feature 2 and the true value. As feature 2 increases, the true value decreases. The data points are scattered and do not follow a clear linear trend. There are some outliers in the data.
    \\\\ There is a weak linear relationship between feature 3 and the true value. The data points are scattered and do not follow a clear linear trend. There are some outliers in the data.
\end{enumerate}
\newpage
\begin{center}
    \large \textbf{Question 4: PCA \& Eigenfaces}
\end{center}
PCA is used to project high dimensional vectors onto a lower dimensional subspace whose basis vectors
capture the maximum variance in the data of high dimensional vectors.

An image can be seen as a high dimensional vector. For example, a 100x100 image is a 10000 dimensional
vector. PCA can be used to reduce this dimensionality by projecting images onto a lower dimensional
subspace, given a dataset of images.

Eigenfaces is an application of PCA to obtain a lower dimensional subspace onto which face images can
be projected, such that face images can be reconstructed by linearly combining the basis vectors of the
lower dimensional subspace. The basis vectors of this lower dimensional subspace are called the eigenfaces.
Eigenfaces can be applied in face recognition.
[
You can read more about PCA in chapter 10 of the textbook and study this Wikipedia article on Eigenface
]
In this exercise, you will go through the process of implementing a face recognition system using eigenfaces.
You will use faces from the olivetti faces dataset. A starter notebook has been provided for you in the
handout.

\begin{enumerate}
    \item In the starter code, the olivetti faces dataset has been loaded, using a scikit-learn function. A database
    of faces, a list of test faces, and a list of faces of people not in the database have been created.
    \begin{enumerate}
        \item How many face images are in the database? 
        \[
        \text{Number of face images in the database} = \textbf{400}
        \]
        \item How many images are in the list of test faces?
        \[
        \text{Number of face images in the test faces} = \textbf{36}
        \]
        \item How many images are in the list of faces of people not in the database?
        \[
        \text{Number of face images in the faces of people not in the database} = \textbf{40}
        \]
        \item What are the dimensions of each face image?
        \[
        \text{Dimensions of each face image} = \textbf{64 x 64}
        \]
    \end{enumerate}    
    \item Create a matrix T containing all faces in the database, such that each row corresponds to the pixels
    of a person’s face. Note that the height of this matrix should be equal to the number of faces in the
    database. What are the dimensions of T ?
    \[
    \text{Dimensions of matrix T} = \textbf{324 x 4096} 
    \]
    \item Obtain and visualize the mean face
    \[
    \text{Mean face} = \textbf{in the notebook}
    \]
    \item Obtain the covariance matrix, \(S\), of the matrix \(T\)
    \[
        \text{Covariance matrix, S} = \textbf{in the notebook}
    \]
    \item Obtain the eigenfaces of the covariance matrix \(S\). Report the time (in seconds) it took to compute the eigenfaces.
   \[
   \text{Time taken to compute the eigenfaces} = \textbf{1 minute 28 seconds}
   \]
    \item  Why is the process of computing the eigenfaces as done above computationally expensive? Please
    explain.
    \\\\ The process of computing the eigenfaces as done above is computationally expensive because the covariance matrix is of size \(4096 \times 4096\). Computing the eigenfaces involves finding the eigenvectors of the covariance matrix. The eigenvectors are computed by solving the eigenvalue problem, which involves finding the roots of the characteristic equation of the covariance matrix. This process is computationally expensive because it involves finding the roots of a polynomial of degree \(4096\).
    \item Given the mean face \(\bar{x}\), and the matrix \(V\) whose columns are the eigenvectors of \(S\), if \(M\) is a matrix whose rows are equal to the result of subtracting the mean face from the rows of \(T\), show that \(V = M^TU\), where \(U\) is the matrix whose columns are the eigenvectors of \(MM^T\).
    \\\\ There are two ways we can get the covariance matrix: row wise and feature wise:
    \\row wise: \(S = \frac{1}{n} \sum_{i=1}^{n} (x_i - \bar{x})(x_i - \bar{x})^T\) = \(MM^T\)
    \\feature wise: \(S = \frac{1}{n} \sum_{i=1}^{n} (x_i - \bar{x})^T(x_i - \bar{x})\) = \(M^TM\)
    \[
    V = \text{eigenvectors of } S = \text{eigenvectors of } M^TM
    \]
    \(MM^T\) is a covariance matrix, with eigenvectors \(U\)
    \[
    MM^TU = \lambda U
    \]
    multiply both sides by \(M^T\):
    \[
    M^TMM^TU = M^T\lambda U
    \]
    but, \(M^TM = S\), therefore:
    \[
    SM^TU = \lambda M^TU
    \]
    and we know that \(SV=\lambda V\):
    \[
    V = M^TU
    \]
    \item Compute the eigenfaces using \(M^TU\)
    \[
    \text{Eigenfaces using } M^TU = \textbf{in the notebook}
    \]
    \item  Why is obtaining the eigenfaces using \(M^TU\) more computationally efficient than the approach used in (5)? Please explain.
    \\\\ Obtaining the eigenfaces using \(M^TU\) is more computationally efficient than the approach used in (5) because the matrix \(M\) is of size \(324 \times 4096\), which is smaller than the covariance matrix \(S\) of size \(4096 \times 4096\). Computing the eigenvectors of the covariance matrix \(S\) involves finding the roots of a polynomial of degree \(4096\), which is computationally expensive. On the other hand, computing the eigenvectors of the matrix \(MM^T\) involves finding the roots of a polynomial of degree \(400\), which is less computationally expensive. Therefore, obtaining the eigenfaces using \(M^TU\) is more computationally efficient than the approach used in (5).
    \item Observe the numerical values of the eigenfaces obtain using both approaches. Are they the same? If they are the same, explain why? otherwise explain why they are different.
    \\\\ They are not the same. This is because the eigenfaces obtained using the covariance matrix \(S\) are the eigenvectors of the covariance matrix, while the eigenfaces obtained using \(M^TU\) are the eigenvectors of the matrix \(MM^T\). The eigenvectors of the covariance matrix \(S\) and the matrix \(MM^T\) are different because the matrices are different. The eigenvectors of the covariance matrix \(S\) capture the directions of maximum variance in the data, while the eigenvectors of the matrix \(MM^T\) capture the directions of maximum variance in the mean-centered data. Therefore, the eigenfaces obtained using the covariance matrix \(S\) and \(M^TU\) are different.
    \item Now visualize the first eigenface (the eigenface that captures the most variance in faces in the database) obtained using the approach in (5) and (8). Compare the images, they should be similar, why are they similar?
    \[
    \text{First eigenface} = \textbf{in the notebook}
    \]
    The first eigenface obtained using the approach in (5) and (8) are similar. This is because the first eigenface captures the direction of maximum variance in the data. The eigenvectors of the covariance matrix \(S\) and the matrix \(MM^T\) capture the directions of maximum variance in the data. Therefore, the first eigenface obtained using the approach in (5) and (8) are similar.
    \item Using the eigenfaces obtain in (5), select the top eigenfaces that capture 90\% of the variance in the database. How many eigenfaces were selected?
    \[
    \text{Number of eigenfaces selected} = \textbf{62 faces}
    \]
    \item Visualize the top 10 eigenfaces
    \[
    \text{Top 10 eigenfaces} = \textbf{[in the notebook]}
    \]
    
    \item  Given a matrix, \(A\), whose columns are the top eigenfaces selected in (12), show that the projection matrix unto the column space of \(A\) is the identity matrix.
    \[
    A = \text{top eigenfaces} = \textbf{[in the notebook]}
    \]
    \item Given a new face, \(x\), and the mean image \(\bar{x}\), show that the coordinate of \(x\) in the eigenfaces subspace is given by \(A^T(x-\bar{x})\)
    \[
    x = \text{new face}, \bar{x} = \text{mean image}, A = \text{top eigenfaces}
    \]
    \[
    x = \bar{x} + \sum_{i=1}^{62} (A_i^T(x-\bar{x}))A_i
    \]
    \[
    x = \bar{x} + A^T(x-\bar{x})A
    \]
    \[
    x = \bar{x} + A^T(x-\bar{x})
    \]
    \[
    x - \bar{x} = A^T(x-\bar{x})
    \]
    \[
    A^T(x-\bar{x}) = A^T(x-\bar{x})
    \]
    \item Given the coordinate of a new face, we can calculate the distance between this coordinate and the
    coordinates of all faces in the database. The new face is recognized if the minimum distance is less
    than a certain threshold. With the information you now have about eigenfaces, design a face
    recognition system that recognizes faces similar to those in the face database. Identify a
    threshold such that the accuracy of your system is greater than 80\%.
    \[
    \text{Threshold} = \textbf{14.214214214214214}
    \]
    \item For the obtained threshold, Evaluate the face recognition system’s performance using the following
    metrics
    \begin{enumerate}
        \item accuracy
        \[
            \textbf{0.8333333333333334}
        \]
        \item Precision
        \[
            \textbf{1.0}
        \]
        \item Recall
        \[
            \textbf{0.8333333333333334}
        \]
        \item F1-score
        \[
        \textbf{0.9090909090909091}
        \]
    \end{enumerate}
    \item For your face recognition system, since you already know the coordinates of all faces in the database
    with respect to the eigenfaces, you can decide to discard the database of faces and only work with this
    coordinates. This is also done when you decide to include faces of new people to the system, so that
    they can be recognized. Using this information, calculate the compression ratio of your face recognition
    system if you decide to store only the coordinates of faces. Show your workings.
    \[
    \text{Compression ratio} = \frac{\text{Original size}}{\text{Compressed size}}
    \]
    \[
    \text{Original size} = \text{Number of elements in the database} \times \text{Number of eigenfaces}
    \]
    \[
    \text{Compressed size} = \text{Number of elements in the database} \times \text{Number of faces}
    \]
    \[
    \text{Compression ratio} = \frac{400 \times 62}{400 \times 4096} = \textbf{0.01513671875}
    \]
    \item Conduct experiments to assess the impact of different parameters and settings on your face recognition
    system. Consider factors like the number of eigenfaces.
    \[
    \text{Impact of different parameters and settings} = \textbf{[in the notebook]}
    \]
    \item Analyze and interpret the results, discussing the strengths and limitations of the Eigenfaces method
    in your context.
    \\ I ran a simulation to evaluate the performance of the face recognition system with different numbers of eigenfaces. The results show that the accuracy of the system increases as the number of eigenfaces increases. However, the computational cost of the system also increases with the number of eigenfaces. Therefore, there is a trade-off between accuracy and computational cost. The Eigenfaces method is effective in capturing the variations in the data and recognizing faces based on these variations. However, the method has limitations in handling variations in lighting, pose, and facial expressions. The Eigenfaces method is sensitive to variations in the data and may not perform well in the presence of noise or outliers. Overall, the Eigenfaces method is a powerful technique for face recognition, but it has limitations in handling complex variations in the data.
\end{enumerate}

\end{document}