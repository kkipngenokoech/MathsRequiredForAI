\documentclass{article}
\usepackage{amsmath}
\usepackage{ulem}
\usepackage{amssymb}
\usepackage{listings}
\usepackage{forest}


\title{Applied Stochastic Processes}
\author{Kipngeno Koech (bkoech - andrew ID)}
\date{\today}

\begin{document}

\maketitle
\begin{center}
    \large \textbf{Question 1: Probability Desntiy Function}
\end{center}

1. Consider the PDF of a continuous random variable Y that models the duration (in hours) of a
sudden electrical outage in a city, described by the function:
\[g(y) = \lambda^2ye^{-\lambda y} \quad \text{for} \quad y > 0\]

(a) (2 points): Prove that g(y) is a valid probability density function by verifying the normalization
condition:\newline

normalization condition: \(\int_{-\infty}^{\infty}\lambda^2ye^{-\lambda y}dy = 1\)



\begin{center}
    \large \textbf{Question 2: Eigenvalues and Eigenvectors}
\end{center}
You are analyzing a system modeled by a 3x3 state transition matrix, which represents the interaction
of three interconnected features in a neural network, given by:

\[
B =
\begin{pmatrix}
4 & 2 & 1 \\
1 & 3 & 2 \\
0 & 1 & 2
\end{pmatrix}
\]

To get the eigenvectors, you use the formula \(\det(B - \lambda I) = 0\) where \(I\) is the identity matrix.

\section{Eigen Values}
The eigenvalues of the matrix \(B\) are given by the roots of the equation \(\det(B - \lambda I) = 0\).
\(\det(B - \lambda I) = 0\) is given by:

\[
\det\begin{vmatrix}
4-\lambda & 2 & 1 \\
1 & 3-\lambda & 2 \\
0 & 1 & 2-\lambda
\end{vmatrix} = 0
\]

\[(4-\lambda)\uline{((3-\lambda)(2-\lambda) - 2)} - 2(2-\lambda) + 1  = 0\]
\[(4-\lambda)(6 - 5\lambda + \lambda^2 - 2) - \uline{2(2-\lambda)} + 1 = 0\]
\[(4-\lambda)(4 - 5\lambda + \lambda^2) -\uline{4} + 2\lambda + \uline{1} = 0\]
\[\uline{(4-\lambda)(4 - 5\lambda + \lambda^2)} + 2\lambda - 3 = 0\]
\[4(4 - 5\lambda + \lambda^2) - \lambda(4 - 5\lambda + \lambda^2) + 2\lambda - 3 = 0\]
\[16 - 20\lambda + 4\lambda^2 - 4\lambda + 5\lambda^2 - \lambda^3 + 2\lambda - 3 = 0\]
\[-\lambda^3 + 9\lambda^2 - 18\lambda + 13 = 0\]
\[\lambda^3 - 9\lambda^2 + 18\lambda - 13 = 0\]

The roots of the equation \(\lambda^3 - 9\lambda^2 + 18\lambda - 13 = 0\) are the eigenvalues of the matrix \(B\):
\[\lambda_1 \approx 5.33005874, \quad \lambda_2 \approx 2.79836032, \quad \lambda_3 \approx 0.87158094\]

since all the three eigen values are positive, the system is unstable

\section{Modification of the off diagonal elements of the matrix}

let us say the off-diagonal elements of the matrix are modified to be \(0.5\) as follows:
\[\begin{pmatrix}
4 & 0.5 & 0.5 \\
0.5 & 3 & 0.5 \\
0.5 & 0.5 & 2
\end{pmatrix}\]

The eigenvalues of the matrix \(B\) are given by the roots of the equation \(\det(B - \lambda I) = 0\). which is given by:

the matrix \(B\) is given by:
\[\begin{vmatrix}
4-\lambda & 0.5 & 0.5 \\
0.5 & 3-\lambda & 0.5 \\
0.5 & 0.5 & 2-\lambda
\end{vmatrix} = 0\]
\[4-\lambda\uline{((3-\lambda)(2-\lambda) - 0.5^2)} - \uline{0.5((0.5(2-\lambda) - 0.5(0.5)))} + 0.5(\uline{(0.5)(0.5)-0.5(3-\lambda)}) = 0\]
\[(4-\lambda)(6 - 5\lambda + \lambda^2 - 0.5^2) - 0.5(0.75-0.5\lambda) + 0.5(0.5\lambda -1.25) = 0\]
\[4(6 - 5\lambda + \lambda^2 - 0.25) -\lambda(6-5\lambda + \lambda^2 - 0.25) - 0.375 + 0.25\lambda + 0.25\lambda - 1.25 = 0\]
\[24 - 20\lambda + 4\lambda^2 - 1 - 6\lambda + 5\lambda^2 - \lambda^3 + 0.25\lambda - 0.375 + 0.25\lambda - 1.25 = 0\]
\[-\lambda^3 + 9\lambda^2 - 18\lambda + 13 = 0\]

The roots of the equation \(\lambda^3 - 9\lambda^2 + 18\lambda - 13 = 0\) are the eigenvalues of the matrix \(B\):
\[\lambda_1 \approx 5.33005874, \quad \lambda_2 \approx 2.79836032, \quad \lambda_3 \approx 0.87158094\]

the eigen values still remain the same, hence the system is still unstable

\begin{center}
    \large \textbf{Question 3: Markov Chains}
\end{center}

The transition matrix for a Markov chain is given by:
\[Q=\begin{pmatrix}
0.7 & 0.2 & 0.1 \\
0.3 & 0.4 & 0.3 \\
0.2 & 0.5 & 0.3
\end{pmatrix}\]

It is modelling the transition of a system between three states high performance, medium performance, low performance, for employees respectively.


initial state is given by: \(\pi_0 = \begin{pmatrix} 0.5 & 0.3 & 0.2 \end{pmatrix}\)

\section{find the state of the system after 1 step}
The state of the system after 1 step is given by:
\[\pi_1 = \pi_0Q\]
\[\pi_1 = \begin{pmatrix} 0.5 & 0.3 & 0.2 \end{pmatrix}\begin{pmatrix}0.7 & 0.2 & 0.1 \\ 0.3 & 0.4 & 0.3 \\ 0.2 & 0.5 & 0.3\end{pmatrix}\]
\[\pi_1 = \begin{pmatrix}
0.5(0.7) + 0.3(0.3) + 0.2(0.2) & 0.5(0.2) + 0.3(0.4) + 0.2(0.5) & 0.5(0.1) + 0.3(0.3) + 0.2(0.3)
\end{pmatrix}\]

\[\pi_1 = \begin{pmatrix}0.48 & 0.32 & 0.2\end{pmatrix}\]

\section{steady state distribution}
The steady state distribution is given by:
\[\pi Q= \pi\]
\[\begin{pmatrix}
\pi_a & \pi_b & \pi_c
\end{pmatrix}\begin{pmatrix}07 & 0.2 & 0.1 \\ 0.3 & 0.4 & 0.3 \\ 0.2 & 0.5 & 0.3\end{pmatrix} = \begin{pmatrix} \pi_a & \pi_b & \pi_c \end{pmatrix}\]
also:
\[\pi_a + \pi_b + \pi_c = 1 -\text{ since the sum of the probabilities must be 1 }\]
The  above matrix equation can be written as:
\[\pi_a(0.7) + \pi_b(0.3) + \pi_c(0.2) = \pi_a \ldots \text{equation 1}\]
\[\pi_a(0.2) + \pi_b(0.4) + \pi_c(0.5) = \pi_b \ldots\text{equation 2}\]
\[\pi_a(0.1) + \pi_b(0.3) + \pi_c(0.3) = \pi_c \ldots\text{equation 3}\]
\[\pi_a + \pi_b + \pi_c = 1 \ldots\text{equation 4}\]

with this system of equations, we can solve for the steady state distribution. The solution is given by:

step 1: we can rewrite the equations as:
\[\pi_a(0.7 - 1) + \pi_b(0.3) + \pi_c(0.2) = 0 \ldots\text{equation 1}\]
\[\pi_a(0.2) + \pi_b(0.4 - 1) + \pi_c(0.5) = 0 \ldots\text{equation 2}\]
\[\pi_a(0.1) + \pi_b(0.3) + \pi_c(0.3 - 1) = 0 \ldots\text{equation 3}\]
\[\pi_a + \pi_b + \pi_c = 1 \ldots\text{equation 4}\]

this is equivalent to:
\[-(0.3)\pi_a + (0.3)\pi_b + (0.2)\pi_c = 0 \ldots\text{equation 1}\]
\[0.2\pi_a + -(0.3)\pi_b + (0.5)\pi_c = 0 \ldots\text{equation 2}\]
\[0.1\pi_a + (0.3)\pi_b + -(0.7)\pi_c = 0 \ldots\text{equation 3}\]

multiply equation 1 by 10, equation 2 by 10 and equation 3 by 10:
\[-3\pi_a + 3\pi_b + 2\pi_c = 0 \ldots\text{equation 1}\]
\[2\pi_a - 3\pi_b + 5\pi_c = 0 \ldots\text{equation 2}\]
\[\pi_a + 3\pi_b - 7\pi_c = 0 \ldots\text{equation 3}\]
\[\pi_a + \pi_b + \pi_c = 1 \ldots\text{equation 4}\]

we can write \[\pi_a = 1 - \pi_b - \pi_c\] and substitute in the above equations to get:
\[-3(1 - \pi_b - \pi_c) + 3\pi_b + 2\pi_c = 0 \ldots\text{equation 1}\]
\[2(1 - \pi_b - \pi_c) - 3\pi_b + 5\pi_c = 0 \ldots\text{equation 2}\]
\[1 - \pi_b - \pi_c + 3\pi_b - 7\pi_c = 0 \ldots\text{equation 3}\]

let us expand equation 1:
\[-3 + 3\pi_b + 3\pi_c + 3\pi_b + 2\pi_c = 0\]
\[-3 + 6\pi_b + 5\pi_c = 0\]
\[6\pi_b + 5\pi_c = 3 \ldots\text{equation 5}\]

let us expand equation 2:
\[2 - 2\pi_b - 2\pi_c - 3\pi_b + 5\pi_c = 0\]
\[2 - 5\pi_b + 3\pi_c = 0\]
\[-5\pi_b + 3\pi_c = -2 \ldots\text{equation 6}\]

the above equation 5 and 6 can be written as:
\[\begin{pmatrix}6&5\\-5&3\end{pmatrix}\begin{pmatrix}\pi_b\\\pi_c\end{pmatrix} = \begin{pmatrix}3\\-2\end{pmatrix}\]
its argumented matrix is:
\[\begin{pmatrix}6&5&|&3\\-5&3&|&-2\end{pmatrix}\]
by row reduction, we get:
\[R_1 => \frac{1}{6}R_1 => \begin{pmatrix}1&\frac{5}{6}&|&\frac{1}{2}\\-5&3&|&-2\end{pmatrix}\]
\[R_2 => R_2 + 5R_1 => \begin{pmatrix}1&\frac{5}{6}&|&\frac{1}{2}\\0&\frac{43}{6}&|&\frac{1}{2}\end{pmatrix}\]
\[R_2 => \frac{6}{43}R_2 => \begin{pmatrix}1&\frac{5}{6}&|&\frac{1}{2}\\0&1&|&\frac{3}{43}\end{pmatrix}\]
\[R_1 => R_1 - \frac{5}{6}R_2 => \begin{pmatrix}1&0&|&\frac{19}{43}\\0&1&|&\frac{9}{43}\end{pmatrix}\]

this gives us:
\[\pi_b = \frac{19}{43}, \quad \pi_c = \frac{9}{43}\]
\[\Pi_a = 1 - \frac{19}{43} - \frac{9}{43} = \frac{15}{43}\]

therefore the steady state distribution is given by:
\[\pi = \begin{pmatrix}\frac{15}{43}&\frac{19}{43}&\frac{9}{43}\end{pmatrix}\]


A lot of employees are in the medium performance state, while the least number of employees are in the low performance state. The high performance state has the second highest number of employees. So the company might introduce Initiatives to push the medium performance employees to high performance state and the low performance employees to medium performance state.

\begin{center}
    \large \textbf{Question 6: Conditional Probability}
\end{center}
\section{storyline}
An insurance company examines its pool of auto insurance customers and makes the following observations:
\begin{itemize}
    \item All customers insure at least one car
    \item 70\% of the customers insure more than one car
    \item 20\% of the customers insure a sports car
    \item of those customers who insure more than one car, 15\% insure a sports car
\end{itemize}

\section{find the probability that a randomly chosen customer insures one car and it is not a sports car}

for Conditional probability, the formula is:
\[P(A|B) = \frac{P(A \cap B)}{P(B)}\]

let \(A\) be the event that a customer insures one car and \(B\) be the event that the customer insure a sports car.

let us list out the probability of the events we have:

\begin{itemize}
    \item \(P(A) = 0.3\) = probability that a customer insures one car
    \item \(P(A^c) = 0.7\) = probability that a customer insures more than one car
    \item \(P(B) = 0.2\) = probability that a customer insures a sports car
    \item \(P(B^c = 0.8)\) = probability that a customer does not insure a sports car
    \item \(P(A^c \cap B) = 0.105\) = probability that a customer insures more than one car and it is a sports car
    \item \(P(A^c \cap B^c ) = 0.595\) = probability that a customer insures more than one car and it is not a sports car
\end{itemize}

this is how the tree diagram looks like:

\begin{forest}
    for tree={grow=east,
    edge={-latex, line width=1pt},
    parent anchor=east,
    child anchor=west,
    align=center,
    anchor=west,
    l sep+=10pt,
    s sep+=10pt,
    inner sep=2pt,}
    [Customer
    [One Car, edge label={node[midway,left]{0.3}}
    [Not a sports car, edge label={{node[midway,left]{0.8}}}, label=right:{= \(P(A \cap B^c) = 0.24\)}]
    [Sports car, edge label={{node[midway,left]{0.2}}}, label=right:{= \(P(A \cap B) = 0.06\)}]
]
        [More than one car, edge label={node[midway,left]{0.7}}
            [Not a sports car, edge label={node[midway,left]{0.85}}, label=right:{= \(P(A^c \cap B^c) = 0.595\)}]
            [Sports car, edge label={node[midway,left]{0.15}}, label=right:{= \(P(A^c \cap B) = 0.105\)}]
        ]
    ]
\end{forest}


so, let us do the calculations:


What is the probability that a customer insures one car and it is a sports car? this are dependent events
\begin{itemize}
   \item \(P(B^c = 0.8)\) = probability that a customer does not insure a sports car
   \item \(P(B^c|A) = \frac{P(B^c \cap A)}{P(A)}\) = probability that a customer insures one car and it is not a sports car
   \item \(P(B^c \cap A) = P(A) - P(A \cap B) = 0.3 - (0.3 \times 0.2) = 0.24\) = probability that a customer insures one car and it is not a sports car
   \item \(P(A \cap B) = P(A) \times P(B) = 0.3 \times 0.2 = 0.06\) = probability that a customer insures one car and it is a sports car
\end{itemize}

Therefore the probability that a randomly chosen customer insures one car and it is not a sports car is \(\mathbf{0.24}\)

\begin{center}
    \large \textbf{Question 7: Bayes Theorem}
\end{center}

\section{questions}

Bayes Theorem = \(P(A|B) = \frac{P(B|A)P(A)}{P(B)}\)

(a) (4 points) A medical test for a new strain of a viral respiratory disease has a 98\% sensitivity and
a 97\% specificity. If 0.5\% of the population has the disease, calculate the probability that a person
has the disease given they tested positive.

\begin{itemize}
    \item \(P(D) = 0.005\) = probability that a person has the disease
    \item \(P(D^c) = 0.995\) = probability that a person does not have the disease
    \item \(P(T|D) = 0.98\) = probability that a person tests positive given they have the disease
    \item \(P(T|D^c) = 0.03\) = probability that a person tests positive given they do not have the disease
    \item \(P(T) = P(T|D)P(D) + P(T|D^c)P(D^c) = (0.98 \times 0.005) + (0.03 \times 0.995) = 0.03475\) = probability that a person tests positive
    \item \(P(D/T) = \frac{P(T|D)P(D)}{P(T)} = \frac{0.98 \times 0.005}{0.03475} \approx 0.141\)
\end{itemize}


(b) (4 points) A financial credit scoring model is 95 percent effective in identifying individuals who
are likely to default on their loans when they actually will default. However, the model also yields a
“false positive” result for 1 percent of individuals who are creditworthy. (That is, if a creditworthy
individual is tested, then, with probability 0.01, the model will incorrectly classify them as likely
to default.) If 0.5 percent of the population actually defaults on their loans, what is the probability
that an individual will default given that the model predicts they are likely to default?

\begin{itemize}
    \item \(P(E) = 0.005\) = probability that an individual will default on their loans
    \item \(P(E|D) = 0.95\) = probability that an individual will default on their loans given they actually will default
    \item \(P(D) = P(E|D) \cdot P(D) + P(E|D^c) \cdot P(D^c)\) = probability that an individual will default on their loans
    \item \(P(E|D^c) = 0.01\) = probability that an individual will default on their loans given they are creditworthy
    \item \(P(E) = (0.95)(0.005) + (0.01)(0.995)\)
    \item \(P(D) = 0.0147\) = probability that an individual will default on their loans
\end{itemize}

Bayes Theorem = \(P(D|E) = \frac{P(E|D)P(D)}{P(E)} = \frac{0.95 \times 0.005}{0.0147} \approx 0.323\)

\begin{center}
    \large \textbf{Question 8: Independence of Events, Inclusion-Exclusion Principle \& Mutual Exclusivity}
\end{center}

\section{Positive probabilities \& Independent events}
1. show that if A \& B are events with positive probabilities, then so are:

1. \(A^c\) and B 

for indepedent events: \(P(A \cap B) = P(A) \cdot P(B)\)\newline

\(A^c = 1-P(A)\)

(b) (2 points) A and Bc

(c) (2 points) Ac and Bc\newline

2. (3 points) In the ECE class of 2026 consisting of 25 students, 15 take Robotics, 10 take Introduction
to Systems Software Engineering, and 5 take both. Calculate the number of students who
take either Robotics or Intro. to Systems Software Engineering or both.\newline\newline
total population = 25\newline
students who take Robotics = 15\newline
students who take Introduction to Systems Software Engineering = 10\newline
students who take both = 5 \(P(R \cap E)\)\newline
we are trying to find the number of students who take either Robotics or Intro. to Systems Software Engineering or both.\newline

\(P(R \cup E ) = P(R) + P(E) - P(R \cap E)\)\newline
\(P(R \cup E ) = 15 + 10 - 5 = 20\)\newline

therefore the number of students who take either Robotics or Intro. to Systems Software Engineering or both is 20\newline

3. You are at Nyandungu amusement park with your friends and you want to play a game where you
roll a fair six-sided die. There are prizes for rolling specific numbers:\newline
• If you roll a 1 or a 2, you win a small prize (Event A).\newline
• If you roll a 3, 4, or 5, you win a medium prize (Event B).\newline
• If you roll a 6, you win a large prize (Event C).\newline

(a) (2 points) Are Events A and B mutually exclusive? Why or why not?

Yes they are mutually exclusive because the events do not have any outcomes in common. If you roll a 1 or 2, you cannot roll a 3, 4, or 5. Therefore, the events are mutually exclusive.\newline
(b) (2 points) Are Events A and C mutually exclusive? Why or why not?
Yes they are mutually exclusive because the events do not have any outcomes in common. If you roll a 1 or 2, you cannot roll a 6. Therefore, the events are mutually exclusive.\newline
(c) (2 points) What is the probability of winning either a small or a medium prize?
Probability of winning a small prize = \(P(A) = \frac{2}{6} = \frac{1}{3}\)\newline
Probability of winning a medium prize = \(P(B) = \frac{3}{6} = \frac{1}{2}\)\newline
Probability of winning either a small or a medium prize = \(P(A \cup B) = P(A) + P(B) = \frac{1}{3} + \frac{1}{2} = \frac{5}{6}\)\newline

answer = \(\mathbf{\frac{5}{6}}\)\newline

\begin{center}
    \large \textbf{Question 9: Combinatorics}
\end{center}

\section{questions}

1. calculate the number of ways to arrange the letters in the word "ALGORITHM" such that the vowels are together.

permutations: \(P(n) = \frac{n!}{(n-r)!}\)

the number of letters in the word "ALGORITHM" is 8, with 3 vowels and 5 consonants.

the number of ways to arrange the vowels is \(P(3) = \frac{8!}{(8-3)!} = 336\) \newline

2. (4 points) Every fall semester, elections are held at CMU-Africa to choose members of the student
guild and other club representatives. After, the elections were held, the elected members conducted
a survey on students problems and it was revealed that tuition funding and housing ranked first.
The guild decided on decentralized task forces to address specific issues from a group of 5 females
and 7 males, how many different committees consisting of 2 females and 3 males can be formed?
What if 2 of the males are feuding and refuse to serve on the committee together?\newline

Combinations to choose 2 females in a group of 5(here order doesn't matter): \(C(n,r) = \frac{n!}{r!(n-r)!}\)\newline

Combinations\newline

\(C\binom{5}{2} = \frac{5!}{2!(5-2)!} = \frac{120}{12} = 10\)\newline

Combinations to choose 3 males in a group of 7(here order doesn't matter):\newline

\(C\binom{7}{3} = \frac{7!}{3!(7-3)!} = \frac{5040}{144} = 35\)\newline

Therefore the number of different committees without restrictions: \(10 \times 35 = 350\)\newline

If 2 of the males refuse to work together, then the number of different committees is given by:

we need three males from the 7\newline

let us say we are forced to choose the two of them, as in the C(2,2) = 1\newline

we need an extra male from the remaining 5 males so that we fill up the committee slots for males\newline

this means C(5,1) = 5, there are five different ways to form the committee since this two male can be paired up with any of the remaining gents\newline

this means the number of invalid committees (ones with the two feuding male) is  \(1 \times 5 = 5\)\newline

so the total number of valid committees is 350 - 5 = 345\newline

answer = \(\mathbf{345}\)\newline


3. A recent census summary reveals that there are about 1.75 million people in Kigali, of which approximately 55\% are females. A survey conducted by a major exchange student in cosmetics reveals an alarming balding rate among men. The survey reports that the average hair count for people in Kigali is 150, 000 hairs. Men aged 35 years and above have about 35\% of the average
hair count. This age group constitutes 60\% of the male population. Assume the hair counts for men aged 35 years and above follow a normal distribution with a mean of 52, 500 hairs and a standard deviation of 10, 000 hairs.

This is what we have so far:

\begin{itemize}
    \item total population = 1.75 million
    \item female population = 0.55 of 1.75 million
    \item male population = 0.45 of 1.75 million
    \item male 35 years and above = 0.6 of 0.45 of 1.75 million
\end{itemize}


(a) (2 points) Calculate the expected number of men aged 35 years and above in Kigali.

\(E(X) = \mu = 0.6 \times 0.45 \times 1.75 million = \mathbf{472,500}\) \newline

(2 points) Determine the range within which approximately 68\% of the hair counts for men aged 35 years and above lie, using the given normal distribution

\begin{itemize}
    \item \(\mu = 52,500\)
    \item \(\sigma = 10,000\)
    \item 68\% of the hair counts lie within 1 standard deviation from the mean
    \item the range is given by: \(\mu \pm \sigma\)
    \item the range is given by: \(52,500 \pm 10,000 = 42,500 \text{ and } 62,500\)
\end{itemize}

3. (2 points) Considering the calculated range, use the pigeonhole principle to show that at
least 2 men will have the same hair count within this range. Assume hair counts are discrete
values.\newline\newline
the range is given by: \(42,500 \text{ and } 62,500\)\newline
men in kigali each 35 and above = 472,500\newline
the range is 20,000 + 1 = the range is inclusive\newline
men in kigali that are likely to fall within this range = 0.86 of 472,500 = 321,300\newline

since the range is 20,000 and the number of men is 321,300, then by the pigeonhole principle, at least 2 men will have the same hair count within this range.\newline




\begin{center}
    \large \textbf{Question 10: LAW OF TOTAL PROBABILITY}
\end{center}

\textbf{Questions}

1. A factory produces three types of gadgets. Type A constitutes 50\%, Type B constitutes 30\%, and Type C constitutes 20\% of the total production. The defect rates for these gadgets are 1\%, 2\%, and 3\% respectively. Calculate the probability that a randomly selected gadget is defective

this is what we have:
\begin{itemize}
    \item \(P(A) = 0.5\), \(P(D) = 0.01\)
    \item \(P(B) = 0.3\), \(P(D) = 0.02\)
    \item \(p(c) = 0.2\), \(P(D) = 0.03\)
\end{itemize}

\begin{forest}
    for tree={grow=east,
    edge={-latex, line width=1pt},
    parent anchor=east,
    child anchor=west,
    align=center,
    anchor=west,
    l sep+=10pt,
    s sep+=10pt,
    inner sep=2pt,}
    [Gadgets
    [A, edge label={node[midway,left]{0.5}}
        [Not  Defective, edge label={{node[midway,left]{0.99}}}]
        [Defective, edge label={{node[midway,left]{0.03}}}]
    ]
    [B, edge label={node[midway,left]{0.2}}
        [Not  Defective, edge label={node[midway,left]{0.98}}]
        [Defective, edge label={node[midway,left]{0.02}}]
    ]
    [C, edge label={node[midway,left]{0.3}}
        [Not  Defective, edge label={node[midway,left]{0.97}}]
        [Defective, edge label={node[midway,left]{0.03}}]
    ]
    ]
\end{forest}

a probability that a randomly selected gadget is defective is given by:

\[P(D) = P(D|A)P(A) + P(D|B)P(B) + P(D|C)P(C)\] = \[0.5 \times 0.01 + 0.3 \times 0.02 + 0.2 \times 0.03 = 0.017\]

2. (4 marks) Samsung has three factories—Factory Korea, Factory Japan, and Factory USA that produce the same type of electronic component. The factory in Korea produces 50\% of the components, 30\% are produced in Japan, and the factory in the USA is responsible for 20\% of production.
The probability of a component being defective is 2\% for Factory Korea, 4\% for Factory Japan, and 5\% for Factory USA. Suppose a randomly selected component from the company’s entire production is found to be
defective. What is the probability that it was produced by Factory Japan?

This is what we have so far:

\begin{itemize}
    \item \(P(K) = 0.5\), \(P(D) = 0.02\) - Korea
    \item \(P(J) = 0.3\), \(P(D) = 0.04\) - Japan
    \item \(P(U) = 0.2\), \(P(D) = 0.05\) - USA
\end{itemize}

\begin{forest}
    for tree={grow=east,
    edge={-latex, line width=1pt},
    parent anchor=east,
    child anchor=west,
    align=center,
    anchor=west,
    l sep+=10pt,
    s sep+=10pt,
    inner sep=2pt,}
    [Factories
    [Korea, edge label={node[midway,left]{0.5}}
        [Not  Defective, edge label={{node[midway,left]{0.98}}}]
        [Defective, edge label={{node[midway,left]{0.02}}}]
    ]
    [Japan, edge label={node[midway,left]{0.3}}
        [Not  Defective, edge label={node[midway,left]{0.96}}]
        [Defective, edge label={node[midway,left]{0.04}}]
    ]
    [USA, edge label={node[midway,left]{0.2}}
        [Not  Defective, edge label={node[midway,left]{0.95}}]
        [Defective, edge label={node[midway,left]{0.05}}]
    ]
    ]
\end{forest}

the probability that a component was produced by Factory Japan given that it is defective is given by:

we first need to calculate the probability that a component is defective using law of total probability:

\[P(D) = P(D|K)P(K) + P(D|J)P(J) + P(D|U)P(U)\] = \[0.5 \times 0.02 + 0.3 \times 0.04 + 0.2 \times 0.05 = 0.032\]

\[P(J|D) = \frac{P(D|J)P(J)}{P(D)} = \frac{0.04 \times 0.3}{0.032} = \frac{0.012}{0.032} \approx 0.375\]

\end{document}