\documentclass[11pt]{article}

\usepackage{amsmath,amssymb,amsfonts}
\usepackage{graphicx}

\usepackage{geometry}
\geometry{a3paper, margin=0.5in} % Adjust the margin as needed


\setlength{\topmargin}{-.5in} \setlength{\textheight}{9.25in}
\setlength{\oddsidemargin}{0in} \setlength{\textwidth}{6.8in}


\begin{document}

\Large


\noindent{\bf Kipngeno koech - bkoech  Assignment 1\hfill 18 September 2024}


\medskip\hrule
\begin{enumerate}

\item Consider the system of linear equations below:
\[
\begin{aligned}
    3x + y &= 6 \\
    -2x + 2y + 8z &= -8 \\
    4x + 4y + 8z &= 4
\end{aligned}
\]

\begin{enumerate}
    \item Express the system in a compact matrix form, Ax = b 
    \[
        A = \begin{bmatrix}
            3 & 1 & 0 \\
            -2 & 2 & 8 \\
            4 & 4 & 8
        \end{bmatrix}
        x = \begin{bmatrix}
            x \\
            y \\
            z
        \end{bmatrix}
        b = \begin{bmatrix}
            6 \\
            -8 \\
            4
        \end{bmatrix}\newline
    \]  
    \text{To represent the system in a compact matrix form:}
    \[
        \begin{bmatrix}
            3 & 1 & 0 \\
            -2 & 2 & 8 \\
            4 & 4 & 8
        \end{bmatrix}
        \begin{bmatrix}
            x \\
            y \\
            z
        \end{bmatrix}
        = \begin{bmatrix}
            6 \\
            -8 \\
            4
        \end{bmatrix}
    \]
    \item \textbf{Use Gaussian elimination to determine if the system has no solution, one unique solution or infinitely many solutions and justify your answer}
    \[
        \begin{bmatrix}
            3 & 1 & 0 &|& 6 \\
            -2 & 2 & 8 &|& -8 \\
            4 & 4 & 8 & |&4
        \end{bmatrix}
        R_1 \leftrightarrow R_2
        \begin{bmatrix}
            -2 & 2 & 8 &|& -8 \\
            3 & 1 & 0 &|& 6 \\
            4 & 4 & 8 & |&4
        \end{bmatrix}
        R_1 = R_1 + R_2
        \begin{bmatrix}
            1 & 3 & 8 &|& -2 \\
            3 & 1 & 0 &|& 6 \\
            4 & 4 & 8 & |&4
        \end{bmatrix}
    \]
    \[
        R_2 = R_2 - 3R_1
        \begin{bmatrix}
            1 & 3 & 8 &|& -2 \\
            0 & -8 & -24 &|& 12 \\
            4 & 4 & 8 & |&4
        \end{bmatrix}
        R_3 = R_3 - 4R_1
        \begin{bmatrix}
            1 & 3 & 8 &|& -2 \\
            0 & -8 & -24 &|& 12 \\
            0 & -8 & -24 & |&12
        \end{bmatrix}
    \]

    \[
        R_3 = R_3 - R_2
        \begin{bmatrix}
            1 & 3 & 8 &|& -2 \\
            0 & -8 & -24 &|& 12 \\
            0 & 0 & 0 & |&0
        \end{bmatrix}
        R_2 = -\dfrac{1}{8}R_2
        \begin{bmatrix}
            1 & 3 & 8 &|& -2 \\
            0 & 1 & 3 &|& -1.5 \\
            0 & 0 & 0 & |&0
        \end{bmatrix}
    \]
    This system has infinitely many solutions since the last row of the matrix is all zeros. The system is consistent.
    \item \textbf{If the system has a solution, what is the solution?}
    \[
        \begin{aligned}
            x + 3y + 8z &= -2 \\
            y + 3z &= -1.5
        \end{aligned}
    \]
    \[
        \begin{aligned}
            x &= -2 - 3y - 8z \\
            y &= -1.5 - 3z
        \end{aligned}
    \]
    \[
        \begin{aligned}
            x &= -2 - 3(-1.5 - 3z) - 8z \\
            x &= -2 + 4.5 + 9z - 8z \\
            x &= 2.5 + z
        \end{aligned}
    \]
    The solution to the system is:
    \[
        \begin{aligned}
            x &= 2.5 + z \\
            y &= -1.5 - 3z \\
            z &= z
        \end{aligned}
    \]
    where Z is a free variable and can take any value.
\end{enumerate}

\item \textbf{Determine whether the following systems of equations (or matrix equations) described below have no solution, one unique solution, or infinitely many solutions, and justify your answer.}

\begin{enumerate}
    \item \textbf{}
    \[
        \begin{aligned}
            ax + by &= c \\
            dx + ey &= f
        \end{aligned}
    \]
    where \(a, b, c, d, e, f\) are scalars satisfying \(\frac{a}{d} = \frac{b}{e} = \frac{c}{f}\)
    \[
        \begin{aligned}
            \frac{a}{d} = \frac{b}{c} = \frac{c}{f} \\
            e = \frac{bd}{a}
        \end{aligned}
    \]
    \[
        \begin{bmatrix}
            a & b \\
            d & e
        \end{bmatrix}
        \begin{bmatrix}
            x \\
            y
        \end{bmatrix}
        = \begin{bmatrix}
            c \\
            f
        \end{bmatrix}
    \]

    \[
        \begin{bmatrix}
            a & b \\
            d & \frac{bd}{a}
        \end{bmatrix}
        \begin{bmatrix}
            x \\
            y
        \end{bmatrix}
        = \begin{bmatrix}
            c \\
            f
        \end{bmatrix}
    \]
    in matrix argumented form:
    \[
        \begin{bmatrix}
            a & b &|& c \\
            d & \frac{bd}{a} &|& f
        \end{bmatrix}
        R_1 = \frac{1}{a}R_1
        \begin{bmatrix}
            1 & \frac{b}{a} &|& \frac{c}{a} \\
            d & \frac{bd}{a} &|& f
        \end{bmatrix}
        R_2 = R_2 - dR_1
        \begin{bmatrix}
            1 & \frac{b}{a} &|& \frac{c}{a} \\
            0 & 0 &|& f - d\frac{c}{a}
        \end{bmatrix}
    \]

    The system has no solution if \(f - d\frac{c}{a} \neq 0\), or infinitely many solutions if \(f - d\frac{c}{a} = 0\)
  
        
    \item A homogeneous system of 3 equations in 4 unknowns
    \[
        \begin{aligned}
            ax + by + cz + dw &= 0 \\
            ex + fy + gz + hw &= 0 \\
            ix + jy + kz + lw &= 0
        \end{aligned}
    \]
    \[
        \begin{bmatrix}
            a & b & c & d \\
            e & f & g & h \\
            i & j & k & l
        \end{bmatrix}
        \begin{bmatrix}
            x \\
            y \\
            z \\
            w
        \end{bmatrix}
        = \begin{bmatrix}
            0 \\
            0 \\
            0
        \end{bmatrix}
    \]
    The system has infinitely many solutions since there will be a lot of free variables.
    \item \textbf{Ax = b, where the row-reduced echelon form of the augmented matrix [A | b] looks as follows:}
    \[
        \begin{bmatrix}
            1 & 0 & -1 &|& 0 \\
            0 & 1 & 2 &|& 0 \\
            0 & 0 & 0 &|& 1
        \end{bmatrix}
    \]
    The system has no solution since the last row of the matrix is all zeros with the last element being 1, this means that the system is inconsistent.
\end{enumerate}

\item Given the matrix A:
\[
    A = \begin{bmatrix}
        0 & 1 & 1 \\
        1 & 0 & -1 \\
        -2 & 1 & 0 \\
        -1 & 1 & 1
    \end{bmatrix}
\]
\begin{enumerate}
\item Compute \(A^\top A\) and show that it is symmetric
\[
    A^\top A = \begin{bmatrix}
        0 & 1 & -2 & -1 \\
        1 & 0 & 1 & 1 \\
        1 & -1 & 0 & 1
    \end{bmatrix}
    \begin{bmatrix}
        0 & 1 & 1 \\
        1 & 0 & -1 \\
        -2 & 1 & 0 \\
        -1 & 1 & 1
    \end{bmatrix}
    = \begin{bmatrix}
        6 & -1 & -1 \\
        -1 & 4 & 0 \\
        -1 & 0 & 4
    \end{bmatrix}
\]

To show that it is symmetric: \(A^\top A = (A^\top A)^\top\)
\[
    \begin{bmatrix}
        6 & -1 & -1 \\
        -1 & 4 & 0 \\
        -1 & 0 & 4
    \end{bmatrix}
    = \begin{bmatrix}
        6 & -1 & -1 \\
        -1 & 4 & 0 \\
        -1 & 0 & 4
    \end{bmatrix}
\]

\item Compute \((A^\top A)^{-1}\) using Gaussian Elimination
\[
    \begin{bmatrix}
        6 & -1 & -1 &|& 1 & 0 & 0 \\
        -1 & 4 & 0 &|& 0 & 1 & 0 \\
        -1 & 0 & 4 &|& 0 & 0 & 1
    \end{bmatrix}
    R_1 = \frac{1}{6}R_1
    \begin{bmatrix}
        1 & -\frac{1}{6} & -\frac{1}{6} &|& \frac{1}{6} & 0 & 0 \\
        -1 & 4 & 0 &|& 0 & 1 & 0 \\
        -1 & 0 & 4 &|& 0 & 0 & 1
    \end{bmatrix}
\]
\[
    R_2 = R_2 + R_1
    \begin{bmatrix}
        1 & -\frac{1}{6} & -\frac{1}{6} &|& \frac{1}{6} & 0 & 0 \\
        0 & \frac{23}{6} & -\frac{1}{6} &|& \frac{1}{6} & 1 & 0 \\
        -1 & 0 & 4 &|& 0 & 0 & 1
    \end{bmatrix}
    R_3 = R_3 + R_1
    \begin{bmatrix}
        1 & -\frac{1}{6} & -\frac{1}{6} &|& \frac{1}{6} & 0 & 0 \\
        0 & \frac{23}{6} & -\frac{1}{6} &|& \frac{1}{6} & 1 & 0 \\
        0 & -\frac{1}{6} & \frac{17}{6} &|& \frac{1}{6} & 0 & 1
    \end{bmatrix}
\]

\[
    R_2 = \frac{6}{23}R_2
    \begin{bmatrix}
        1 & -\frac{1}{6} & -\frac{1}{6} &|& \frac{1}{6} & 0 & 0 \\
        0 & 1 & -\frac{1}{23} &|& \frac{1}{23} & \frac{6}{23} & 0 \\
        0 & -\frac{1}{6} & \frac{17}{6} &|& \frac{1}{6} & 0 & 1
    \end{bmatrix}
    R_1 = R_1 + \frac{1}{6}R_2
    \begin{bmatrix}
        1 & 0 & -\frac{1}{23} &|& \frac{1}{23} & \frac{1}{6} & 0 \\
        0 & 1 & -\frac{1}{23} &|& \frac{1}{23} & \frac{6}{23} & 0 \\
        0 & -\frac{1}{6} & \frac{17}{6} &|& \frac{1}{6} & 0 & 1
    \end{bmatrix}
\]

\[
    R_3 = R_3 + \frac{1}{6}R_2
    \begin{bmatrix}
        1 & 0 & -\frac{1}{23} &|& \frac{1}{23} & \frac{1}{6} & 0 \\
        0 & 1 & -\frac{1}{23} &|& \frac{1}{23} & \frac{6}{23} & 0 \\
        0 & 0 & \frac{17}{23} &|& \frac{1}{23} & \frac{1}{6} & 1
    \end{bmatrix}
    R_3 = \frac{23}{17}R_3
    \begin{bmatrix}
        1 & 0 & -\frac{1}{23} &|& \frac{1}{23} & \frac{1}{6} & 0 \\
        0 & 1 & -\frac{1}{23} &|& \frac{1}{23} & \frac{6}{23} & 0 \\
        0 & 0 & 1 &|& \frac{1}{17} & \frac{3}{17} & \frac{23}{17}
    \end{bmatrix}
\]

\[
    R_1 = R_1 + \frac{1}{23}R_3
    \begin{bmatrix}
        1 & 0 & 0 &|& \frac{1}{17} & \frac{5}{17} & \frac{23}{17} \\
        0 & 1 & -\frac{1}{23} &|& \frac{1}{23} & \frac{6}{23} & 0 \\
        0 & 0 & 1 &|& \frac{1}{17} & \frac{3}{17} & \frac{23}{17}
    \end{bmatrix}
    R_2 = R_2 + \frac{1}{23}R_3
    \begin{bmatrix}
        1 & 0 & 0 &|& \frac{1}{17} & \frac{5}{17} & \frac{23}{17} \\
        0 & 1 & 0 &|& \frac{2}{17} & \frac{9}{17} & \frac{23}{17} \\
        0 & 0 & 1 &|& \frac{1}{17} & \frac{3}{17} & \frac{23}{17}
    \end{bmatrix}
\]

\[
    (A^\top A)^{-1} = \begin{bmatrix}
        \frac{1}{17} & \frac{5}{17} & \frac{23}{17} \\
        \frac{2}{17} & \frac{9}{17} & \frac{23}{17} \\
        \frac{1}{17} & \frac{3}{17} & \frac{23}{17}
    \end{bmatrix}
\]

\item  Let \( b = \begin{bmatrix}
    1 \\
    0 \\
    1 \\
    0
\end{bmatrix} \). If \( Ax = b \), show that \( x = (A^\top A)^{-1}A^\top b \) and obtain the value of \( x \). [Note: \( x = (A^\top A)^{-1}A^\top b \) is called the Normal equation]

\[
    Ax = b, 
    where, A = A^\top A = \begin{bmatrix}
        6 & -1 & -1 \\
        -1 & 4 & 0 \\
        -1 & 0 & 4
    \end{bmatrix}
    b = \begin{bmatrix}
        1 \\
        0 \\
        1 \\
        0
    \end{bmatrix}
\]

\[
    Ax=b, (A^\top A)x =  b
\]
\[
    (A^\top A)^{-1}(A^\top A)x = (A^\top A)^{-1}b
\]
\[
    x = (A^\top A)^{-1}b
\]
\[
    x = \begin{bmatrix}
        \frac{1}{17} & \frac{5}{17} & \frac{23}{17} \\
        \frac{2}{17} & \frac{9}{17} & \frac{23}{17} \\
        \frac{1}{17} & \frac{3}{17} & \frac{23}{17}
    \end{bmatrix}
    \begin{bmatrix}
        1 \\
        0 \\
        1
    \end{bmatrix}
\]
\[
    x =
    \begin{bmatrix}
        24/17 \\
        25/17 \\
        24/17
    \end{bmatrix}
\]

\end{enumerate}

\item CMU-Africa is trying to understand the pricing strategy used by Delight Canteen. The Canteen sells various types of food, each with a different price, but only charges a total price for all food types on a student’s plate. As a machine learning engineer, you want to help students determine the price of their food based on the quantity (measured in Grams) of each foot type they add to their plate. In other to achieve this, you collect data from 6 of your friends on the quantity of each food type they served and the total price. The observations from your friends are recorded below.

\begin{tabular}{|c|c|c|c|c|c|}
    \hline
    Transaction & Food a (g) & Food b (g) & Food c (g) & Food d (g) & Total Cost (RWF) \\
    \hline
    1 & 100 & 50 & 150 & 200 & 2500 \\
    2 & 50 & 50 & 100 & 300 & 2300 \\
    3 & 100 & 150 & 200 & 100 & 3000 \\
    4 & 50 & 200 & 300 & 50 & 2900 \\
    5 & 200 & 50 & 250 & 50 & 3100 \\
    6 & 300 & 50 & 50 & 200 & 4300 \\
    \hline
\end{tabular}

From the observations in the table above, you are expected to obtain the price per quantity of each food
type (mearsured in RWF/g).

\begin{enumerate}
\item Describe the above using a system of linear equations
\[
    \begin{aligned}
        100a + 50b + 150c + 200d &= 2500 \\
        50a + 50b + 100c + 300d &= 2300 \\
        100a + 150b + 200c + 100d &= 3000 \\
        50a + 200b + 300c + 50d &= 2900 \\
        200a + 50b + 250c + 50d &= 3100 \\
        300a + 50b + 50c + 200d &= 4300
    \end{aligned}
\]
\item Write the system of linear equations in a compact matrix form, Ax = b
\[
    A = \begin{bmatrix}
        100 & 50 & 150 & 200 \\
        50 & 50 & 100 & 300 \\
        100 & 150 & 200 & 100 \\
        50 & 200 & 300 & 50 \\
        200 & 50 & 250 & 50 \\
        300 & 50 & 50 & 200
    \end{bmatrix}
    x = \begin{bmatrix}
        a \\
        b \\
        c \\
        d
    \end{bmatrix}
    b = \begin{bmatrix}
        2500 \\
        2300 \\
        3000 \\
        2900 \\
        3100 \\
        4300
    \end{bmatrix}
\]
\[
    Ax = b = \begin{bmatrix}
        100 & 50 & 150 & 200 \\
        50 & 50 & 100 & 300 \\
        100 & 150 & 200 & 100 \\
        50 & 200 & 300 & 50 \\
        200 & 50 & 250 & 50 \\
        300 & 50 & 50 & 200
    \end{bmatrix}
    \begin{bmatrix}
        a \\
        b \\
        c \\
        d
    \end{bmatrix}
    = \begin{bmatrix}
        2500 \\
        2300 \\
        3000 \\
        2900 \\
        3100 \\
        4300
    \end{bmatrix}
\]
\item Use Gaussian elimination to obtain a solution for the system.
\[
    \begin{bmatrix}
        100 & 50 & 150 & 200 &|& 2500 \\
        50 & 50 & 100 & 300 &|& 2300 \\
        100 & 150 & 200 & 100 &|& 3000 \\
        50 & 200 & 300 & 50 &|& 2900 \\
        200 & 50 & 250 & 50 &|& 3100 \\
        300 & 50 & 50 & 200 &|& 4300
    \end{bmatrix} = 
    \begin{bmatrix}
        2 & 1 & 3 & 4 &|& 50 \\
        1 & 1 & 2 & 6 &|& 46 \\
        2 & 3 & 4 & 2 &|& 60 \\
        1 & 4 & 6 & 1 &|& 58 \\
        4 & 1 & 5 & 1 &|& 62 \\
        6 & 1 & 1 & 4 &|& 86
    \end{bmatrix}
\]
\[
R_1 \leftrightarrow R_2
\begin{bmatrix}
    1 & 1 & 2 & 6 &|& 46 \\
    2 & 1 & 3 & 4 &|& 50 \\
    2 & 3 & 4 & 2 &|& 60 \\
    1 & 4 & 6 & 1 &|& 58 \\
    4 & 1 & 5 & 1 &|& 62 \\
    6 & 1 & 1 & 4 &|& 86
\end{bmatrix}
R_2 = R_2 - 2R_1
\begin{bmatrix}
    1 & 1 & 2 & 6 &|& 46 \\
    0 & -1 & -1 & -8 &|& -42 \\
    2 & 3 & 4 & 2 &|& 60 \\
    1 & 4 & 6 & 1 &|& 58 \\
    4 & 1 & 5 & 1 &|& 62 \\
    6 & 1 & 1 & 4 &|& 86
\end{bmatrix}
R_3 = R_3 - 2R_1
\begin{bmatrix}
    1 & 1 & 2 & 6 &|& 46 \\
    0 & -1 & -1 & -8 &|& -42 \\
    0 & 1 & 0 & -10 &|& -32 \\
    1 & 4 & 6 & 1 &|& 58 \\
    4 & 1 & 5 & 1 &|& 62 \\
    6 & 1 & 1 & 4 &|& 86
\end{bmatrix}
\]
\[
R_4 = R_4 - R_1
\begin{bmatrix}
    1 & 1 & 2 & 6 &|& 46 \\
    0 & -1 & -1 & -8 &|& -42 \\
    0 & 1 & 0 & -10 &|& -32 \\
    0 & 3 & 4 & -5 &|& 12 \\
    4 & 1 & 5 & 1 &|& 62 \\
    6 & 1 & 1 & 4 &|& 86
\end{bmatrix}
R_5 = R_5 - 4R_1
\begin{bmatrix}
    1 & 1 & 2 & 6 &|& 46 \\
    0 & -1 & -1 & -8 &|& -42 \\
    0 & 1 & 0 & -10 &|& -32 \\
    0 & 3 & 4 & -5 &|& 12 \\
    0 & -3 & -3 & -23 &|& -122 \\
    6 & 1 & 1 & 4 &|& 86
\end{bmatrix}
\]
\[
    R_6 = R_6 - 6R_1
    \begin{bmatrix}
        1 & 1 & 2 & 6 &|& 46 \\
        0 & -1 & -1 & -8 &|& -42 \\
        0 & 1 & 0 & -10 &|& -32 \\
        0 & 3 & 4 & -5 &|& 12 \\
        0 & -3 & -3 & -23 &|& -122 \\
        0 & -5 & -11 & -32 &|& -190
    \end{bmatrix}
    R_2 = -R_2
    \begin{bmatrix}
        1 & 1 & 2 & 6 &|& 46 \\
        0 & 1 & 1 & 8 &|& 42 \\
        0 & 1 & 0 & -10 &|& -32 \\
        0 & 3 & 4 & -5 &|& 12 \\
        0 & -3 & -3 & -23 &|& -122 \\
        0 & -5 & -11 & -32 &|& -190
    \end{bmatrix}
\]
\[
    R_3 = R_3 - R_2
    \begin{bmatrix}
        1 & 1 & 2 & 6 &|& 46 \\
        0 & 1 & 1 & 8 &|& 42 \\
        0 & 0 & -1 & -18 &|& -74 \\
        0 & 3 & 4 & -5 &|& 12 \\
        0 & -3 & -3 & -23 &|& -122 \\
        0 & -5 & -11 & -32 &|& -190
    \end{bmatrix}
    R_4 = R_4 - 3R_2
    \begin{bmatrix}
        1 & 1 & 2 & 6 &|& 46 \\
        0 & 1 & 1 & 8 &|& 42 \\
        0 & 0 & -1 & -18 &|& -74 \\
        0 & 0 & 1 & -29 &|& -114 \\
        0 & -3 & -3 & -23 &|& -122 \\
        0 & -5 & -11 & -32 &|& -190
    \end{bmatrix}
\]
\[
    R_5 = R_5 + 3R_2
    \begin{bmatrix}
        1 & 1 & 2 & 6 &|& 46 \\
        0 & 1 & 1 & 8 &|& 42 \\
        0 & 0 & -1 & -18 &|& -74 \\
        0 & 0 & 1 & -29 &|& -114 \\
        0 & 0 & 0 & 1 &|& 4 \\
        0 & -5 & -11 & -32 &|& -190
    \end{bmatrix}
    R_6 = R_6 + 5R_2
    \begin{bmatrix}
        1 & 1 & 2 & 6 &|& 46 \\
        0 & 1 & 1 & 8 &|& 42 \\
        0 & 0 & -1 & -18 &|& -74 \\
        0 & 0 & 1 & -29 &|& -114 \\
        0 & 0 & 0 & 1 &|& 4 \\
        0 & 0 & -6 & 8 &|& 20
    \end{bmatrix}
\]
\[
    R_3 = -R_3
    \begin{bmatrix}
        1 & 1 & 2 & 6 &|& 46 \\
        0 & 1 & 1 & 8 &|& 42 \\
        0 & 0 & 1 & 18 &|& 74 \\
        0 & 0 & 1 & -29 &|& -114 \\
        0 & 0 & 0 & 1 &|& 4 \\
        0 & 0 & -6 & 8 &|& 20
    \end{bmatrix}
  R_2 = R_2 - R_3
    \begin{bmatrix}
        1 & 1 & 2 & 6 &|& 46 \\
        0 & 1 & 0 & -10 &|& -32 \\
        0 & 0 & 1 & 18 &|& 74 \\
        0 & 0 & 1 & -29 &|& -114 \\
        0 & 0 & 0 & 1 &|& 4 \\
        0 & 0 & -6 & 8 &|& 20
    \end{bmatrix}
\]
\[
    R_2 = R_2 + 10R_4
    \begin{bmatrix}
        1 & 1 & 2 & 6 &|& 46 \\
        0 & 1 & 0 & 0 &|& 8 \\
        0 & 0 & 1 & 18 &|& 74 \\
        0 & 0 & 1 & -29 &|& -114 \\
        0 & 0 & 0 & 1 &|& 4 \\
        0 & 0 & -6 & 8 &|& 20
    \end{bmatrix}
\]
\[
    R_2 = 8,  
    R_4 = 4
\]
\[
    -6R_3 + -8R_4 = 20\\
\]
\[
    -6R_3 + -8(4) = 20 \\
\]
\[
    -6R_3 + 32 = 20 \\
\]
\[
    -6R_3 = -12 \\
\]
\[
    R_3 = \textbf{2} \\
\]

\[
    R_1 + R_2 + 2R_3 + 6R_4 = 46 \\
\]
\[
    R_1 + 8 + 2(2) + 6(4) = 46 \\
\]
\[
    R_1 + 8 + 4 + 24 = 46 \\
\]
\[
    R_1 = 10 \\
\]
\[
\text{ food a } = \textbf{10}, \text{ food b } = \textbf{8}, \text{ food c } = \textbf{2}, \text{ food d } = \textbf{4}
\]
\item What is the rank of the matrix A of the compact matrix form
\[
    A = \begin{bmatrix}
        100 & 50 & 150 & 200 \\
        50 & 50 & 100 & 300 \\
        100 & 150 & 200 & 100 \\
        50 & 200 & 300 & 50 \\
        200 & 50 & 250 & 50 \\
        300 & 50 & 50 & 200
    \end{bmatrix}
    = 
    \begin{bmatrix}
        2 & 1 & 3 & 4 \\
        1 & 1 & 2 & 6 \\
        2 & 3 & 4 & 2 \\
        1 & 4 & 6 & 1 \\
        4 & 1 & 5 & 1 \\
        6 & 1 & 1 & 4
    \end{bmatrix} 
\]
Rank of A is \textbf{4}


\item What is rank of the augmented matrix [A|b]
\[
    \begin{bmatrix}
        100 & 50 & 150 & 200 &|& 2500 \\
        50 & 50 & 100 & 300 &|& 2300 \\
        100 & 150 & 200 & 100 &|& 3000 \\
        50 & 200 & 300 & 50 &|& 2900 \\
        200 & 50 & 250 & 50 &|& 3100 \\
        300 & 50 & 50 & 200 &|& 4300
    \end{bmatrix}
\]
Rank of the argumented matrix is \textbf{4}

\item 

\end{enumerate}

\item Which of the following sets are subspaces of \( \mathbb{R}^3 \)?\newline
conditions to check if a set is a subspace of \( \mathbb{R}^3 \):
\[
    \begin{aligned}
        &\text{1. The zero vector is in the set} \\
        &\text{2. The set is closed under addition} \\
        &\text{3. The set is closed under scalar multiplication}
    \end{aligned}
\]
\begin{enumerate}
    \item \( U_1 = \left\{ \begin{bmatrix}
        x_1 \\
        x_2 \\
        x_3
        \end{bmatrix} \in \mathbb{R}^3 \mid x_1 \leq 0 \right\} \)
    \begin{enumerate}
        \item 
        \[    
        \text{zero vector = let x be 0} =
        \begin{bmatrix}
        0 \\
        0 \\
        0
        \end{bmatrix}  is \in U_1 \text{ since } 0 \leq 0 \text{ and } X_1 \leq 0
        \]
        \item closed under addition condition:
        \[
        \begin{aligned}
            &\text{x and y are two vectors in  vector space } U_1 \\
            &\text{let vector } x = \begin{bmatrix}
                x_1 \\
                x_2 \\
                x_3
            \end{bmatrix} \text{ and vector } y = \begin{bmatrix}
                y_1 \\
                y_2 \\
                y_3
            \end{bmatrix} \text{ be in vector space } U_1 \\
            &\text{let } x + y = \begin{bmatrix}
                x_1 + y_1 \\
                x_2 + y_2 \\
                x_3 + y_3
            \end{bmatrix} \text{ be in } U_1 \\
            &\text{If } x_1 \leq 0 \text{ and } y_1 \leq 0, \text{ then } x_1 + y_1 \leq 0 \text{ because } 0 + 0 = 0
        \end{aligned}
        \]
        \item closed under scalar multiplication condition:
        \[
        \begin{aligned}
            &\text{let vector } x = \begin{bmatrix}
                x_1 \\
                x_2 \\
                x_3
            \end{bmatrix} \text{ be in vector space } U_1 \\
            &\text{let } c \text{ be a scalar} \\
            &\text{let } cx = \begin{bmatrix}
                cx_1 \\
                cx_2 \\
                cx_3
            \end{bmatrix} \text{ be in } U_1 \\
           & \text{If } c < 0 \text{ and } x_1 < 0, \text{ then } c x_1 > 0, \text{ which violates the closed under scalar condition for } U_1.
        \end{aligned}
        \]
        \textbf{ \( U_1 \) is not a subspace of \( \mathbb{R}^3 \)}
    \end{enumerate}
    \item \( U_2 = \left\{ \begin{bmatrix}
        x_1 \\
        x_2 \\
        x_3
        \end{bmatrix} \in \mathbb{R}^3 \mid x_1 + 2x_2 + x_3 = 0 \right\} \)
    \begin{enumerate}
        \item the zero vector is in the set:
        \[
        \begin{aligned}
            &\text{let } x = \begin{bmatrix}
                0 \\
                0 \\
                0
            \end{bmatrix} \text{ be in vector space } U_2 \\
            &\text{since } 0 + 2(0) + 0 = 0 \text{ then } x \text{ is in } U_2
        \end{aligned}
        \]
        \item closed under addition condition:
        \[
        \begin{aligned}
            &\text{x and y are two vectors in  vector space } U_2 \\
            &\text{let vector } x = \begin{bmatrix}
                x_1 \\
                x_2 \\
                x_3
            \end{bmatrix} \text{ and vector } y = \begin{bmatrix}
                y_1 \\
                y_2 \\
                y_3
            \end{bmatrix} \text{ be in vector space } U_2 \\
            &\begin{bmatrix}
                x_1 + y_1 \\
                x_2 + y_2 \\
                x_3 + y_3
            \end{bmatrix} \text{ is in } U_2  \text{ if } (x_1 + y_1) + 2(x_2 + y_2) + (x_3 + y_3) = 0.\\ 
            &\text{Since } x_1 + 2x_2 + x_3 = 0 \text{ and } y_1 + 2y_2 + y_3 = 0, \text{ then } (x_1 + y_1) + 2(x_2 + y_2) + (x_3 + y_3) = 0 \text{ because } 0 + 0  = 0
        \end{aligned}
        \]
        \item closed under scalar multiplication condition:
        \[
        \begin{aligned}
            &\text{let vector } x = \begin{bmatrix}
                x_1 \\
                x_2 \\
                x_3
            \end{bmatrix} \text{ be in vector space } U_2 \\
            &\text{let } c \text{ be a scalar} \\
            &\text{let } cx = \begin{bmatrix}
                cx_1 \\
                cx_2 \\
                cx_3
            \end{bmatrix} \text{ be in } U_2 \\
            &\text{If } x_1 + 2x_2 + x_3 = 0 \text{ then } c(x_1 + 2x_2 + x_3) = 0 \text{ because } c(0) = 0
        \end{aligned}
        \]
    \end{enumerate}
    \textbf{therefore \( U_2 \) is a subspace of \( \mathbb{R}^3 \)}
    \item \item \( U_3 = \left\{ \begin{bmatrix}
        x_1 \\
        x_2 \\
        x_3
        \end{bmatrix} \in \mathbb{R}^3 \mid x_3 = 1, x_2 = 2x_1 \right\} \)
    \begin{enumerate}
        \item the zero vector is in the set if:
        \[
        \begin{aligned}
            &\text{let } x = \begin{bmatrix}
                0 \\
                0 \\
                0
            \end{bmatrix} \text{ be in vector space } U_3 \\
            &\text{since } 0 = 1 \text{ and } 0 = 2(0) \text{ then } x \text{ is not in } U_3
        \end{aligned}
        \]
    \end{enumerate}
    \textbf{therefore \( U_3 \) is not a subspace of \( \mathbb{R}^3 \)}
    \item \( U_4 = \left\{ \begin{bmatrix}
    x_1 \\
    x_2 \\
    x_3
    \end{bmatrix} \in \mathbb{R}^3 \mid x_3 = 0 \right\} \)
    \begin{enumerate}
        \item the zero vector is in the set if:
        \[
        \begin{aligned}
            &\text{let } x = \begin{bmatrix}
                0 \\
                0 \\
                0
            \end{bmatrix} \text{ be in vector space } U_4 \\
            &\text{since } 0 = 0 \text{ and } X_3 = 0 \text{ then } x \text{ is in } U_4
        \end{aligned}
        \]
        \item closed under addition condition:
        \[
        \begin{aligned}
            &\text{x and y are two vectors in  vector space } U_4 \\
            &\text{let vector } x = \begin{bmatrix}
                x_1 \\
                x_2 \\
                x_3
            \end{bmatrix} \text{ and vector } y = \begin{bmatrix}
                y_1 \\
                y_2 \\
                y_3
            \end{bmatrix} \text{ be in vector space } U_4 \\
            &\begin{bmatrix}
                x_1 + y_1 \\
                x_2 + y_2 \\
                x_3 + y_3
            \end{bmatrix} \text{ is in } U_4  \text{ if } x_3 + y_3 = 0.\\ 
            &\text{Since } x_3 = 0 \text{ and } y_3 = 0, \text{ then } x_3 + y_3 = 0
        \end{aligned}
        \]
        \item closed under scalar multiplication condition:
        \[
        \begin{aligned}
            &\text{let vector } x = \begin{bmatrix}
                x_1 \\
                x_2 \\
                x_3
            \end{bmatrix} \text{ be in vector space } U_4 \\
            &\text{let } c \text{ be a scalar} \\
            &\text{let } cx = \begin{bmatrix}
                cx_1 \\
                cx_2 \\
                cx_3
            \end{bmatrix} \text{ be in } U_4 \\
            &\text{If } x_3 = 0 \text{ then } c(x_3) = 0
        \end{aligned}
        \]
    \end{enumerate}
    \textbf{therefore \( U_4 \) is a subspace of \( \mathbb{R}^3 \)}
\end{enumerate}
\item consider the following vectors \[
u = \begin{bmatrix}
3 \\
2 \\
3
\end{bmatrix}, \quad
v = \begin{bmatrix}
5 \\
3 \\
4
\end{bmatrix}, \quad
w = \begin{bmatrix}
3 \\
3 \\
6
\end{bmatrix} \in \mathbb{R}^3
\]
and
\(\alpha, \beta \in \mathbb{R}\)

\begin{enumerate}
    \item Express \( w \) as a linear combination of \( u \) and \( v \), of the form,
    \[ w = \alpha u + \beta v \]
systems of linear equations
\[
\begin{aligned}
    3\alpha + 5\beta &= 3 \\
    2\alpha + 3\beta &= 3 \\
    3\alpha + 4\beta &= 6
\end{aligned}
\]

\[
\begin{aligned}
    \begin{bmatrix}
        3 & 5 \\
        2 & 3 \\
        3 & 4
    \end{bmatrix}
    \begin{bmatrix}
        \alpha \\
        \beta
    \end{bmatrix}
    &= \begin{bmatrix}
        3 \\
        3 \\
        6
    \end{bmatrix}
\end{aligned}
\]
the augumented matrix:
\[
\begin{bmatrix}
    3 & 5 &|& 3 \\
    2 & 3 &|& 3 \\
    3 & 4 &|& 6
\end{bmatrix}
\]
\[
R_1 = R_1 - R_2
\begin{bmatrix}
    1 & 2 &|& 0 \\
    2 & 3 &|& 3 \\
    3 & 4 &|& 6
\end{bmatrix}
R_2 = R_2 - 2R_1
\begin{bmatrix}
    1 & 2 &|& 0 \\
    0 & -1 &|& 3 \\
    3 & 4 &|& 6
\end{bmatrix}
R_3 = R_3 - 3R_1
\begin{bmatrix}
    1 & 2 &|& 0 \\
    0 & -1 &|& 3 \\
    0 & -2 &|& 6
\end{bmatrix}
\]
\[
R_3 = R_3 - 2R_2
\begin{bmatrix}
    1 & 2 &|& 0 \\
    0 & -1 &|& 3 \\
    0 & 0 &|& 0
\end{bmatrix}
R_2 = -R_2
\begin{bmatrix}
    1 & 2 &|& 0 \\
    0 & 1 &|& -3 \\
    0 & 0 &|& 0
\end{bmatrix}
R_1 = R_1 - 2R_2
\begin{bmatrix}
    1 & 0 &|& 6 \\
    0 & 1 &|& -3 \\
    0 & 0 &|& 0
\end{bmatrix}
\]
so, the solution is \( \alpha = 6 \) and \( \beta = -3 \)
\item Are u, v, and w linearly independent? What is the rank of the matrix whose columns are u, v, and w ?
\begin{enumerate}
    \item u,v and  w are not linearly independent because w can be expressed as a linear combination of u and v.
    \item the rank of the matrix whose columns are u, v, and w is 2.
\end{enumerate}
\item Let \( x = \begin{bmatrix} 3 \\ 1 \\ 2 \end{bmatrix} \in \mathbb{R}^3 \). Show that the set of vectors \(\{u, v, x\}\) are linearly independent. What is the rank of the matrix whose columns are \(u\), \(v\), and \(x\)?\\
for linear independent vectors, the only solution to the equation \( \alpha u + \beta v + \gamma x = 0 \) 
\[
    u = \begin{bmatrix}
        3 \\
        2 \\
        3
    \end{bmatrix}, \quad
    v = \begin{bmatrix}
        5 \\
        3 \\
        4
    \end{bmatrix}, \quad
    x = \begin{bmatrix}
        3 \\
        1 \\
        2
    \end{bmatrix}
\]
\[
\begin{aligned}
    3\alpha + 5\beta + 3\gamma &= 0 \\
    2\alpha + 3\beta + \gamma &= 0 \\
    3\alpha + 4\beta + 2\gamma &= 0
\end{aligned}
\]
\[
\begin{aligned}
    \begin{bmatrix}
        3 & 5 & 3 \\
        2 & 3 & 1 \\
        3 & 4 & 2
    \end{bmatrix}
    \begin{bmatrix}
        \alpha \\
        \beta \\
        \gamma
    \end{bmatrix}
    &= \begin{bmatrix}
        0 \\
        0 \\
        0
    \end{bmatrix}
\end{aligned}
\]
the augumented matrix:
\[
\begin{bmatrix}
    3 & 5 & 3 &|& 0 \\
    2 & 3 & 1 &|& 0 \\
    3 & 4 & 2 &|& 0
\end{bmatrix}
\]
\[
R_1 = R_1 - R_2
\begin{bmatrix}
    1 & 2 & 2 &|& 0 \\
    2 & 3 & 1 &|& 0 \\
    3 & 4 & 2 &|& 0
\end{bmatrix}
R_2 = R_2 - 2R_1
\begin{bmatrix}
    1 & 2 & 2 &|& 0 \\
    0 & -1 & -3 &|& 0 \\
    3 & 4 & 2 &|& 0
\end{bmatrix}
R_3 = R_3 - 3R_1
\begin{bmatrix}
    1 & 2 & 2 &|& 0 \\
    0 & -1 & -3 &|& 0 \\
    0 & -2 & -4 &|& 0
\end{bmatrix}
\]
\[
R_3 = R_3 - 2R_2
\begin{bmatrix}
    1 & 2 & 2 &|& 0 \\
    0 & -1 & -3 &|& 0 \\
    0 & 0 & -1 &|& 0
\end{bmatrix}
R_3 = -R_3
\begin{bmatrix}
    1 & 2 & 2 &|& 0 \\
    0 & -1 & -3 &|& 0 \\
    0 & 0 & 1 &|& 0
\end{bmatrix}
R_2 = -R_2
\begin{bmatrix}
    1 & 2 & 2 &|& 0 \\
    0 & 1 & 3 &|& 0 \\
    0 & 0 & 1 &|& 0
\end{bmatrix}
\]
\[
R_1 = R_1 - 2R_3
\begin{bmatrix}
    1 & 2 & 0 &|& 0 \\
    0 & 1 & 3 &|& 0 \\
    0 & 0 & 1 &|& 0
\end{bmatrix}
R_2 = R_2 - 3R_3
\begin{bmatrix}
    1 & 2 & 0 &|& 0 \\
    0 & 1 & 0 &|& 0 \\
    0 & 0 & 1 &|& 0
\end{bmatrix}
R_1 = R_1 - 2R_2
\begin{bmatrix}
    1 & 0 & 0 &|& 0 \\
    0 & 1 & 0 &|& 0 \\
    0 & 0 & 1 &|& 0
\end{bmatrix}
\]
As per the reduced row echelon form,  the vectors \( u, v, x \) are linearly independent. The rank of the matrix whose columns are \( u, v, x \) is 3.
\item  Find \(a\), \(b\), and \(c \in \mathbb{R}\) such that:
\[
    a\mathbf{u} + b\mathbf{v} + c\mathbf{x} = \begin{bmatrix} 5 \\ 2 \\ 1 \end{bmatrix}
\]
\[
    u = \begin{bmatrix}
        3 \\
        2 \\
        3
    \end{bmatrix}, \quad
    v = \begin{bmatrix}
        5 \\
        3 \\
        4
    \end{bmatrix}, \quad
    x = \begin{bmatrix}
        3 \\
        1 \\
        2
    \end{bmatrix}
\]
\[
\begin{aligned}
    3a + 5b + 3c &= 5 \\
    2a + 3b + c &= 2 \\
    3a + 4b + 2c &= 1
\end{aligned}
\]
\[
\begin{aligned}
    \begin{bmatrix}
        3 & 5 & 3 \\
        2 & 3 & 1 \\
        3 & 4 & 2
    \end{bmatrix}
    \begin{bmatrix}
        a \\
        b \\
        c
    \end{bmatrix}
    &= \begin{bmatrix}
        5 \\
        2 \\
        1
    \end{bmatrix}
\end{aligned}
\]
the augumented matrix:
\[
\begin{bmatrix}
    3 & 5 & 3 &|& 5 \\
    2 & 3 & 1 &|& 2 \\
    3 & 4 & 2 &|& 1
\end{bmatrix}
\]
\[
R_1 = R_1 - R_2
\begin{bmatrix}
    1 & 2 & 2 &|& 3 \\
    2 & 3 & 1 &|& 2 \\
    3 & 4 & 2 &|& 1
\end{bmatrix}
R_2 = R_2 - 2R_1
\begin{bmatrix}
    1 & 2 & 2 &|& 3 \\
    0 & -1 & -3 &|& -4 \\
    3 & 4 & 2 &|& 1
\end{bmatrix}
\]
\[
R_3 = R_3 - 3R_1
\begin{bmatrix}
    1 & 2 & 2 &|& 3 \\
    0 & -1 & -3 &|& -4 \\
    0 & -2 & -4 &|& -8
\end{bmatrix}
R_3 = R_3 - 2R_2
\begin{bmatrix}
    1 & 2 & 2 &|& 3 \\
    0 & -1 & -3 &|& -4 \\
    0 & 0 & 2 &|& 0
\end{bmatrix}
\]
\[
R_3 = \frac{1}{2}R_3
\begin{bmatrix}
    1 & 2 & 2 &|& 3 \\
    0 & -1 & -3 &|& -4 \\
    0 & 0 & 1 &|& 0
\end{bmatrix}
R_2 = -R_2
\begin{bmatrix}
    1 & 2 & 2 &|& 3 \\
    0 & 1 & 3 &|& 4 \\
    0 & 0 & 1 &|& 0
\end{bmatrix}
\]
\[
R_1 = R_1 - 2R_3
\begin{bmatrix}
    1 & 2 & 0 &|& 3 \\
    0 & 1 & 3 &|& 4 \\
    0 & 0 & 1 &|& 0
\end{bmatrix}
R_2 = R_2 - 3R_3
\begin{bmatrix}
    1 & 2 & 0 &|& 3 \\
    0 & 1 & 0 &|& 4 \\
    0 & 0 & 1 &|& 0
\end{bmatrix}
R_1 = R_1 - 2R_2
\begin{bmatrix}
    1 & 0 & 0 &|& -5 \\
    0 & 1 & 0 &|& 4 \\
    0 & 0 & 1 &|& 0
\end{bmatrix}
\]
\( a =\textbf{ -5} \), \( b = \textbf{4} \) and \( c = \textbf{0} \)
\item  Let \( \mathbf{y} = \begin{bmatrix} 4 \\ 2 \\ k \end{bmatrix} \in \mathbb{R}^3 \). Find \( k \) such that \( \mathbf{u} \), \( \mathbf{v} \), and \( \mathbf{y} \) are linearly dependent.
\[
\begin{aligned}
    3a + 5b  &= 4 \\
    2a + 3b  &= 2 \\
    3a + 4b  &= k
\end{aligned}
\]
The augumented matrix:
\[
\begin{bmatrix}
    3 & 5 & | & 4 \\
    2 & 3 &|&2 \\
    3 & 4 &|& k
\end{bmatrix}
\]
\[
R_1 = R_1 - R_2 =
\begin{bmatrix}
    1 & 2 &|& 2 \\
    2 & 3 &|& 2 \\
    3 & 4 &|& k
\end{bmatrix}
R_2 = R_2 - 2R_1
\begin{bmatrix}
    1 & 2 &|& 2 \\
    0 & -1 &|& -2 \\
    3 & 4 &|& k
\end{bmatrix}
\]
\[
R_3 = R_3 - 3R_1
\begin{bmatrix}
    1 & 2 &|& 2 \\
    0 & -1 &|& -2 \\
    0 & -2 &|& k - 6
\end{bmatrix}
R_3 = R_3 - 2R_2
\begin{bmatrix}
    1 & 2 &|& 2 \\
    0 & -1 &|& -2 \\
    0 & 0 &|& k - 2
\end{bmatrix}
\]
\[
R_1 = R_1 - 2R_2 = 
\begin{bmatrix}
    1 & 0 &|& 6 \\
    0 & -1 &|& -2 \\
    0 & 0 &|& k - 2
\end{bmatrix}
R_2 = -R_2
\begin{bmatrix}
    1 & 0 &|& 6 \\
    0 & 1 &|& 2 \\
    0 & 0 &|& k - 2
\end{bmatrix}
\]
a = 6, b = 2, k = 2
K = \text{2}

\item Is the set \(\{\mathbf{u}, \mathbf{v}, \mathbf{x}\}\) a basis of \(\mathbb{R}^3\)? Explain why or why not.\\
yes there are linearly independent and the rank of the matrix whose columns are \( u, v, x \) is 3. Therefore, the set \(\{u, v, x\}\) is a basis of \(\mathbb{R}^3\).

\item If \( B = \{\mathbf{u}, \mathbf{v}, \mathbf{x}\} \) is a basis of \(\mathbb{R}^3\), express \(\begin{bmatrix} -1 \\ -6 \\ 6 \end{bmatrix}\) as a linear combination of \( B \).\\
\[
u = \begin{bmatrix}
    3 \\
    2 \\
    3   
\end{bmatrix}, \quad
v = \begin{bmatrix}
    5 \\
    3 \\
    4
\end{bmatrix}, \quad
x = \begin{bmatrix}
    3 \\
    1 \\
    2
\end{bmatrix}
\]
\[
\begin{aligned}
    x\begin{bmatrix}
        3 \\
        2 \\
        3
    \end{bmatrix} +y\begin{bmatrix}
        5 \\
        3 \\
        4
    \end{bmatrix} + z\begin{bmatrix}
        3 \\
        1 \\
        2
    \end{bmatrix} &= \begin{bmatrix}
        -1 \\
        -6 \\
        6
    \end{bmatrix}
\end{aligned}
\]
\[
\begin{aligned}
    3x + 5y + 3z &= -1 \\
    2x + 3y + z &= -6 \\
    3x + 4y + 2z &= 6
\end{aligned}
\]
\[
\begin{aligned}
    \begin{bmatrix}
        3 & 5 & 3 \\
        2 & 3 & 1 \\
        3 & 4 & 2
    \end{bmatrix}
    \begin{bmatrix}
        x \\
        y \\
        z
    \end{bmatrix}
    &= \begin{bmatrix}
        -1 \\
        -6 \\
        6
    \end{bmatrix}
\end{aligned}
\]
the augumented matrix:
\[
\begin{bmatrix}
    3 & 5 & 3 &|& -1 \\
    2 & 3 & 1 &|& -6 \\
    3 & 4 & 2 &|& 6
\end{bmatrix}
\]
\[
R_1 = R_1 - R_2
\begin{bmatrix}
    1 & 2 & 2 &|& 5\\
    2 & 3 & 1 &|& -6 \\
    3 & 4 & 2 &|& 6
\end{bmatrix}
R_2 = R_2 - 2R_1
\begin{bmatrix}
    1 & 2 & 2 &|& 5 \\
    0 & -1 & -3 &|& -16 \\
    3 & 4 & 2 &|& 6
\end{bmatrix}
\]
\[
R_3 = R_3 - 3R_1
\begin{bmatrix}
    1 & 2 & 2 &|& 5 \\
    0 & -1 & -3 &|& -16 \\
    0 & -2 & -4 &|& -9
\end{bmatrix}
R_3 = R_3 - 2R_2
\begin{bmatrix}
    1 & 2 & 2 &|& 5 \\
    0 & -1 & -3 &|& -16 \\
    0 & 0 & 2 &|& 23
\end{bmatrix}
\]
\[
R_3 = \frac{1}{2}R_3
\begin{bmatrix}
    1 & 2 & 2 &|& 5 \\
    0 & -1 & -3 &|& -16 \\
    0 & 0 & 1 &|& 11.5
\end{bmatrix}
R_2 = -R_2
\begin{bmatrix}
    1 & 2 & 2 &|& 5 \\
    0 & 1 & 3 &|& 16 \\
    0 & 0 & 1 &|& 11.5
\end{bmatrix}
R_2 = R_2 - 3R_3
\begin{bmatrix}
    1 & 2 & 2 &|& 5 \\
    0 & 1 & 0 &|& -18.5 \\
    0 & 0 & 1 &|& 11.5
\end{bmatrix}
\]
\[
R_1 = R_1 - 2R_3
\begin{bmatrix}
    1 & 2 & 0 &|& -18 \\
    0 & 1 & 0 &|& -18.5 \\
    0 & 0 & 1 &|& 11.5
\end{bmatrix}
R_1 = R_1 - 2R_2
\begin{bmatrix}
    1 & 0 & 0 &|& 19 \\
    0 & 1 & 0 &|& -18.5 \\
    0 & 0 & 1 &|& 11.5
\end{bmatrix}
\]
The linear combination of \( B \) is \( 19u - 18.5v + 11.5x \) = \(\begin{bmatrix} -1 \\ -6 \\ 6 \end{bmatrix}\)
\end{enumerate}
\end{enumerate}

\vspace{0.5cm} % Add some vertical space after the text

\end{document} 