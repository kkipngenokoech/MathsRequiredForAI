\documentclass[11pt]{article}

\usepackage{amsmath,amssymb,amsfonts}
\usepackage{graphicx}

\usepackage{geometry}
\geometry{a3paper, margin=0.5in} % Adjust the margin as needed


\setlength{\topmargin}{-.5in} \setlength{\textheight}{9.25in}
\setlength{\oddsidemargin}{0in} \setlength{\textwidth}{6.8in}


\begin{document}

\Large


\noindent{\bf Kipngeno koech - bkoech  Assignment 1\hfill 18 September 2024}


\medskip\hrule
\begin{enumerate}

\item Consider the system of linear equations below:
\[
\begin{aligned}
    3x + y &= 6 \\
    -2x + 2y + 8z &= -8 \\
    4x + 4y + 8z &= 4
\end{aligned}
\]

\begin{enumerate}
    \item Express the system in a compact matrix form, Ax = b 
    \[
        A = \begin{bmatrix}
            3 & 1 & 0 \\
            -2 & 2 & 8 \\
            4 & 4 & 8
        \end{bmatrix}
        x = \begin{bmatrix}
            x \\
            y \\
            z
        \end{bmatrix}
        b = \begin{bmatrix}
            6 \\
            -8 \\
            4
        \end{bmatrix}\newline
    \]  
    \text{To represent the system in a compact matrix form:}
    \[
        \begin{bmatrix}
            3 & 1 & 0 \\
            -2 & 2 & 8 \\
            4 & 4 & 8
        \end{bmatrix}
        \begin{bmatrix}
            x \\
            y \\
            z
        \end{bmatrix}
        = \begin{bmatrix}
            6 \\
            -8 \\
            4
        \end{bmatrix}
    \]
    \item \textbf{Use Gaussian elimination to determine if the system has no solution, one unique solution or infinitely many solutions and justify your answer}
    \[
        \begin{bmatrix}
            3 & 1 & 0 &|& 6 \\
            -2 & 2 & 8 &|& -8 \\
            4 & 4 & 8 & |&4
        \end{bmatrix}
        R_1 \leftrightarrow R_2
        \begin{bmatrix}
            -2 & 2 & 8 &|& -8 \\
            3 & 1 & 0 &|& 6 \\
            4 & 4 & 8 & |&4
        \end{bmatrix}
        R_1 = R_1 + R_2
        \begin{bmatrix}
            1 & 3 & 8 &|& -2 \\
            3 & 1 & 0 &|& 6 \\
            4 & 4 & 8 & |&4
        \end{bmatrix}
    \]
    \[
        R_2 = R_2 - 3R_1
        \begin{bmatrix}
            1 & 3 & 8 &|& -2 \\
            0 & -8 & -24 &|& 12 \\
            4 & 4 & 8 & |&4
        \end{bmatrix}
        R_3 = R_3 - 4R_1
        \begin{bmatrix}
            1 & 3 & 8 &|& -2 \\
            0 & -8 & -24 &|& 12 \\
            0 & -8 & -24 & |&12
        \end{bmatrix}
    \]

    \[
        R_3 = R_3 - R_2
        \begin{bmatrix}
            1 & 3 & 8 &|& -2 \\
            0 & -8 & -24 &|& 12 \\
            0 & 0 & 0 & |&0
        \end{bmatrix}
        R_2 = -\dfrac{1}{8}R_2
        \begin{bmatrix}
            1 & 3 & 8 &|& -2 \\
            0 & 1 & 3 &|& -1.5 \\
            0 & 0 & 0 & |&0
        \end{bmatrix}
    \]
    This system has infinitely many solutions since the last row of the matrix is all zeros. The system is consistent.
    \item \textbf{If the system has a solution, what is the solution?}
    \[
        \begin{aligned}
            x + 3y + 8z &= -2 \\
            y + 3z &= -1.5
        \end{aligned}
    \]
    \[
        \begin{aligned}
            x &= -2 - 3y - 8z \\
            y &= -1.5 - 3z
        \end{aligned}
    \]
    \[
        \begin{aligned}
            x &= -2 - 3(-1.5 - 3z) - 8z \\
            x &= -2 + 4.5 + 9z - 8z \\
            x &= 2.5 + z
        \end{aligned}
    \]
    The solution to the system is:
    \[
        \begin{aligned}
            x &= 2.5 + z \\
            y &= -1.5 - 3z \\
            z &= z
        \end{aligned}
    \]
    where Z is a free variable and can take any value.
\end{enumerate}

\item \textbf{Determine whether the following systems of equations (or matrix equations) described below have no solution, one unique solution, or infinitely many solutions, and justify your answer.}

\begin{enumerate}
    \item \textbf{}
    \[
        \begin{aligned}
            ax + by &= c \\
            dx + ey &= f
        \end{aligned}
    \]
    where \(a, b, c, d, e, f\) are scalars satisfying \(\frac{a}{d} = \frac{b}{e} = \frac{c}{f}\)
    \[
        \begin{aligned}
            \frac{a}{d} = \frac{b}{c} = \frac{c}{f} \\
            e = \frac{bd}{a}
        \end{aligned}
    \]
    \[
        \begin{bmatrix}
            a & b \\
            d & e
        \end{bmatrix}
        \begin{bmatrix}
            x \\
            y
        \end{bmatrix}
        = \begin{bmatrix}
            c \\
            f
        \end{bmatrix}
    \]

    \[
        \begin{bmatrix}
            a & b \\
            d & \frac{bd}{a}
        \end{bmatrix}
        \begin{bmatrix}
            x \\
            y
        \end{bmatrix}
        = \begin{bmatrix}
            c \\
            f
        \end{bmatrix}
    \]
    in matrix argumented form:
    \[
        \begin{bmatrix}
            a & b &|& c \\
            d & \frac{bd}{a} &|& f
        \end{bmatrix}
        R_1 = \frac{1}{a}R_1
        \begin{bmatrix}
            1 & \frac{b}{a} &|& \frac{c}{a} \\
            d & \frac{bd}{a} &|& f
        \end{bmatrix}
        R_2 = R_2 - dR_1
        \begin{bmatrix}
            1 & \frac{b}{a} &|& \frac{c}{a} \\
            0 & 0 &|& f - d\frac{c}{a}
        \end{bmatrix}
    \]

    The system has no solution if \(f - d\frac{c}{a} \neq 0\), or infinitely many solutions if \(f - d\frac{c}{a} = 0\)
  
        
    \item A homogeneous system of 3 equations in 4 unknowns
    \[
        \begin{aligned}
            ax + by + cz + dw &= 0 \\
            ex + fy + gz + hw &= 0 \\
            ix + jy + kz + lw &= 0
        \end{aligned}
    \]
    \[
        \begin{bmatrix}
            a & b & c & d \\
            e & f & g & h \\
            i & j & k & l
        \end{bmatrix}
        \begin{bmatrix}
            x \\
            y \\
            z \\
            w
        \end{bmatrix}
        = \begin{bmatrix}
            0 \\
            0 \\
            0
        \end{bmatrix}
    \]
    The system has infinitely many solutions since there will be a lot of free variables.
    \item \textbf{Ax = b, where the row-reduced echelon form of the augmented matrix [A | b] looks as follows:}
    \[
        \begin{bmatrix}
            1 & 0 & -1 &|& 0 \\
            0 & 1 & 2 &|& 0 \\
            0 & 0 & 0 &|& 1
        \end{bmatrix}
    \]
    The system has no solution since the last row of the matrix is all zeros with the last element being 1, this means that the system is inconsistent.
\end{enumerate}

\item Given the matrix A:
\[
    A = \begin{bmatrix}
        0 & 1 & 1 \\
        1 & 0 & -1 \\
        -2 & 1 & 0 \\
        -1 & 1 & 1
    \end{bmatrix}
\]
\begin{enumerate}
\item Compute \(A^\top A\) and show that it is symmetric
\[
    A^\top A = \begin{bmatrix}
        0 & 1 & -2 & -1 \\
        1 & 0 & 1 & 1 \\
        1 & -1 & 0 & 1
    \end{bmatrix}
    \begin{bmatrix}
        0 & 1 & 1 \\
        1 & 0 & -1 \\
        -2 & 1 & 0 \\
        -1 & 1 & 1
    \end{bmatrix}
    = \begin{bmatrix}
        6 & -1 & -1 \\
        -1 & 4 & 0 \\
        -1 & 0 & 4
    \end{bmatrix}
\]

To show that it is symmetric: \(A^\top A = (A^\top A)^\top\)
\[
    \begin{bmatrix}
        6 & -1 & -1 \\
        -1 & 4 & 0 \\
        -1 & 0 & 4
    \end{bmatrix}
    = \begin{bmatrix}
        6 & -1 & -1 \\
        -1 & 4 & 0 \\
        -1 & 0 & 4
    \end{bmatrix}
\]

\item Compute \((A^\top A)^{-1}\) using Gaussian Elimination
\[
    \begin{bmatrix}
        6 & -1 & -1 &|& 1 & 0 & 0 \\
        -1 & 4 & 0 &|& 0 & 1 & 0 \\
        -1 & 0 & 4 &|& 0 & 0 & 1
    \end{bmatrix}
    R_1 = \frac{1}{6}R_1
    \begin{bmatrix}
        1 & -\frac{1}{6} & -\frac{1}{6} &|& \frac{1}{6} & 0 & 0 \\
        -1 & 4 & 0 &|& 0 & 1 & 0 \\
        -1 & 0 & 4 &|& 0 & 0 & 1
    \end{bmatrix}
\]
\[
    R_2 = R_2 + R_1
    \begin{bmatrix}
        1 & -\frac{1}{6} & -\frac{1}{6} &|& \frac{1}{6} & 0 & 0 \\
        0 & \frac{23}{6} & -\frac{1}{6} &|& \frac{1}{6} & 1 & 0 \\
        -1 & 0 & 4 &|& 0 & 0 & 1
    \end{bmatrix}
    R_3 = R_3 + R_1
    \begin{bmatrix}
        1 & -\frac{1}{6} & -\frac{1}{6} &|& \frac{1}{6} & 0 & 0 \\
        0 & \frac{23}{6} & -\frac{1}{6} &|& \frac{1}{6} & 1 & 0 \\
        0 & -\frac{1}{6} & \frac{17}{6} &|& \frac{1}{6} & 0 & 1
    \end{bmatrix}
\]

\[
    R_2 = \frac{6}{23}R_2
    \begin{bmatrix}
        1 & -\frac{1}{6} & -\frac{1}{6} &|& \frac{1}{6} & 0 & 0 \\
        0 & 1 & -\frac{1}{23} &|& \frac{1}{23} & \frac{6}{23} & 0 \\
        0 & -\frac{1}{6} & \frac{17}{6} &|& \frac{1}{6} & 0 & 1
    \end{bmatrix}
    R_1 = R_1 + \frac{1}{6}R_2
    \begin{bmatrix}
        1 & 0 & -\frac{1}{23} &|& \frac{1}{23} & \frac{1}{6} & 0 \\
        0 & 1 & -\frac{1}{23} &|& \frac{1}{23} & \frac{6}{23} & 0 \\
        0 & -\frac{1}{6} & \frac{17}{6} &|& \frac{1}{6} & 0 & 1
    \end{bmatrix}
\]

\[
    R_3 = R_3 + \frac{1}{6}R_2
    \begin{bmatrix}
        1 & 0 & -\frac{1}{23} &|& \frac{1}{23} & \frac{1}{6} & 0 \\
        0 & 1 & -\frac{1}{23} &|& \frac{1}{23} & \frac{6}{23} & 0 \\
        0 & 0 & \frac{17}{23} &|& \frac{1}{23} & \frac{1}{6} & 1
    \end{bmatrix}
    R_3 = \frac{23}{17}R_3
    \begin{bmatrix}
        1 & 0 & -\frac{1}{23} &|& \frac{1}{23} & \frac{1}{6} & 0 \\
        0 & 1 & -\frac{1}{23} &|& \frac{1}{23} & \frac{6}{23} & 0 \\
        0 & 0 & 1 &|& \frac{1}{17} & \frac{3}{17} & \frac{23}{17}
    \end{bmatrix}
\]

\[
    R_1 = R_1 + \frac{1}{23}R_3
    \begin{bmatrix}
        1 & 0 & 0 &|& \frac{1}{17} & \frac{5}{17} & \frac{23}{17} \\
        0 & 1 & -\frac{1}{23} &|& \frac{1}{23} & \frac{6}{23} & 0 \\
        0 & 0 & 1 &|& \frac{1}{17} & \frac{3}{17} & \frac{23}{17}
    \end{bmatrix}
    R_2 = R_2 + \frac{1}{23}R_3
    \begin{bmatrix}
        1 & 0 & 0 &|& \frac{1}{17} & \frac{5}{17} & \frac{23}{17} \\
        0 & 1 & 0 &|& \frac{2}{17} & \frac{9}{17} & \frac{23}{17} \\
        0 & 0 & 1 &|& \frac{1}{17} & \frac{3}{17} & \frac{23}{17}
    \end{bmatrix}
\]

\[
    (A^\top A)^{-1} = \begin{bmatrix}
        \frac{1}{17} & \frac{5}{17} & \frac{23}{17} \\
        \frac{2}{17} & \frac{9}{17} & \frac{23}{17} \\
        \frac{1}{17} & \frac{3}{17} & \frac{23}{17}
    \end{bmatrix}
\]

\item  Let \( b = \begin{bmatrix}
    1 \\
    0 \\
    1 \\
    0
\end{bmatrix} \). If \( Ax = b \), show that \( x = (A^\top A)^{-1}A^\top b \) and obtain the value of \( x \). [Note: \( x = (A^\top A)^{-1}A^\top b \) is called the Normal equation]

\[
    Ax = b, 
    where, A = A^\top A = \begin{bmatrix}
        6 & -1 & -1 \\
        -1 & 4 & 0 \\
        -1 & 0 & 4
    \end{bmatrix}
    b = \begin{bmatrix}
        1 \\
        0 \\
        1 \\
        0
    \end{bmatrix}
\]

\[
    Ax=b, (A^\top A)x =  b
\]
\[
    (A^\top A)^{-1}(A^\top A)x = (A^\top A)^{-1}b
\]
\[
    x = (A^\top A)^{-1}b
\]
\[
    x = \begin{bmatrix}
        \frac{1}{17} & \frac{5}{17} & \frac{23}{17} \\
        \frac{2}{17} & \frac{9}{17} & \frac{23}{17} \\
        \frac{1}{17} & \frac{3}{17} & \frac{23}{17}
    \end{bmatrix}
    \begin{bmatrix}
        1 \\
        0 \\
        1
    \end{bmatrix}
\]
\[
    x =
    \begin{bmatrix}
        24/17 \\
        25/17 \\
        24/17
    \end{bmatrix}
\]

\end{enumerate}

\item CMU-Africa is trying to understand the pricing strategy used by Delight Canteen. The Canteen sells various types of food, each with a different price, but only charges a total price for all food types on a student’s plate. As a machine learning engineer, you want to help students determine the price of their food based on the quantity (measured in Grams) of each foot type they add to their plate. In other to achieve this, you collect data from 6 of your friends on the quantity of each food type they served and the total price. The observations from your friends are recorded below.

\begin{tabular}{|c|c|c|c|c|c|}
    \hline
    Transaction & Food a (g) & Food b (g) & Food c (g) & Food d (g) & Total Cost (RWF) \\
    \hline
    1 & 100 & 50 & 150 & 200 & 2500 \\
    2 & 50 & 50 & 100 & 300 & 2300 \\
    3 & 100 & 150 & 200 & 100 & 3000 \\
    4 & 50 & 200 & 300 & 50 & 2900 \\
    5 & 200 & 50 & 250 & 50 & 3100 \\
    6 & 300 & 50 & 50 & 200 & 4300 \\
    \hline
\end{tabular}

From the observations in the table above, you are expected to obtain the price per quantity of each food
type (mearsured in RWF/g).

\begin{enumerate}
\item Describe the above using a system of linear equations
\[
    \begin{aligned}
        100a + 50b + 150c + 200d &= 2500 \\
        50a + 50b + 100c + 300d &= 2300 \\
        100a + 150b + 200c + 100d &= 3000 \\
        50a + 200b + 300c + 50d &= 2900 \\
        200a + 50b + 250c + 50d &= 3100 \\
        300a + 50b + 50c + 200d &= 4300
    \end{aligned}
\]
\item Write the system of linear equations in a compact matrix form, Ax = b
\[
    A = \begin{bmatrix}
        100 & 50 & 150 & 200 \\
        50 & 50 & 100 & 300 \\
        100 & 150 & 200 & 100 \\
        50 & 200 & 300 & 50 \\
        200 & 50 & 250 & 50 \\
        300 & 50 & 50 & 200
    \end{bmatrix}
    x = \begin{bmatrix}
        a \\
        b \\
        c \\
        d
    \end{bmatrix}
    b = \begin{bmatrix}
        2500 \\
        2300 \\
        3000 \\
        2900 \\
        3100 \\
        4300
    \end{bmatrix}
\]
\[
    Ax = b = \begin{bmatrix}
        100 & 50 & 150 & 200 \\
        50 & 50 & 100 & 300 \\
        100 & 150 & 200 & 100 \\
        50 & 200 & 300 & 50 \\
        200 & 50 & 250 & 50 \\
        300 & 50 & 50 & 200
    \end{bmatrix}
    \begin{bmatrix}
        a \\
        b \\
        c \\
        d
    \end{bmatrix}
    = \begin{bmatrix}
        2500 \\
        2300 \\
        3000 \\
        2900 \\
        3100 \\
        4300
    \end{bmatrix}
\]
\item Use Gaussian elimination to obtain a solution for the system.
\[
    \begin{bmatrix}
        100 & 50 & 150 & 200 &|& 2500 \\
        50 & 50 & 100 & 300 &|& 2300 \\
        100 & 150 & 200 & 100 &|& 3000 \\
        50 & 200 & 300 & 50 &|& 2900 \\
        200 & 50 & 250 & 50 &|& 3100 \\
        300 & 50 & 50 & 200 &|& 4300
    \end{bmatrix}
    R_1 = \frac{1}{100}R_1
    \begin{bmatrix}
        1 & 0.5 & 1.5 & 2 &|& 25 \\
        50 & 50 & 100 & 300 &|& 2300 \\
        100 & 150 & 200 & 100 &|& 3000 \\
        50 & 200 & 300 & 50 &|& 2900 \\
        200 & 50 & 250 & 50 &|& 3100 \\
        300 & 50 & 50 & 200 &|& 4300
    \end{bmatrix}
\]

\[
    R_2 = R_2 - 50R_1
    \begin{bmatrix}
        1 & 0.5 & 1.5 & 2 &|& 25 \\
        0 & 25 & 25 & 200 &|& 1050 \\
        100 & 150 & 200 & 100 &|& 3000 \\
        50 & 200 & 300 & 50 &|& 2900 \\
        200 & 50 & 250 & 50 &|& 3100 \\
        300 & 50 & 50 & 200 &|& 4300
    \end{bmatrix}
    R_3 = R_3 - 100R_1
    \begin{bmatrix}
        1 & 0.5 & 1.5 & 2 &|& 25 \\
        0 & 25 & 25 & 200 &|& 1050 \\
        0 & 100 & 50 & -100 &|& 2000 \\
        50 & 200 & 300 & 50 &|& 2900 \\
        200 & 50 & 250 & 50 &|& 3100 \\
        300 & 50 & 50 & 200 &|& 4300
    \end{bmatrix}
\]

\[
    R_4 = R_4 - 50R_1
    \begin{bmatrix}
        1 & 0.5 & 1.5 & 2 &|& 25 \\
        0 & 25 & 25 & 200 &|& 1050 \\
        0 & 100 & 50 & -100 &|& 2000 \\
        0 & 150 & 225 & -100 &|& 1650 \\
        200 & 50 & 250 & 50 &|& 3100 \\
        300 & 50 & 50 & 200 &|& 4300
    \end{bmatrix}
    R_5 = R_5 - 200R_1
    \begin{bmatrix}
        1 & 0.5 & 1.5 & 2 &|& 25 \\
        0 & 25 & 25 & 200 &|& 1050 \\
        0 & 100 & 50 & -100 &|& 2000 \\
        0 & 150 & 225 & -100 &|& 1650 \\
        0 & 0 & -200 & -350 &|& 2600 \\
        300 & 50 & 50 & 200 &|& 4300
    \end{bmatrix}
\]

\[
    R_6 = R_6 - 300R_1
    \begin{bmatrix}
        1 & 0.5 & 1.5 & 2 &|& 25 \\
        0 & 25 & 25 & 200 &|& 1050 \\
        0 & 100 & 50 & -100 &|& 2000 \\
        0 & 150 & 225 & -100 &|& 1650 \\
        0 & 0 & -200 & -350 &|& 2600 \\
        0 & -100 & -350 & -400 &|& 3550
    \end{bmatrix}
    R_6 = R_6 + R_2
    \begin{bmatrix}
        1 & 0.5 & 1.5 & 2 &|& 25 \\
        0 & 25 & 25 & 200 &|& 1050 \\
        0 & 100 & 50 & -100 &|& 2000 \\
        0 & 150 & 225 & -100 &|& 1650 \\
        0 & 0 & -200 & -350 &|& 2600 \\
        0 & 0 & -325 & -200 &|& 4600
    \end{bmatrix}
\]

\[
    R_6 = R_6 + 4R_3
    \begin{bmatrix}
        1 & 0.5 & 1.5 & 2 &|& 25 \\
        0 & 25 & 25 & 200 &|& 1050 \\
        0 & 100 & 50 & -100 &|& 2000 \\
        0 & 150 & 225 & -100 &|& 1650 \\
        0 & 0 & -200 & -350 &|& 2600 \\
        0 & 0 & 0 & -800 &|& 6600
    \end{bmatrix}
    R_6 = \frac{1}{-800}R_6
    \begin{bmatrix}
        1 & 0.5 & 1.5 & 2 &|& 25 \\
        0 & 25 & 25 & 200 &|& 1050 \\
        0 & 100 & 50 & -100 &|& 2000 \\
        0 & 150 & 225 & -100 &|& 1650 \\
        0 & 0 & -200 & -350 &|& 2600 \\
        0 & 0 & 0 & 1 &|& 8.25
    \end{bmatrix}
\]

\[
    R_5 = R_5 + 2R_6
    \begin{bmatrix}
        1 & 0.5 & 1.5 & 2 &|& 25 \\
        0 & 25 & 25 & 200 &|& 1050 \\
        0 & 100 & 50 & -100 &|& 2000 \\
        0 & 150 & 225 & -100 &|& 1650 \\
        0 & 0 & -200 & 0 &|& 2766.5 \\
        0 & 0 & 0 & 1 &|& 8.25
    \end{bmatrix}
    R_4 = R_4 + 2R_6
    \begin{bmatrix}
        1 & 0.5 & 1.5 & 2 &|& 25 \\
        0 & 25 & 25 & 200 &|& 1050 \\
        0 & 100 & 50 & -100 &|& 2000 \\
        0 & 150 & 225 & 0 &|& 2782.5 \\
        0 & 0 & -200 & 0 &|& 2766.5 \\
        0 & 0 & 0 & 1 &|& 8.25
    \end{bmatrix}
\]

\[
    R_3 = R_3 + R_6
    \begin{bmatrix}
        1 & 0.5 & 1.5 & 2 &|& 25 \\
        0 & 25 & 25 & 200 &|& 1050 \\
        0 & 100 & 50 & 0 &|& 2782.5 \\
        0 & 150 & 225 & 0 &|& 2782.5 \\
        0 & 0 & -200 & 0 &|& 2766.5 \\
        0 & 0 & 0 & 1 &|& 8.25
    \end{bmatrix}
    R_2 = R_2 - R_6
    \begin{bmatrix}
        1 & 0.5 & 1.5 & 2 &|& 25 \\
        0 & 25 & 25 & 0 &|& 1041.75 \\
        0 & 100 & 50 & 0 &|& 2782.5 \\
        0 & 150 & 225 & 0 &|& 2782.5 \\
        0 & 0 & -200 & 0 &|& 2766.5 \\
        0 & 0 & 0 & 1 &|& 8.25
    \end{bmatrix}
\]

\[
    R_1 = R_1 - 2R_6
    \begin{bmatrix}
        1 & 0.5 & 1.5 & 0 &|& 8.5 \\
        0 & 25 & 25 & 0 &|& 1041.75 \\
        0 & 100 & 50 & 0 &|& 2782.5 \\
        0 & 150 & 225 & 0 &|& 2782.5 \\
        0 & 0 & -200 & 0 &|& 2766.5 \\
        0 & 0 & 0 & 1 &|& 8.25
    \end{bmatrix}
    R_1 = R_1 - 0.5R_2
    \begin{bmatrix}
        1 & 0 & 1 & 0 &|& 8.5 \\
        0 & 25 & 25 & 0 &|& 1041.75 \\
        0 & 100 & 50 & 0 &|& 2782.5 \\
        0 & 150 & 225 & 0 &|& 2782.5 \\
        0 & 0 & -200 & 0 &|& 2766.5 \\
        0 & 0 & 0 & 1 &|& 8.25
    \end{bmatrix}
\]

\[
    R_3 = R_3 - 4R_2
    \begin{bmatrix}
        1 & 0 & 1 & 0 &|& 8.5 \\
        0 & 25 & 25 & 0 &|& 1041.75 \\
        0 & 0 & -50 & 0 &|& 1778.5 \\
        0 & 150 & 225 & 0 &|& 2782.5 \\
        0 & 0 & -200 & 0 &|& 2766.5 \\
        0 & 0 & 0 & 1 &|& 8.25
    \end{bmatrix}
    R_4 = R_4 - 6R_2
    \begin{bmatrix}
        1 & 0 & 1 & 0 &|& 8.5 \\
        0 & 25 & 25 & 0 &|& 1041.75 \\
        0 & 0 & -50 & 0 &|& 1778.5 \\
        0 & 0 & 75 & 0 &|& 1738.5 \\
        0 & 0 & -200 & 0 &|& 2766.5 \\
        0 & 0 & 0 & 1 &|& 8.25
    \end{bmatrix}
\]

\[
    R_3 = -\frac{1}{50}R_3
    \begin{bmatrix}
        1 & 0 & 1 & 0 &|& 8.5 \\
        0 & 25 & 25 & 0 &|& 1041.75 \\
        0 & 0 & 1 & 0 &|& 35.57 \\
        0 & 0 & 75 & 0 &|& 1738.5 \\
        0 & 0 & -200 & 0 &|& 2766.5 \\
        0 & 0 & 0 & 1 &|& 8.25
    \end{bmatrix}
    R_4 = R_4 - 75R_3
    \begin{bmatrix}
        1 & 0 & 1 & 0 &|& 8.5 \\
        0 & 25 & 25 & 0 &|& 1041.75 \\
        0 & 0 & 1 & 0 &|& 35.57 \\
        0 & 0 & 0 & 0 &|& -1.25 \\
        0 & 0 & -200 & 0 &|& 2766.5 \\
        0 & 0 & 0 & 1 &|& 8.25
    \end{bmatrix}
\]

\[
    R_5 = R_5 + 200R_3
    \begin{bmatrix}
        1 & 0 & 1 & 0 &|& 8.5 \\
        0 & 25 & 25 & 0 &|& 1041.75 \\
        0 & 0 & 1 & 0 &|& 35.57 \\
        0 & 0 & 0 & 0 &|& -1.25 \\
        0 & 0 & 0 & 0 &|& 2766.5 \\
        0 & 0 & 0 & 1 &|& 8.25
    \end{bmatrix}
    R_2 = R_2 - 25R_3
    \begin{bmatrix}
        1 & 0 & 1 & 0 &|& 8.5 \\
        0 & 25 & 0 & 0 &|& 963.18 \\
        0 & 0 & 1 & 0 &|& 35.57 \\
        0 & 0 & 0 & 0 &|& -1.25 \\
        0 & 0 & 0 & 0 &|& 2766.5 \\
        0 & 0 & 0 & 1 &|& 8.25
    \end{bmatrix}
\]

\[
    R_1 = R_1 - R_3
    \begin{bmatrix}
        1 & 0 & 0 & 0 &|& -27.07 \\
        0 & 25 & 0 & 0 &|& 963.18 \\
        0 & 0 & 1 & 0 &|& 35.57 \\
        0 & 0 & 0 & 0 &|& -1.25 \\
        0 & 0 & 0 & 0 &|& 2766.5 \\
        0 & 0 & 0 & 1 &|& 8.25
    \end{bmatrix}
    R_1 = R_1 + R_2
    \begin{bmatrix}
        1 & 0 & 0 & 0 &|& 935.11 \\
        0 & 25 & 0 & 0 &|& 963.18 \\
        0 & 0 & 1 & 0 &|& 35.57 \\
        0 & 0 & 0 & 0 &|& -1.25 \\
        0 & 0 & 0 & 0 &|& 2766.5 \\
        0 & 0 & 0 & 1 &|& 8.25
    \end{bmatrix}
\]



\item Find the least squares line for the linearized data set \hfill[8]

\item Predict the bacteria population after 10 minutes.\hfill[5]


Note: Answers should be correct to four decimal places for this item.

\end{enumerate}

\begin{center}
$m=\dfrac{n\sum xy -\sum x\sum y}{n\sum x^2 -(\sum x)^2}$

$b=\dfrac{\sum x^2\sum y - \sum x\sum xy}{n\sum x^2 -(\sum x)^2}$

\end{center}







\end{enumerate}
\vspace{0.5cm} % Add some vertical space after the text

\end{document} 