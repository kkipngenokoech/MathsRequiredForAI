\documentclass{article}
\usepackage{amsmath}

\begin{document}

\title{Height Calculation of a Pile of Paper}
\author{}
\date{}
\maketitle

\section*{Problem Statement}
Given that the height of a pile of 500 sheets of paper is 4.5 cm, calculate the height of a pile of 79 million sheets of paper.

\section*{Solution}
Let \( h \) be the height of 79 million sheets of paper. We can set up a proportion based on the given data:

\[
\frac{4.5 \text{ cm}}{500 \text{ sheets}} = \frac{h \text{ cm}}{79,000,000 \text{ sheets}}
\]

Solving for \( h \):

\[
h = \frac{4.5 \text{ cm} \times 79,000,000 \text{ sheets}}{500 \text{ sheets}}
\]

\[
h = \frac{4.5 \times 79,000,000}{500}
\]

\[
h = \frac{355,500,000}{500}
\]

\[
h = 711,000 \text{ cm}
\]

\section*{Conclusion}
The height of a pile of 79 million sheets of paper is 711,000 cm, or 7,110 meters.

\end{document}